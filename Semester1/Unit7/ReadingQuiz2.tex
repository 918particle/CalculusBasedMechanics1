\documentclass{article}
\usepackage{graphicx}
\usepackage[margin=1.5cm]{geometry}
\usepackage{amsmath}

\begin{document}
\twocolumn
\small

\title{Friday Reading Assessment: Unit 4, Power and Conservation of Energy}
\author{Prof. Jordan C. Hanson}

\maketitle

\section{Memory Bank}

\begin{itemize}
\item \textbf{Conservative force:}
\begin{equation}
\vec{F} = -\nabla U(x,y,z) = -\frac{\partial U}{\partial x}\hat{i}-\frac{\partial U}{\partial y}\hat{j}-\frac{\partial U}{\partial z}\hat{k}
\end{equation}
\item The \textit{curl} of a vector field:
\begin{multline}
\nabla \times \vec{F} = \\ \left(\frac{\partial F_z}{\partial y}-\frac{\partial F_y}{\partial z}\right) \hat{i} - \left(\frac{\partial F_z}{\partial x}-\frac{\partial F_x}{\partial z}\right) \hat{j} + \left(\frac{\partial F_y}{\partial x}-\frac{\partial F_x}{\partial y}\right) \hat{k}
\end{multline}
\item \textbf{Conservative force:}
\begin{equation}
\nabla \times \vec{F} = 0
\end{equation}
\item \textbf{Gravitational force:}
\begin{equation}
\vec{F}_{\rm G} = G \frac{m_1 m_2}{r^2}\hat{r}
\end{equation}
\item \textbf{Gravitational potential energy:}
\begin{equation}
U(r) = - G \frac{m_1 m_2}{r} \label{eq:pot}
\end{equation}
\end{itemize}

\begin{figure}
\centering
\includegraphics[width=0.35\textwidth]{figures/orbit_phet.png}
\caption{\label{fig:work} Gravity and orbits PhET.}
\end{figure}

\section{Conservative Forces and Gravity}

\begin{enumerate}
\item The \textit{partial derivative} $\partial U/\partial x$ is the derivative with respect to one variable while holding the others constant. (a) Suppose $U(x,y) = \frac{1}{2}k(x^2+y^2)$.  What is $-\nabla U$?  (b) Suppose $U(x) = -k x^{-1}$.  What is $-dU/dx$? \\ \vspace{3cm}
\item A \textit{conservative force} obeys the following rule:
\begin{equation}
\nabla \times \vec{F} = 0
\end{equation}
This is called the \textit{curl} of the force.  Suppose the force $\vec{F}$ is restricted to the xy-plane, meaning neither the x-component nor the y-component depends on $z$, and there is no $z$ component.  Suppose $\nabla \times \vec{F} = 0$.  Using the definition of the curl, show that
\begin{equation}
\frac{dF_x}{dy} = \frac{dF_y}{dx}
\end{equation}
Suppose $\vec{F} = -k(x\hat{i} + y\hat{j})$.  Does this force conserve energy?  What type of system is described by such a force? \\ \vspace{2.5cm}
\item The force of gravity between objects of masses $m_1$ and $m_2$, separated by $r$, is given by
\begin{equation}
\vec{F}_{\rm G} = G \frac{m_1 m_2}{r^2}\hat{r}
\end{equation}
The force is interpreted as \textit{attractive} if it is positive, and $G = 6.674 \times 10^{-11}$ N kg$^{-2}$ m$^{2}$. (a) What is the force of gravity between the Earth and Moon?  Look up the masses in kg, and let $r=384400$ km. (b) Let $F_{\rm C} = F_{\rm G}$.  Show that $r^3 \propto T^2$. (c) Work out the period of the Moon around the Earth. \\ \vspace{4cm}
\item (a) Show that, if $F = -dU(r)/dr$, that the potential energy of the gravitational force is given by Eq. \ref{eq:pot}.  (b) What is the change in potential energy if an object goes from orbital radius $r_A$ to $r_B$? (c) For a circular orbit, prove that total energy is constant.
\end{enumerate}

\end{document}
\title{Rubric for the Final Project, Physics}
\author{Dr. Jordan Hanson - Whittier College Dept. of Physics and Astronomy}
\date{\today}
\documentclass[10pt]{article}
\usepackage[a4paper, total={18cm, 27cm}]{geometry}
\usepackage{outlines}
\usepackage{hyperref}

\begin{document}
\maketitle

\noindent
\textit{\textbf{Requirements}: 1. Propose an experiment to the professor by submitting a detailed project proposal.  2. Build the experiment and collect the data.  3. Create a 10-minute presentation on your results.  4. Give the presentation in the final week of class.} \\
\begin{itemize}
\item\textbf{ Project proposal}: The project proposal should be a 1-2 page description of the planned experiment.  It should include a diagram of the experiment, and include details about how the setup may be used to collect the appropriate data.  The proposal should be submitted on behalf of the group.
\item \textbf{Experiment}: The take-home experiments proposed in the text are a good start for ideas.  The experiment should be a device or setup that is cheap, safe, and easy to build.  The experiment may be focused on a topic covered this semester, but it is not limited to that.  It must be a concept from physics.
\item \textbf{Presentation}: The presentation should be 7-10 slides and include an introduction that explains why the experiment is interesting.  Next, it should include and explanation of the setup (including a diagram), and how the setup is used.  Third, it should include the data, represented clearly and with correct units.  The statistical errors, if any, should be quantified.  Finally, the presentation should include a conclusion that either confirms or rejects the hypothesis.
\item \textbf{Speaking}: When the group gives the presentation, the each member of the group should give at least part of the presentation.  Which parts and how much of the presentation is left to the group to decide.
\end{itemize}
\textit{\textbf{Example outline of the presentation}:}
\begin{outline}[enumerate]
\1 Slide 1: \textit{Measuring the coefficient of static friction} - by Jordan C. Hanson
\1 Slide 2: Introduction: ``The force of friction experienced by a stationary object is proportional to the coefficient of static friction, $\mu_{\rm s}$.  In this experiment, we measure $\mu_{\rm s}$ for a variety of materials.''
\1 Slide 3: ``(Diagram) A textbook is titled at an increasing angle until a given object begins to slide across it.  The angle is measured with a protracter, and the mass of the object is measured with a scale.  The result for $\mu_{\rm s}$ will be given by $\tan\theta$, where $\theta$ is the angle of incline.  The angle must be the maximum angle acheived before the object slides.''
\1 Slide 4: \textit{Tables of data for the angle $\theta$ based on object type are given.}  ``Here is our data.  As you can see...''
\1 Slide 5: ``The predicted coefficients of static friction for the objects are compared to the measured ones.  There are a few discrepancies...but we agree in general with the predictions.''
\1 Slide 6: ``In conclusion, the predicted coefficients of friction were measured with \textbf{standard deviations} in agreement with the global values.''
\end{outline}
\textbf{Grading}: 30\% of the grade will be assigned based on \textit{attention to detail} in the project proposal.  What parts will you need?  What needs to be built?  Is this feasible?  Another 50\% will be assigned based on the execution of the experiment.  Are we allowing any unnecessary errors?  Are there any ways we can be more precise?  Finally, 20\% of the grade will be assigned on \textit{how clearly you related the findings to the class.}  Are you plotting or listing the data in such a way that other people can understand it?  Are there unit errors?  \textbf{\textit{Can people read your graphs and tables?}}
\end{document}

\title{Calculus-Based Physics-1: Midterm 2}
\author{Dr. Jordan Hanson - Whittier College Dept. of Physics and Astronomy}
\date{\today}
\documentclass[10pt]{article}
\usepackage[margin=1.5cm]{geometry}
\usepackage{outlines}
\usepackage{graphicx}
\usepackage{amsmath}
\usepackage{hyperref}

\begin{document}
\twocolumn

\maketitle

\section{Unit 4: Potential Energy, Conservative Forces, and Energy Conservation}

\begin{enumerate}
\item  Suppose a potential energy function for a system is $U(x) = k(x^4-x^2)$. Assume that $k$ has units of Joules, and that $x$ is a unit-less measure of distance. 
 (a)  Where is the force zero? (b) Suppose a system is at rest at the origin.  If it is perturbed to the right by a small push, what is the largest displacement the system achieves from the origin? \\ \vspace{3.0cm}
 \item Suppose a 1 kg system is pushed along a surface with a coefficient of kinetic friction 0.5.  (a) If a force $F = 5\hat{i}$ N displaces the system $\Delta x = 1\hat{i}$ m from the origin, what is the work done by this force? (b) At the moment the system reaches $\vec{x} = 1\hat{i} + 0\hat{j}$, the force becomes $\vec{F} = 5\hat{j}$ N, displacing the system to $\vec{x} = 1\hat{i} + 1\hat{j}$.  What is the work done by $\vec{F}$?  (c) Next, $\vec{F} = -5 \hat{i}$ N, taking the system to $\vec{x} = 0\hat{i} + 1\hat{j}$.  What is the work? (d) Finally, $\vec{F} = -5 \hat{j}$ N, and the system returns to the origin.  What is the work?  (e) What is the total work for the four paths?  (f) What should the total be if the force of friction were conservative? \\ \vspace{4cm}
\end{enumerate}

\section{Unit 5: Linear Momentum}

\begin{enumerate}
\item Two molecules collide and stick together, forming one larger molecule. Each molecule weighs $20 \times 10^{-25}$ kg.  One has a velocity of 350 m/s, and the other has a velocity of -350 m s$^{-1}$.  What is the final speed of the big new molecule?
\begin{itemize}
\item A: 350 m s$^{-1}$
\item B: -350 m s$^{-1}$
\item C: 0 m s$^{-1}$
\item D: 700 m s$^{-1}$
\end{itemize}
\item Assume the same two molecules approach each other with momentum vectors that are 45 degrees apart.  What is the final velocity after they stick together? \\ \vspace{3.0cm}
\item Suppose a child drives a bumper car head on into the side rail, which exerts a force of 4000 N on the car for 0.200 s. (a) What impulse is imparted by this force? (b) Find the final velocity of the bumper car if its initial velocity was 2.80 m/s and the car plus driver have a mass of 200 kg. You may neglect friction between the car and floor. \\ \vspace{3.0cm}
\item  Two identical billiard balls have a one-dimensional collision in which one is initially motionless. After the collision, the struck ball moves with the same velocity as the ball that was originally moving, while the ball that was originally moving stops.  This interaction is
\begin{itemize}
\item A: Totally inelastic
\item B: Inelastic
\item C: Elastic
\item D: Totally elastic
\end{itemize}
\item A 30,000-kg freight car is coasting at 0.850 m/s with negligible friction under a hopper that dumps 110,000 kg of scrap metal into it. (a) What is the final velocity of the loaded freight car? (b) How much kinetic energy is lost? \\ \vspace{3.0cm}
\item Two objects, one with mass $m$ and the other with mass $2m$ approach each other with equal speed $v$. The objects interact inelastically.  Determine the location of the center of mass versus time, before and after the collision. \\ \vspace{3.0cm}
\end{enumerate}

\section{Unit 6: Fixed-axis rotation and Angular Momentum}

\begin{enumerate}
\item Suppose a music record spins once every 1.33 seconds.  What are the rotations per minute (rpm) of the record?  What is the angular velocity, in radians per second?
\begin{itemize}
\item A: 30 rpm, 0.75 radians sec$^{-1}$
\item B: 45 rpm, 4.7 radians sec$^{-1}$
\item C: 60 rpm, 0.75 radians sec$^{-1}$
\item D: 120 rpm, 4.7 radians sec$^{-1}$
\end{itemize}
\item What is the centripetal acceleration of a coin at the edge of the record?
\begin{itemize}
\item A: 1.1 m s$^{-1}$
\item B: 1.1 m s$^{-2}$
\item C: 2.2 m s$^{-1}$
\item D: 2.2 m s$^{-2}$
\end{itemize}
\item Suppose a centrifuge is spinning with angular velocity of $\omega$.  Which of the following will increase the centripetal acceleration by a factor of 100?
\begin{itemize}
\item A: $\omega \rightarrow 2\omega$
\item B: $\omega \rightarrow 3\omega$
\item C: $\omega \rightarrow 5\omega$
\item D: $\omega \rightarrow 10\omega$
\end{itemize}
\item Assume the angular velocity of the centrifuge is 200 rpm, and the radius is 12 cm.  If the solid contents of the vial in the centrifuge have a mass of 10 grams, what is the centripetal force?
\begin{itemize}
\item A: 0.13 N
\item B: 0.53 N
\item C: 0.83 N
\item D: 1.23 N
\end{itemize}
\item Consider \textit{las bolas} we covered in class: two balls of mass $m$, attached by a chord, thrown such that the balls spin around the center of mass (COM) while the COM goes forward.  If the balls have a mass 0.5 kg, rotate around the COM 4 times per second, and the COM moves forward at 8 m s$^{-1}$, (a) What is the total kinetic energy of \textit{las bolas}? (b) What is the angular momentum of \textit{las bolas}? (c) If \textit{las bolas} are thrown straight upward at 8 m s$^{-1}$, how high will they go? \\ \vspace{4cm}
\item  Suppose a bolt needs to be unstuck from a flat bulkhead. Assume the bolt is at the origin. We must twist it counterclockwise to loosen it. (a) Suppose we have a wrench with length $\vec{r} = 5\hat{i} + 5\hat{j}$ cm attached to the bolt. If we push on the end of the wrench with
$\vec{F} = -10\hat{i} + 10\hat{j}$ N of force, what is the torque on the bolt? (b)  If $\vec{r} \rightarrow 2\vec{r}$, what would the torque be? (c) For the same wrench, find an example of the force if the torque is $\vec{\tau} = 30$ N cm. \\ \vspace{3cm}
\item Recall that the moment of inertia of a disc of mass $M$ and radius $R$ is $I = \frac{1}{2} MR^2$. (a) Suppose a bicycle chain is turning a gear that has a disc-like mass distribution.  If the angular speed is described by $\omega(t) = 10t + 60$ rpm, what torque (in N m) is provided by the chain? \\ \vspace{3cm}
\item Suppose a a playground has a wheel that spins horizontally, allowing the children to spin around in a circle.  The mass of the wheel is 100 kg, and the radius is 1.5 meters.  (a) If the object is spinning at 30 rpm, what is the angular momentum? (b) If a 40 kg child lands on the edge, what is the new rpm? \\ \vspace{4cm}
\end{enumerate}

\section{Unit 7: Statics}

\begin{enumerate}
\item A steel beam is left balancing on one support in the middle. A 40 kg pile of bricks rests 3 m from the origin, and a 60 kg sack of concrete lies 2 m from the origin.  The beam will:
\begin{itemize}
\item A: Remain motionless
\item B: Rotate towards the bricks
\item C: Rotate towards the concrete
\end{itemize}
\item A person carries a plank of wood 2.00 m long with one hand pushing down on it at one end with a force $\vec{F}_1$ and the other hand holding it up at 0.500 m from the end of the plank with force $\vec{F}_2$.  If the plank has a mass of 20.0 kg and its center of gravity is at the middle of the plank, what are the magnitudes of the forces $\vec{F}_1$ and $\vec{F}_2$? \textit{Hint: the net force and the net torque should both be zero (the two stability criteria).}\\ \vspace{3cm}
\end{enumerate}

\end{document}
\documentclass{article}
\usepackage{graphicx}
\usepackage[margin=1.5cm]{geometry}
\usepackage{amsmath}

\begin{document}
\twocolumn

\title{Wednesday warm-up: Kinematics, II and III}
\author{Prof. Jordan C. Hanson}

\maketitle

\section{Memory Bank}

\begin{enumerate}
\item Assume that acceleration is constant: $a = 3.0$ (m/s$^2$), and that $\Delta x = x_f - x_i$
\item $v_f(t) = at + v_{i}$ (m/s)
\item $x(t) = \frac{1}{2}at^2 + v_{i} t + x_{i}$ (m)
\item $v_f^2 = v_i^2 + 2a\Delta x$ (m/s)$^2$.
\item $R = v_i^2 \sin(2\theta)/g$ ... Range formula for projectile motion.
\end{enumerate}

\section{Chapter 3 - Constant Acceleration}

\begin{enumerate}
\item A particle moves in a straight line with an initial velocity of 30 m s$^{-1}$ and constant acceleration 30 m s$^{-2}$. (a) What is its displacement at $t = 5$ s? (b) What is its velocity at this same time? \\ \vspace{3cm}
\item A swan on a lake gets airborne by flapping its wings and running on top of the water. (a) If the swan must reach a velocity of 6.00 m s$^{-1}$ to take off and it accelerates from rest at an average rate of 0.35 m s$^{-2}$, how far will it travel before becoming airborne? (b) How long does this take? \\ \vspace{3cm}
\item Notice the final formula in the Memory Bank.  Let $R$ represent the \textit{range} of a ball thrown at an angle $\theta$ with respect to the horizontal plane at an initial speed of $v_i$.  (a) Cook up a reasonable set of numbers for a thrown baseball, and calculate the range. (b) What happens to the range if $v_i$ is doubled?  (c) What happens to the range if $v_i$ is decreased by a factor of 2?
\end{enumerate}

\end{document}

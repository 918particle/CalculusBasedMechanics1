\documentclass{article}
\usepackage{graphicx}
\usepackage[margin=1.5cm]{geometry}
\usepackage{amsmath}
\usepackage{url}

\begin{document}
\twocolumn

\title{Laboratory Activity: Unit 1, Measuring $g$}
\author{Prof. Jordan C. Hanson}

\maketitle

\section{The Acceleration of Gravity, First Measurement}

The goal of this laboratory activity is to measure $g$, the acceleration due to gravity.  Let $g$ be the acceleration downward, and assume it is constant.  Let $v_{i,y}$ be the initial velocity in the y-axis, and assume the y-axis is vertical.  Let $y_i$ be the initial vertical position.  The position of a vertically accelerating object, in general, is given by
\begin{equation}
y(t) = -\frac{1}{2}gt^2 + v_{i,y} t + y_i \label{eq:1}
\end{equation}
In Eq. \ref{eq:1}, $-g$ is the acceleration.  The vector form of acceleration points down, so we give $g$ a minus sign.  We begin to observe the system at time $t=0$.  If a stationary marble is dropped and Eq. \ref{eq:1} is used to predict the position $y(t)$, then $v_{i,y} = 0$.  Let the change in height be $h = y(t) - y_i$.  Show that
\begin{equation}
h = -\frac{1}{2}g t^2
\end{equation}
Use this equation to solve for $g$.  The result should be
\begin{equation}
g = \frac{-2h}{t^2} \label{eq:2}
\end{equation}
Use the following procedure to measure $g$:
\begin{enumerate}
\item Use the ruler to measure the vertical displacement.
\item Use a stopwatch to time the descent of the marble.
\item Using $h$ and $t$ in Eq. \ref{eq:2}, calculate $g$.
\item Repeat 10 times and compute the average for $g = g_{ave}$.
\item Calculate the \textit{percent error} of $g$, using $g = 9.81$ m s$^{-2}$.
\begin{equation}
\Delta g ~ (\%) = \frac{g_{ave} - g}{g} \times 100
\end{equation}
\end{enumerate}

\section{The Acceleration of Gravity, \\ Second Measurement}

Now measure $g$ using the pendulum from the previous lab activity.  Let $T$ be the period of the pendulum, and $L$ be the length.  The relationship between $T$, $L$, and $g$ is

\begin{equation}
T = 2\pi \sqrt{L/g} \label{eq:3}
\end{equation}

Solve Eq. \ref{eq:3} for $g$, and repeat the above procedure to obtain $g_{ave}$ and the percent error using the pendulum.  Compare the results for $g_{ave}$ from each technique.

\section{The Acceleration of Gravity, Third Measurement}

We will measure $g$ more precisely now.  Use the following procedure to extract the $g$ measurement from the data.
\begin{enumerate}
\item Estimate the error on your $T$ measurements.  Assume, for example, that your precision on measuring $T$ is 5\%, fractionally.  Thus, if you measure $T$ to be 1.8 seconds, then a 50\% error is $\pm 0.9$ seconds, and 5\% error is $\pm 0.09$ seconds.
\item Create a spreadsheet with five columns.  Name them ``Length (m),'' ``Period (sec),'' ``Error (sec),'' ``Theory (sec),'' and ``chi squared.''  In the first column, enter your length data in meters.  In the second column, enter your measured period data in seconds.  In the third column, enter your error estimate on the period in seconds.
\item In the fourth column, enter: \verb+=2*3.14*SQRT(A2/9.81)+.  This assumes your first length measurement is in cell A2.  Click and drag the formula down to repeat the calculation for each length measurement.  Note the formula assumes $g = 9.81$ m s$^{-2}$.
\item In the fifth column, enter: \verb+=(B2-D2)^2/C2^2+.  This assumes your first period measurement is in cell B2, the associated error is in C2, and that the theoretical prediction is in D2.
\item In the cell below the ``chi-squared'' data, enter the following: \verb+=SUM(E2:E12)/N+.  Trade ``N'' for the number of measurements minus one.  \textbf{Does your sum result in a number close to one?}  Why should this occur if the theoretical prediction is a \textit{good fit} to the data?
\item The summed quantity is known as the \textit{reduced $\chi^2$ value.}  Summing the fractional squared difference between theory and measurement in this way results in a number that follows the $\chi^2$ distribution, a well-known probability distribution in the subject of probability and statistics.  \textbf{If your chi-squared value is not near 1.0, try tuning your $g$ value within the Theory column.  Tune $g$ until $\chi^2$ is minimized.}
\item Create a graph of your data, including errors and theoretical prediction, below.
\item To explore $\chi^2$ in more depth, follow this link: \url{https://phet.colorado.edu/en/simulations/curve-fitting}.
\end{enumerate}

\end{document}

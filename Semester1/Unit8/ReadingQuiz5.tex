\documentclass{article}
\usepackage{graphicx}
\usepackage[margin=1.5cm]{geometry}
\usepackage{amsmath}

\begin{document}
\twocolumn

\title{Friday Warm Up: Unit 5: Momentum II}
\author{Prof. Jordan C. Hanson}

\maketitle

\section{Memory Bank}

\begin{itemize}
\item Let $M$ be the total mass of a system, and let $m_j$ and $\vec{r}_j$ $(j = 1,...,N)$ be the masses and positions of the constituent parts of the system.  The position of the center of mass is
\begin{equation}
\vec{r}_{\rm CM} = \frac{1}{M}\sum_{j=1}^{N}m_j \vec{r}_j
\end{equation}
\item The momentum of the center of mass $\vec{P}_{\rm CM}$ is
\begin{equation}
\vec{P}_{\rm CM} = M \frac{d\vec{r}_{\rm CM}}{dt} = \sum_{j=1}^N \vec{p}_j
\end{equation}
\item The net external force on a system obeys
\begin{equation}
\vec{F} = \frac{d\vec{P}_{\rm CM}}{dt} = M \frac{d^2\vec{r}_{\rm CM}}{dt^2}
\end{equation}
\end{itemize}

\begin{figure}[ht]
\centering
\includegraphics[width=0.28\textwidth]{figures/Bola.jpg} \hspace{0.25cm}
\includegraphics[width=0.18\textwidth]{figures/orbit.png}
\caption{\label{fig:1} (Left) A \textit{gaucho} using a bola weapon to hunt a rhea bird. (Right) A planet orbits a star.}
\end{figure}

\section{Momentum II}

\begin{enumerate}
\item In Pre-columbian and colonial period Latin America, \textit{gauchos} would sometimes hunt with a weapons known as \textit{bolas} (Fig. \ref{fig:1}, left).  The bolas were thrown, and would spin around the center of mass until they wrapped the limbs of the prey. (a) Suppose two masses $m$ are separated by a diameter $d$.  The masses orbit the center with frequency $f$.  (a) Graph the positions in an x-y coordinate system, and (b) write down a system of equations describing the positions of the masses versus time.  (c) Suppose $f = 5$ Hz, or 5 rotations per second.  Locate the center of mass at $t=0.2$ seconds. (d) If the bolas are each 1.2 kg, what is the magnitude of the momentum of each bola?  (e) What is the \textit{total momentum} $P_{\rm CM}$? \\ \vspace{5cm}
\item Consider Fig. \ref{fig:1} (right), in which a single planet orbits a star located at the origin at $t=0$. Let the star have mass $M$, the planet have mass $m$, and let the distance between them be $r$.  Let the ratio of the masses be $\mu = m/M$.  (a) Show that the center of mass is given by 
\begin{equation}
\vec{r}_{\rm CM} = \left(\frac{\mu}{\mu+1}\right)\vec{r}
\end{equation}
(b) Show that $\vec{r} = 0$ in the limit that $\mu \ll 1$. \\ \vspace{4cm}
\item Assume there is no \textit{net, external} force on the system.  The center of mass will
\begin{itemize}
\item A: Accelerate 
\item B: Decelerate
\item C: Remain stationary
\item D: Remain at constant velocity
\end{itemize}
\end{enumerate}

\end{document}

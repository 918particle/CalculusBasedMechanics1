\documentclass{article}
\usepackage{graphicx}
\usepackage[margin=1.5cm]{geometry}
\usepackage{amsmath}

\begin{document}
\twocolumn

\title{Monday Warm Up: Unit 5: Momentum II}
\author{Prof. Jordan C. Hanson}

\maketitle

\section{Memory Bank}

\begin{itemize}
\item $\vec{p}_{\rm total} = \vec{p}_1 + \vec{p}_2$ ... Total momentum.
\item $\vec{p}_{\rm total,i} = \vec{p}_{\rm total,f}$ ... Momentum is conserved.
\item $KE = \frac{1}{2}mv^2$ ... Kinetic energy.
\item $KE_{tot,i} = KE_{tot,f}$ ... Totally elastic collisions.
\end{itemize}

\section{Momentum}

\begin{enumerate}
\item \textbf{Proof of momentum conservation.} (a) Suppose two masses $m_1$ and $m_2$ collide.  Write down Newton's Third Law as it applies to $m_1$ and $m_2$. (b) Move all terms to one side of the formula.  (c) Substitute the derivative of velocity for acceleration in the formula. (d) Because the derivative is a linear operator, show that
\begin{equation}
\frac{d\vec{v}_1}{dt} + \frac{d\vec{v}_2}{dt} = \frac{d}{dt}\left(\vec{v}_1 + \vec{v}_2\right)
\end{equation}
(e) Show, therefore, that the total momentum does not change with time. \\ \vspace{4cm}
\item Ernest Rutherford demonstrated that nuclei were very small and dense by scattering helium-4 nuclei from gold-197 nuclei. The energy of the incoming helium nucleus was $8.00\times 10^{-13}$ J, and the masses of the helium and gold nuclei were $6.68\times 10^{-27}$ kg and $3.29\times 10^{-25}$ kg, respectively (note that their mass ratio is 4 to 197). See Fig. \ref{fig:1}.  (a) If a helium nucleus scatters to an angle of $120^{\circ}$ during an elastic collision with a gold nucleus, calculate the final speed of the helium nucleus and the final velocity (magnitude and direction) of the gold nucleus.  (b) What is the final kinetic energy of the helium nucleus? \\ \vspace{5cm}
\item Suppose we want to integrate a function $v(t)$, but we do not know its form.  Instead, we only have measurements of it, and we can graph it.  Table \ref{tab:1} contains the measurements of $v(t)$.  Using the data, compute the integral of $v(t)$, and fill in the Position column, assuming the initial position was 0 m.
\end{enumerate}

\begin{figure}
\centering
\includegraphics[width=0.4\textwidth]{figures/nuclei.jpeg}
\caption{\label{fig:1} An alpha particle (helium nucleus) interacting with a gold nucleus.}
\end{figure}

\begin{table}
\small
\centering
\begin{tabular}{| c | c | c |}
\hline
$t$ (sec) & $v(t)$ (m s$^{-1}$) & Position (m) \\ \hline
0 & 0 & \\ \hline
1 & 2.1 & \\ \hline
2 & 3.9 & \\ \hline
3 & 9.5 & \\ \hline
4 & 16.1 & \\ \hline
5 & 23.9 & \\ \hline
6 & 35.0 & \\ \hline
7 & 49.1 & \\ \hline
8 & 62.0 & \\ \hline
9 & 82.0 & \\ \hline
10 & 102.0 & \\ \hline
\end{tabular}
\caption{\label{tab:1} A table of velocity data, sampling a function $f(t)$.}
\end{table}

\end{document}

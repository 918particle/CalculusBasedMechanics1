\documentclass{article}
\usepackage{graphicx}
\usepackage[margin=1.5cm]{geometry}
\usepackage{amsmath}
\usepackage{hyperref}

\begin{document}

\title{Lab Activity: Smashing Things Together (Momentum)}
\author{Prof. Jordan C. Hanson}

\maketitle

\section{Introduction}
This lab activity demonstrates momentum conservation through the use of carts that repel each other via magnets, and also simply collide.  Figure \ref{fig:carts} demonstrates the general setup. When momentum is conserved between two objects labeled 1 and 2, with $i$ and $f$ referring to the initial and final states,
\begin{equation}
p_{1i} + p_{2i} = p_{1f} + p_{2f}
\end{equation}

\begin{figure}[ht]
\centering
\includegraphics[width=0.6\textwidth,trim=0cm 1.2cm 0cm 5cm,clip=true]{figures/carts.png}
\caption{\label{fig:carts} Carts that collide with different interaction properties.}
\end{figure}

\section{System Setup}

Set the long frictionless rail along your lab table, and ensure that it is level by adjusting the four screws at the corners.  The lab carts should not roll one way or the other if placed on the rail once it is level.  Place one cart in the middle and leave it stationary.  Place another cart near one end of the rail.  Prepare two (smartphone) stopwatches and a ruler to measure time and distance for carts moving along the rail.

\section{Measuring Momentum Conservation}

Roll the cart near the end of the rail towards the stationary one at \textbf{low speed}.  (a) Measure the velocity of the carts by marking the start of the motion, the time of collision, and the end of the motion in both time and distance.  (b) Now alter the mass of the stationary cart by added a weight to it, and repeat the experiment.  \\ \vspace{1.5cm}

The mass of the carts is 510 grams.  Do your observations match momentum conservation?  There are two cases: equal masses and unequal masses.  Record the initial and final total momentum for the first and second cases below.  Was kinetic energy conserved in either interaction?  \textbf{Bonus:} using the PhET Collision Lab (\url{https://phet.colorado.edu/en/simulations/collision-lab}), simulate the situation on the rail as accurately as you can.  Do you get the same results?

\end{document}
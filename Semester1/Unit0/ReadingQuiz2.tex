\documentclass{article}
\usepackage{graphicx}
\usepackage[margin=1.5cm]{geometry}
\usepackage{amsmath}

\begin{document}
\twocolumn

\title{Friday warm-up: unit analysis and vectors}
\author{Prof. Jordan C. Hanson}

\maketitle

\section{Chapter 1 - Unit analysis}

\begin{enumerate}
\item Convert 100 km hr$^{-2}$ to m s$^{-1}$:
\begin{itemize}
\item A: 27.8 m
\item B: 27.8 m s$^{-1}$
\item C: 360 m
\item D: 360 m s$^{-1}$
\end{itemize}
\end{enumerate}

\begin{enumerate}
\item Convert 30 miles to km.  \textit{Hint: 1 mile is 1.6 km.}
\begin{itemize}
\item A: 24 km
\item B: 36 km
\item C: 48 km
\item D: 60 km
\end{itemize}
\end{enumerate}

\begin{enumerate}
\item An empty pool has a volume of 30 m$^3$.  If we must fill the pool with barrels of water, each with a volume of 100 liters, how many barrels of water are needed to fill the pool? \textit{Hint: 1 liter is 1000 mL, and 1 mL is 1 cm$^3$.}
\begin{itemize}
\item A: 3 barrels
\item B: 30 barrels
\item C: 300 barrels
\item D: 3000 barrels
\end{itemize}
\end{enumerate}

\begin{enumerate}
\item How many seconds are in a year? \textit{(This is important for calculations involving cumulative exposure to radiation).}
\begin{itemize}
\item A: $10^4$ seconds
\item B: $10^5$ seconds
\item C: $10^6$ seconds
\item D: $10^7$ seconds
\end{itemize}
\end{enumerate}

\vspace{3cm}

\section{Chapter 2 - Vectors}

\begin{enumerate}
\item Suppose a ship leaves an island, heading Northeast, at 25 km hr$^{-1}$.  After 1 hour, the ship turns North and continues at the same speed for another hour.  What is the final location of the ship?
\begin{itemize}
\item A: (17.7,17.7) km
\item B: (35.5,35.5) km
\item C: (17.7,42.7) km
\item D: (35.5,60.5) km
\end{itemize}
\item If an aircraft flies North for 100 km, and then flies 100 km East (to avoid a mountain range), how far is the craft from the original location?
\begin{itemize}
\item A: 70.7 km
\item B: 141.4 km
\item C: 200 km
\item D: 100 km
\end{itemize}
\item \textbf{Let us introduce the $\hat{i}\hat{j}\hat{k}$ notation for vectors.}  The vector $\hat{i}$ has no units, and a length of 1 in the x-direction.  The $\hat{j}$ vector is identical to $\hat{i}$, but pointed in the y-direction.  The vector $\hat{k}$ goes with the z-direction.  Compute the following:
\begin{itemize}
\item $\vec{x} = 2\hat{i} + 2\hat{j}$, and $\vec{y} = -2\hat{i} + 2\hat{j}$.  $\vec{x} + \vec{y} = $ \\ \vspace{2cm}
\item $\vec{x} = 2\hat{i} + 4\hat{j}$, and $\vec{y} = 2\hat{i} - 4\hat{j}$.  $\vec{x} + \vec{y} = $ \\ \vspace{2cm}
\item $\vec{x} = \hat{i} + \hat{j}$, and $\vec{y} = \hat{i} - \hat{j}$.  $\vec{x} - \vec{y} = $ \\ \vspace{2cm}
\end{itemize}
\end{enumerate}

\end{document}

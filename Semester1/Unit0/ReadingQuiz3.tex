\documentclass{article}
\usepackage{graphicx}
\usepackage[margin=1.5cm]{geometry}
\usepackage{amsmath}

\begin{document}

\title{Warm Up: Unit analysis and vectors}
\author{Prof. Jordan C. Hanson}

\maketitle

\section{Memory Bank}

\begin{enumerate}
\item $\vec{v} = v_x \hat{i} + v_y \hat{j}$ ... Definition of a vector in terms of $\hat{i}$ and $\hat{j}$ components (representing the x-direction and y-direction).
\item $\vec{v} + \vec{w} = (v_x + w_x) \hat{i} + (v_y + w_y) \hat{j}$ ... Vector addition: the $\hat{i}$-components add with each other, and the $\hat{j}$-components add with each other.
\end{enumerate}

\section{Chapter 1 - Unit Analysis and Estimation}

\begin{enumerate}
\item There are 2.2 lbs per kilogram.  Convert 160 lbs. to kilograms: \\ \vspace{0.5cm}
\item A unit of \textit{acceleration} is m s$^{-2}$, spoken like ``meters per second squared.''  Which of the following is true?
\begin{itemize}
\item A: 1 m s$^{-2} = 3.6 \times 10^{-1}$ cm min$^{-2}$
\item B: 1 m s$^{-2} = 3.6 \times 10^{-1}$ cm min$^{-3}$
\item C: 1 m s$^{-2} = 3.6 \times 10^{-1}$ cm min$^{-4}$
\item D: 1 m s$^{-2} = 3.6 \times 10^{-1}$ cm min$^{-5}$
\end{itemize}
\end{enumerate}

\section{Chapter 2 - Algebra of Vectors}

\begin{enumerate}
\item A \textit{vector} is a quantity with a \textit{magnitude} and a \textit{direction}.  If we travel 4 km in the x-direction, and we travel 4 km in the y-direction, our displacement from the origin is $\vec{x} = 4 \hat{i} + 4\hat{j}$ km.  Evaluate the following:
\begin{itemize}
\item A: $\vec{a} = -2 \hat{i} + 2\hat{j}$, $\vec{b} = -1 \hat{i} + 1\hat{j}$.  $\vec{a} + \vec{b} = $
\item B: $\vec{a} = -2 \hat{i} + 2\hat{j}$, $\vec{b} = -1 \hat{i} + 1\hat{j}$.  $\vec{a} - \vec{b} = $
\end{itemize}
\vspace{1cm}
\item Consider the vector $\vec{a}$ above.  Suppose it has units of kilometers, and it represents the displacement of an aircraft after 1 minute.  (a) Draw the vector below in a 2D coordinate system.  That is, make a vector go ``2 left,'' and ``2 up.'' (b) Using the Pythagorean theorem, calculate the magnitude of the vector $|\vec{a}|$.
\end{enumerate}

\end{document}

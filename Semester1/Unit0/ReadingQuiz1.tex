\documentclass{article}
\usepackage{graphicx}
\usepackage[margin=1.5cm]{geometry}
\usepackage{amsmath}

\begin{document}
\twocolumn

\title{Wednesday warm-up: units, vectors, and introductory calculus}
\author{Prof. Jordan C. Hanson}

\maketitle

\section{Chapter 1 - Unit analysis, \\ Estimation}

\begin{enumerate}
\item Which of the following are correct?
\begin{itemize}
\item A: The quantity m s$^{-1}$ is a unit of acceleration.
\item B: The quantity m s$^{-1}$ is a unit of speed.
\item C: The quantity m s$^{-2}$ is a unit of speed.
\item D: The quantity m s$^{-2}$ is a unit of acceleration.
\end{itemize}
\item Which of the following represents the density of lead?
\begin{itemize}
\item A: 0.11 g cm$^{-3}$
\item B: 1.1 g cm$^{-3}$
\item C: 11 g cm$^{-3}$
\item D: 111 g cm$^{-3}$
\end{itemize}
\item If there are 2.2 lbs/kg, which of the following is equivalent to 100 lbs in kg?
\begin{itemize}
\item A: 220 kg
\item B: 100 kg
\item C: 45.5 kg
\item D: 10.5 kg
\end{itemize}
\item A train leaves Los Angeles Union Station for the Bay Area (Evansville) at 60 km/hr.  If the Bay Area (Evansville) is 600 km to the North, how long before the train reaches the destination?
\begin{itemize}
\item A: 1 hour
\item B: 10 hours
\item C: 15 hours
\item D: 24 hours
\end{itemize}
\end{enumerate}

\section{Chapter 2 - Vectors}

\begin{enumerate}
\item Let $(v_x,v_y)$ represent the x and y-components of a vector $\vec{v}$.  The wind velocity is 10 km/hr, Southwest.  North and East vector components are positive, while South and West are negative.  Find $\vec{v}$ below.
\begin{itemize}
\item A: (7.1,7.1) km/hr
\item B: (-7.1,7.1) km/hr
\item C: (7.1,-7.1) km/hr
\item D: (-7.1,-7.1) km/hr
\end{itemize}
\item In the previous problem, the \textit{magnitude} of $\vec{v}$ is 10 km/hr.  This is because
\begin{itemize}
\item A: $\sqrt{7.1} = 10$
\item B: $\sqrt{7.1^2} = 10$
\item C: $\sqrt{7.1^2 + 7.1^2} = 10$
\item D: $2\sqrt{7.1^2} = 10$
\end{itemize}
\item Suppose $\vec{x}_1 = (2,3)$ km and $\vec{x}_2 = (-2,3)$ km.  What is $\vec{x}_1 + \vec{x}_2$?
\begin{itemize}
\item A: (6,0) km
\item B: (0,6) km
\item C: (4,0) km
\item D: (0,4) km
\end{itemize}
\end{enumerate}

\section{Calculus Topic - The Derivative}

\begin{enumerate}
\item The \textit{derivative}, or slope of a function $f(t)$ is defined as 
\begin{equation}
f'(t) = \lim_{dt \to 0} \frac{f(t+dt) - f(t)}{dt} \label{eq:1}
\end{equation}
Suppose $f(t) = a t^2$.  Given Eq. \ref{eq:1}, show that 
\begin{equation}
f'(t) = 2 a t
\end{equation}
\end{enumerate}

\end{document}

\documentclass{article}
\usepackage{graphicx}
\usepackage[margin=1.5cm]{geometry}
\usepackage{amsmath}

\begin{document}
\twocolumn

\title{Monday warm-up: Kinematics III, and Forces I}
\author{Prof. Jordan C. Hanson}

\maketitle

\section{Memory Bank}

\begin{enumerate}
\item $\vec{r}_{PS} = \vec{r}_{PS'}+\vec{r}_{S'S}$ ... Relative positions
\item $\vec{v}_{PS} = \vec{v}_{PS'}+\vec{v}_{S'S}$ ... Relative velocities
\item $\sum_i \vec{F}_i = \vec{F}_{\rm net}$ ... The net force
\item $\vec{F}_{\rm net} = \vec{0}$ $\to$ constant velocity ... Newton's First Law
\item $\vec{F}_{\rm net} = m\vec{a}$ ... Newton's Second Law
\end{enumerate}

\section{Chapter 4 - Relative Motion}

\begin{enumerate}
\item Two speedboats are traveling at the same speed relative to the water in opposite directions in a moving river. An observer on the riverbank sees the boats moving at 4.0 m/s and 5.0 m/s. (a) What is the speed of the boats relative to the river? (b) How fast is the river moving relative to the shore? \\ \vspace{3cm}
\end{enumerate}

\section{Chapter 5 - Forces}

\begin{enumerate}
\item (a) Find the net force, if each of the following forces act on a system: $\vec{F}_1 = -2.50\hat{i}-2.5\hat{j}$ N, $\vec{F}_2 = -2.50\hat{i}+2.5\hat{j}$ N, $\vec{F}_3 = 2.50\hat{i}-2.5\hat{j}$ N, and $\vec{F}_4 = 2.50\hat{i}+2.5\hat{j}$ N. (b) Is the system accelerating? \\ \vspace{3cm}
\item Two small forces, $\vec{F}_1 = -2.50\hat{i}-10.0\hat{j}$ N, and $\vec{F}_2 = 10.0\hat{i}-2.50\hat{j}$ N, are exerted on a rogue asteroid by a pair of space tractors. (a) Find the net force. (b) What are the magnitude and direction of the net force? (c) If the mass of the asteroid is 125 kg, what acceleration does it experience (in vector form)? (d) What are the magnitude and direction of the acceleration? \\ \vspace{3cm}
\item Two forces of 25 and 45 N act on an object. Their directions differ by 70 degrees.  The resulting acceleration has a magnitude of 10.0 m s$^{-2}$.  What is the mass of the object? \\ \vspace{3cm}
\item (a) Find an equation to determine the magnitude of the net force required to stop a car of mass m, given that the initial speed of the car is $v_0$ and the stopping distance is $\Delta x$. (b) Find the magnitude of the net force if the mass of the car is 1050 kg, the initial speed is 40.0 km hr$^{-1}$, and the stopping distance is 25.0 m.
\end{enumerate}

\end{document}

\documentclass{article}
\usepackage{graphicx}
\usepackage[margin=1.5cm]{geometry}
\usepackage{amsmath}

\begin{document}
\twocolumn

\title{Wednesday Warm Up: Unit 6: Fixed-Axis Rotational Motion}
\author{Prof. Jordan C. Hanson}

\maketitle

\section{Memory Bank}

\begin{itemize}
\item $\vec{s} = \vec{\theta} \times \vec{r}$ ... Vector relationship between angular displacement, arc length, and radius.
\item $v = r \omega$ ... Relationship between tangential velocity, angular velocity, and radius.
\item $\vec{v} = \vec{\omega} \times \vec{r}$ ... Vector relationship between tangential velocity, angular velocity, and radius.
\item $\vec{\tau} = \vec{r} \times \vec{F}$ ... Torque is the cross-product of the moment-arm $\vec{r}$ with the tangential force $\vec{F}$.
\item $\tau = I \alpha$ ... The magnitude of the torque is the product of the \textit{moment of inertia}, $I$, and the angular acceleration, $\alpha$.
\item $I = \sum_i m_i r_i^2$ ... The moment of inertia for a collection of point masses.
\end{itemize}

\begin{figure}
\centering
\includegraphics[width=0.35\textwidth]{figures/angle4.png}
\caption{\label{fig:angle5} From the definition of a radian we may derive angular velocity and acceleration.}
\end{figure}

\section{Rotational Variables and Rotational Motion}

\begin{enumerate}
\item On takeoff, the propellers on a UAV (unmanned aerial vehicle) increase their angular velocity for 3.0 s from rest at a rate of $\omega=(25.0t)$ rad s$^{-1}$ where $t$ is measured in seconds. (a) What is the instantaneous angular velocity of the propellers at $t=2.0$ seconds? (b) What is the angular acceleration? \\ \vspace{2cm}
\item Consider Fig. \ref{fig:angle5}.  Suppose we are fishing off the pier into the Pacific ocean.  (a) We are hauling up a 10 kg fish, but the reel breaks and the fish falls freely.  If the radius of the reel is 4.5 cm, what is the angular acceleration of the reel? (b) How long before the reel has spun 100 times? (c) What is the final angular velocity? \\ \vspace{3.0cm}
\end{enumerate}

\section{Rotational Kinetic Energy}

\begin{enumerate}
\item In the previous exercise, treat the reel as a spinning cylinder with moment of inertia $I = \frac{1}{2} M R^2$, and $M = 400$ grams.  What is the final kinetic energy of the reel? \\ \vspace{2cm}
\end{enumerate}

\section{Torque, and Moments of Inertia}

\begin{enumerate}
\item (a) Knowing the moment of inertia is $I = \frac{1}{2} M R^2$, and the net force from the fish is $w = mg$, obtain an expression for the angular acceleration of the reel, $\alpha$. (b) Without changing the \textit{mass} of the reel, how can we change the design to reduce $\alpha$, the angular acceleration?
\end{enumerate}

\end{document}

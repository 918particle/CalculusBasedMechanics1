\documentclass{article}
\usepackage{graphicx}
\usepackage[margin=1.5cm]{geometry}
\usepackage{amsmath}

\begin{document}

\title{Wednesday Reading Assessment: Unit 9, Torque, Angular Momentum, and Cross-Products}
\author{Prof. Jordan C. Hanson}

\maketitle

\section{Memory Bank}

\begin{itemize}
\item $\hat{i} \times \hat{j} = \hat{k}$ ... Example of a right-handed cross-product.
\item $\hat{k} \times \hat{i} = \hat{j}$ ... Example of a right-handed cross-product.
\item $\hat{j} \times \hat{k} = \hat{i}$ ... Example of a right-handed cross-product.
\item $\vec{\tau} = \vec{r} \times \vec{F}$ ... Definition of torque.
\item $\vec{L} = \vec{r} \times \vec{p}$ ... Definition of angular momentum, where $\vec{p}$ is the momentum.
\item $L = I \omega$ ... Relationship between angular momentum and angular velocity.
\end{itemize}

\section{Torque, Angular Momentum, and Cross-Products}

\begin{enumerate}
\item Suppose we have a wrench twisting a bolt.  The vector describing the wrench is $\vec{r} = 0.2\hat{i}+0.2\hat{j}$ (m).  We place a force ``upwards'' on the wrench of $\vec{F} = 15\hat{j}$ N.  What is the torque on the bolt? (Neglect the moment of inertia of the wrench).  \\ \vspace{3cm}
\item Suppose a wrench is tossed in the air, and spinning in a circle.  Suppose the moment of inertia is $I = \frac{1}{12} ML^2/12$, where the $M$ is the mass of the wrench and $L$ is the length of the wrench.  If the wrench is spinning with angular velocity $\omega$, write an expression for the angular momentum of the wrench. \\ \vspace{2cm}
\item If the wrench rotates at a rate of 60 rpm, has a mass of 0.3 kg, and a length of 15 cm, what is the angular momentum of the wrench?
\end{enumerate}
\end{document}

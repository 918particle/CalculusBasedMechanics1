\documentclass{article}
\usepackage{graphicx}
\usepackage[margin=1.5cm]{geometry}
\usepackage{amsmath}

\begin{document}
\twocolumn

\title{Monday Warm-Up: Review}
\author{Prof. Jordan C. Hanson}

\maketitle

\section{Unit 0}

\begin{enumerate}
\item Let $\vec{v} = v_x\hat{i} + v_y\hat{j}$. (a) If $v_x = -2$, $v_y = 2$, $w_x = 4$, and $w_y = -4$, what is $\vec{v} + \vec{w}$?  (b) What is $\vec{v} \cdot \vec{w}$? (c) Graph $\vec{v}$ and $\vec{w}$. \\ \vspace{3cm}
\item Suppose a ship accelerates from 0 km/hr to 10 km/hr in 60 seconds. If acceleration is the change in velocity divided by the change in time, what is the acceleration of the ship? \\ \vspace{2.5cm}
\item Estimate the area of the North Quad of Whittier College (the open space outside the SLC):
\begin{itemize}
\item A: 5000 m$^2$
\item B: 5000 cm$^2$
\item C: 500 m$^2$
\item D: 500 cm$^2$
\end{itemize}
\end{enumerate}

\section{Unit 1}

\begin{enumerate}
\item Suppose a cyclist has a velocity of 15 m/s at t = 0. If the acceleration is 3 m/s$^2$, (a) what is the velocity at t = 4 seconds? (b) What is the displacement of the cyclist at t = 4 seconds? (c) Is the average velocity different from the instantaneous velocity at t = 0 or t = 4 seconds? \\ \vspace{3cm}
\item Suppose a runner accelerates at 3 m/s$^2$, starting from rest. (a) Where does the runner reach 10 m/s? (b) When does the runner reach 10 m/s? \\ \vspace{3cm}
\end{enumerate}

\section{Units 2 and 3}

\begin{enumerate}
\item A soccer ball is kicked with an initial velocity of 16 m/s in the horizontal direction and 12 m/s in the vertical direction. (a) At what speed does the ball hit the ground? (b) For how long does the ball remain in the air? (c) What maximum height is attained by the ball? \\ \vspace{3cm}
\item (a) Show that the acceleration of any object down an incline with friction $f = \mu_k N$ is $a = g(\sin\theta - \mu_k \cos\theta)$. (b) What is the acceleration of a car sliding down an icy slope with a 10 degree grade, and $\mu_k = 0.1$? 
\end{enumerate}

\end{document}

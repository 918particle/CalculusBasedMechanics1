\documentclass{article}
\usepackage{graphicx}
\usepackage[margin=1.5cm]{geometry}
\usepackage{amsmath}

\begin{document}
\twocolumn

\title{Monday Warm Up: Unit 6: Angular Momentum}
\author{Prof. Jordan C. Hanson}

\maketitle

\section{Memory Bank}

\begin{itemize}
\item $\vec{\tau} = \vec{r} \times \vec{F}$ ... The relationship between \textit{torque}, $\vec{\tau}$, the \textit{moment arm}, $\vec{r}$, and the \textit{force}, $\vec{F}$.
\item $\vec{L} = I\vec{\omega}$ ... The relationship between angular momentum, $\vec{L}$, moment of inertia, $I$, and the angular velocity, $\vec{\omega}$.
\item $\vec{F}_{\rm net} = \frac{d\vec{p}}{dt}$ ... Relationship between torque, $\vec{\tau}$, and momentum, $\vec{p}$.
\end{itemize}

\section{Fixed Axis Rotation, and Torque}

\begin{enumerate}
\item Convince yourself that $\tau = I\alpha$.  That is, convince yourself that torque is the product of moment of inertia and angular acceleration. First, write down Newton's 2nd law in linear form.  Next, multiply both sides by the radius $r$ to obtain torque on one side.  Finally, substitute the linear acceleration for the angular acceleration, and identify the moment of inertia.  \\ \vspace{4cm}
\item Show that, if $\tau = I\alpha$, then $\frac{d\vec{L}}{dt} = \vec{\tau}$. \\ \vspace{2cm}
\item Suppose we are considering a point mass $m$ circling an origin at a radius $r$, with tangential velocity $\vec{v}$.  If $\vec{\tau} = \frac{d\vec{L}}{dt}$, what is $\vec{L}$ in terms of $r$, $m$, and $v$? \\ \vspace{3cm}
\end{enumerate}

\section{Angular Momentum}

\begin{enumerate}
\item A particle of mass 5.0 kg has position vector $\vec{r} =(2.0\hat{i}-3.0\hat{j})$ m at a particular instant of time when its velocity is $\vec{v} = 3.0\hat{i}$ m s$^{-1}$ with respect to the origin. (a) What is the angular momentum of the particle? (b) If a force $\vec{F} = 5.0\hat{j}$ N acts on the particle at this instant, what is the torque about the origin? \\ \vspace{4cm}
\item A Formula One race car with mass 750.0 kg is speeding through a course in Monaco and enters a circular turn at 220.0 km/h in the counterclockwise direction about the origin of the circle. At another part of the course, the car enters a second circular turn at 180 km/h also in the counterclockwise direction. If the radius of curvature of the first turn is 130.0 m and that of the second is 100.0 m, compare the angular momenta of the race car in each turn taken about the origin of the circular turn.
\end{enumerate}

\begin{figure}
\centering
\includegraphics[width=0.4\textwidth]{figures/f1.jpg}
\caption{\label{fig:1} A totally boss photo of an F1 racer.}
\end{figure}

\end{document}

\documentclass{article}
\usepackage{graphicx}
\usepackage[margin=1.5cm]{geometry}
\usepackage{amsmath}

\begin{document}
\twocolumn

\title{Wednesday Warm Up: Unit 6, Torque and Angular Momentum}
\author{Prof. Jordan C. Hanson}

\maketitle

\section{Memory Bank}

\begin{itemize}
\item $\vec{\tau} = \vec{r} \times \vec{F}$ ... The relationship between \textit{torque}, $\vec{\tau}$, the \textit{moment arm}, $\vec{r}$, and the \textit{force}, $\vec{F}$.
\item $\vec{\tau} = I \vec{\alpha}$ ... Newton's 2nd law, in angular form.
\item $d\vec{L}/dt = \vec{\tau}$ ... Newton's 2nd law in angular form, with angular momentum.
\item $\vec{L} = \vec{r} \times \vec{p}$ ... Relationship between linear and angular momentum.
\end{itemize}

\section{Fixed Axis Rotation, Torque, and Rotational Kinetic Energy}

\begin{enumerate}
\item A yo-yo can be thought of a solid cylinder of mass m and radius r that has a light string wrapped around its circumference (Fig. \ref{fig:1}, left). One end of the string is held fixed in space. If the cylinder falls as the string unwinds without slipping, what is the acceleration of the cylinder? \\ \vspace{3cm}
\item At a particular instant, a 1.0 kg particle is at $\vec{r} =(2.0\hat{i}-4.0\hat{j}+6.0\hat{k})$ m, with a velocity of $\vec{v} = (-1.0\hat{i}+4.0\hat{j}+1.0\hat{k})$ m s$^{-1}$, experiencing a force $\vec{F} = (10.0\hat{i}+15.0\hat{j})$ N. (a) What is the torque on the particle about the origin? (b) What is the angular momentum of the particle about the origin? (c) What is the rate of change of the particle's angular momentum? \\ \vspace{4cm}
\item An Earth satellite has its apogee at 2500 km above the surface of Earth and perigee at 500 km above the surface of Earth. At apogee its speed is 6260 m s$^{-1}$. What is its speed at perigee? The radius of the Earth is 6370 km (Fig. \ref{fig:1}, right). \\ \vspace{2cm}
\item A Molniya orbit is a highly eccentric orbit of a communication satellite so as to provide continuous communications for Scandinavian countries. As such, these countries have the satellite in view for extended periods (Fig. \ref{fig:2}). If the orbit has an apogee at 40,000.0 km and a velocity of 1.68 km s$^{-1}$, what would be its velocity at perigee measured at 200.0 km altitude?
\end{enumerate}

\begin{figure}
\centering
\includegraphics[width=0.125\textwidth]{figures/yoyo.png}
\includegraphics[width=0.325\textwidth]{figures/orbit.png}
\caption{\label{fig:1} (Left) A yo-yo and string. (Right) Elliptical orbit.}
\end{figure}

\begin{figure}[hb]
\centering
\includegraphics[width=0.2\textwidth]{figures/orbit2.png}
\caption{\label{fig:2} A Molniya orbit.}
\end{figure}

\end{document}

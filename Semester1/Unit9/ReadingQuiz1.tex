\documentclass{article}
\usepackage{graphicx}
\usepackage[margin=1.5cm]{geometry}
\usepackage{amsmath}

\begin{document}
\twocolumn

\title{Monday Warm Up: Unit 6: Rotational Motion I}
\author{Prof. Jordan C. Hanson}

\maketitle

\section{Memory Bank}

\begin{itemize}
\item $\vec{s} = \vec{\theta} \times \vec{r}$ ... Vector relationship between angular displacement, arc length, and radius.
\item $v = r \omega$ ... Relationship between tangential velocity, angular velocity, and radius.
\item $\vec{v} = \vec{\omega} \times \vec{r}$ ... Vector relationship between tangential velocity, angular velocity, and radius.
\item $\vec{\tau} = \vec{r} \times \vec{F}$ ... Torque is the cross-product of the moment-arm $\vec{r}$ with the tangential force $\vec{F}$.
\item $\tau = I \alpha$ ... The magnitude of the torque is the product of the \textit{moment of inertia}, $I$, and the angular acceleration, $\alpha$.
\end{itemize}

\section{Rotational Kinematics and \\ Dynamics}

\begin{enumerate}
\item The wheel of a grinder rotates such that it sweeps out an angle at the rate of 45.0 radians per second. The wheel rotates counterclockwise when viewed in the plane of the page. (a) What is the angular \textit{velocity} of the flywheel, including the direction? (b) Through how many radians does the flywheel rotate in 30 s? (c) What is the tangential speed of a point on the flywheel 10 cm from the axis of rotation? \\ \vspace{4cm}
\item (a) If we accelerate the angular speed from 100 to 700 rpm in 30 seconds, what is the angular acceleration? (b) If we begin to sharpen our knife with the grinder, and it causes the grindstone to decelerate by 7 radians/sec, when will the grindstone come to a stop? \\ \vspace{3cm}
\item Imagine the \textit{bolas} from the previous warm up, where two point masses are rotating around a center of mass.  Let us determine the energy from the rotation.  (a) Write down the total kinetic energy in terms of the masses and the tangential velocities. (b) Substitute for the angular velocity using $v = r\omega$. (c) Factor the angular velocity squared.  What term remains?  (d) Let $I = \sum_i m_i r_i^2$. Show that the total kinetic energy is
\begin{equation}
K = \frac{1}{2}I \omega^2
\end{equation}
\vspace{3cm}
\item Imagine the grindstone is spinning counter-clockwise in the xy-plane, and the angular velocity is along the positive z-axis.  (a) If the radius is $\vec{r} = 10\hat{i}$ cm, and we apply a force $\vec{F} = 100\hat{j}$ N.  What is the torque? (b) Suppose the \textit{moment of inertia} is $I = \frac{1}{2}m r^2$ for a disk.  This accounts for the shape, in addition to the mass.  If the grindstone has a mass of 45 kg, what is the moment of inertia? (c) What is the angular acceleration of the grindstone?
\end{enumerate}

\begin{figure}
\centering
\includegraphics[width=0.25\textwidth]{figures/circle2.jpeg}
\caption{\label{fig:2} Vector relationship between $\vec{\theta}$, $\vec{s}$, and $\vec{r}$ .}
\end{figure}

\end{document}

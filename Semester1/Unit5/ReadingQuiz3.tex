\documentclass{article}
\usepackage{graphicx}
\usepackage[margin=1.5cm]{geometry}
\usepackage{amsmath}

\begin{document}

\title{Warm Up Exercises: Drag, Circular Motion}
\author{Prof. Jordan C. Hanson}

\maketitle

\section{Memory Bank}

\begin{itemize}
\item Force of drag, in air or other gas: $F_D = \frac{1}{2}C \rho A v^2$.
\item In the above formula, $C$ is an empirical constant, $\rho$ is the density of the air or gas, $A$ is the area of the object, and $v$ is the object's velocity.
\item Circular motion position with angular velocity $\omega = \Delta \theta / \Delta t$:
\begin{equation}
\vec{r}(t) = r\cos(\omega t)\hat{i} + r\sin(\omega t)\hat{j}
\end{equation}
\item $a_{\rm C} = r \omega^2$ ... Centripetal force.
\item $v = r\omega$ ... Radial velocity.
\end{itemize}



\section{Drag Forces, Circular Motion}
\begin{enumerate}
\item Suppose the mass of a skydiver is $m = 65.0$ kg, and $C = 1.0$.  Also, $\rho = 1.2$ kg/m$^3$, and $A = 0.5$ m$^2$.  Equate the weight force and the drag force, and solve for the velocity.  This is the terminal velocity. \\ \vspace{1cm}
\item Suppose a system is rotating about the origin with a radius $r = 1.0$ m, and angular speed $\omega = 2\pi/10$ radians per second. (a) Where is the system at $t = 0$ seconds?  (b) Where is the system at $t=5$ seconds? (c) What are the radial velocity and centripetal acceleration? \\ \vspace{2cm}
\item Find the time $t$ that makes the position $\vec{r} = -1.0\hat{j}$ m.
\end{enumerate}

\end{document}

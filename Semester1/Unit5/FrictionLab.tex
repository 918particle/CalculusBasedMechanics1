\documentclass{article}
\usepackage{graphicx}
\usepackage[margin=1.5cm]{geometry}
\usepackage{amsmath}

\begin{document}

\title{Friction Lab with Pulleys: Testing Friction dependence on Normal Force}
\author{Prof. Jordan C. Hanson}

\maketitle

\section{Introduction}

Recall that the friction force is given by

\begin{equation}
f = \mu N
\end{equation}

$N$ represents the normal force.  On a flat solid surface, $N = mg$.  Notice the frictional force therefore does not depend on the area of contact between the two surfaces.  This lab is going to explore that.

\section{Setup}

You will need the following objects:

\begin{itemize}
\item A pulley, and a clamp to attach it to the table.
\item A wooden block with velcro on the sides.
\item Weights and hook.
\end{itemize}

Arrange a pulley at the edge of the table using the clamp.  The block shoud have a string attached to allow weights to be hung from it.  Place the block on the table, and feed the string through the pulley with the hook and weights on the other side.

\section{Free body diagram}

Place weights on the hook and the block.  What happens?  Draw a free body diagram for the forces acting \textit{on the block}: \\ \vspace{1cm}

\section{Measurements}

Hang a weight $m_2$ on the end of the string.  Let $m_1$ be the weight of the block (118 grams) plus any weight you add.  Determine what $m_1$ allows the block to slide at contant velocity.  Repeat the same measurements, but with the block turned on each of the four sides.  Compare the results.  What do you conclude about the frictional force? \textbf{Bonus:} Draw the free body diagram if the tension pulls at an angle $\theta$.  What happens to the normal force, and therefore friction?\\ \vspace{3cm}

\end{document}

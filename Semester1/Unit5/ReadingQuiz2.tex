\documentclass{article}
\usepackage{graphicx}
\usepackage[margin=1.5cm]{geometry}
\usepackage{amsmath}

\begin{document}
\twocolumn

\title{Wednesday warm-up: Forces III}
\author{Prof. Jordan C. Hanson}

\maketitle

\section{Memory Bank}

\begin{itemize}
\item Force of drag, in air or other gas: $F_D = \frac{1}{2}C \rho A v^2$.
\item In the above formula, $C$ is an empirical constant, $\rho$ is the density of the air or gas, $A$ is the area of the object, and $v$ is the object's velocity.
\item The horizontal force of friction: $\vec{f} = -\mu N \hat{i}$, where $\mu$ can be either the \textit{static} or \textit{kinetic} coefficient of friction.
\end{itemize}

\section{Forces III}

\begin{enumerate}
\item Suppose a cyclist with $A = 0.5$ m$^2$, $C = 1.0$, and total mass $m = 70$ kg is pedalling at $20$ m/s.  Assume the density of air is $\rho = 1.2$ kg m$^{-3}$.  (a) What is the drag force on the system? (b) What is the drag force if the speed drops to $10$ m/s? (c)  Suppose the speed is now $20$ m/s again, but the cyclist ducks down to lower the area to $A = 0.25$ m$^2$.  What is the new drag force? \\ \vspace{3cm}
\item Show that the acceleration of any object down a frictionless incline that makes an angle $\theta$ with the horizontal is $a=g\sin\theta$. (Note that this acceleration is independent of mass.) \\ \vspace{3cm}
\item (a) Show that the acceleration of any object down an incline where friction behaves simply (that is, where $f_k = \mu_k N$) is $a = g(\sin\theta - \mu_k\cos\theta)$.  Show that the acceleration reduces to the expression found in the previous problem when friction becomes negligibly small. \\ \vspace{3cm}
\item Calculate the deceleration of a snow boarder going up a slope of 10 degrees, assuming the coefficient of friction for waxed wood on wet snow (0.1). \\ \vspace{3cm}
\item Sometimes, friction can be \textit{static}.  That is, there's enough friction to keep an object stuck in position.  Suppose a sled is on a hill with incline angle $\theta$, and the coefficient of static friction is $\mu_s$.  Show that the maximum $\theta$, such that the sled does not move, satisfies
\begin{equation}
\mu_k = \tan\theta
\end{equation}
\vspace{3cm}
\item Let the force of drag be $F_{\rm D} = (1/2) C \rho A v^2$, and the weight of a falling system be $w=mg$.  Show that the maximum speed the system can achieve is 
\begin{equation}
v = \sqrt{\frac{2mg}{C\rho A}}
\end{equation}
Estimate this speed for a falcon diving to catch prey, letting $C=0.1$.
\end{enumerate}

\end{document}

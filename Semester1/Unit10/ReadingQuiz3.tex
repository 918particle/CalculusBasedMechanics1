\documentclass{article}
\usepackage{graphicx}
\usepackage[margin=1.5cm]{geometry}
\usepackage{amsmath}

\begin{document}
\twocolumn

\title{Wednesday warm-up: Statics II and Gravitation}
\author{Prof. Jordan C. Hanson}

\maketitle

\section{Memory Bank}

\begin{itemize}
\item Young's Modulus, $Y$, has units of N m$^{-2}$, and it relates the change in length $\Delta L$ of a system of original length $L_0$ and cross-sectional area $A$ subject to a force $F$:
\begin{equation}
\frac{\Delta L}{L_0} = \frac{1}{Y} \frac{F}{A}
\end{equation}
Summarized as \textit{stress}$= Y\times$\textit{strain}.
\item Shear Modulus, $S$, has units of N m$^{-2}$, and it relates the sideways change in length $\Delta x$ of a system of length $L_0$ and cross-sectional area $A$ subject to a force $F$:
\begin{equation}
\frac{\Delta x}{L_0} = \frac{1}{S} \frac{F}{A}
\end{equation}
Summarized as \textit{stress}$= S\times$\textit{sheer}.
\item Bulk Modulus, $B$, has units of N m$^{-2}$, and it relates the fractional change in volume to changes in pressure:
\begin{equation}
\Delta p = -B \frac{\Delta V}{V_0}
\end{equation}
The factor $k = 1/B$ is called the \textit{compressibility.}
\item \textbf{Universal Law of Gravitation}:
\begin{equation}
\vec{F}_{\rm G} = G \frac{m_1 m_2 \hat{r}}{r^2}
\end{equation}
$G = 6.67 \times 10^{-11}$ N kg$^{-2}$ m$^2$.
\end{itemize}

\section{Stress, Strain, and \\ Elastic Modulus}

\begin{enumerate}
\item When water freezes, its volume increases by 9.05\%. What force per unit area is water capable of exerting on a container when it freezes?  The bulk modulus of water is $2.2 \times 10^9$ Pa.  \\ \vspace{3cm}
\item The shear modulus of a wooden plank is $10 \times 10^9$ Pa.  Suppose the plank dimensions are 5cm by 10 cm, and it is 5 meters long.  (a) How much shear strain is caused by a 10 kg mass, 4.5 meters from the support point of the plank, if the 10 kg mass hangs freely? \\ \vspace{2.5cm}
\end{enumerate}

\section{Gravitation}

\begin{enumerate}
\item Calculate the value of $g$, the gravitational acceleration near the surface of the Earth, from the Universal Law of Gravitation.  Assume some \textit{test mass} has a weight $mg$ that is equal to the force of gravity one Earth radius from the center of the Earth, $R_{\rm E} = 6370$ km. \\ \vspace{3cm}
\item Note that the volume of a sphere is $V = (4/3) \pi r^3$, and the mass of an object of volume $V$ with density $\rho$ is $m = \rho V$.  (a) Show that the gravitational acceleration near the surface of a spherical mass is $g = (4/3) \pi G \rho r$. (b) Suppose in the future, we land on a new planet, with radius $r = 0.9 R_{\rm E}$, and find that $g=0.67 g_{\rm E}$.  That is, the acceleration due to gravity is 67\% as strong as that of Earth.  What is the (average) density of the new planet, relative to Earth? \\ \vspace{2.5cm}
\item Calculate the velocity required to orbit the Earth.  That is, at what speed will the gravitational attraction barely balance the centripetal force?
\end{enumerate}

\end{document}
\documentclass{article}
\usepackage{graphicx}
\usepackage[margin=1.5cm]{geometry}
\usepackage{amsmath}

\begin{document}
\twocolumn

\title{Warm Up: Energy I}
\author{Prof. Jordan C. Hanson}

\maketitle

\section{Memory Bank}

\begin{itemize}
\item $W = \vec{F} \cdot \Delta \vec{x}$ ... Definition of work
\item $W = \frac{1}{2}k(\Delta x)^2$ ... Work done to compress or stretch a spring by $\Delta x$.
\item $KE = \frac{1}{2}m v^2$ ... Definition of Kinetic Energy
\item $W = KE_f - KE_i$ ... Work-energy theorem
\item $U = mgh$ ... Work done against gravity in lifting an object of mass $m$ a distance $h$.
\item $P = \Delta E / \Delta t$ ... Definition of power.
\item Conservative force: the work done on a system is zero if the displacement is a \textbf{closed path.}
\end{itemize}

\section{Work and Energy, Power}

\begin{enumerate}
\item Suppose a car travels 108 km at a speed of 30.0 m/s, and uses 2.0 gal of gasoline. Only 30\% of the gasoline goes into useful work by the force that keeps the car moving at constant speed despite friction. (The energy content of gasoline is about 140 MJ/gal.) (a) What is the magnitude of the force exerted to keep the car moving at constant speed? (b) If the required force is directly proportional to speed, how many gallons will be used to drive 108 km at a speed of 27.0 m/s? \\ \vspace{4cm}
\item It takes 500 J of work to compress a spring 10 cm. What is the force constant of the spring? \\ \vspace{3cm}
\item A unit of power is called the \textbf{Watt}, and it is equal to one Joule per second.  (a) If a 2.2 kW system runs for one hour, how many Joules are consumed? (b) If a person burns 10 MJ of energy in a day (12 hours), what is the power consumption? (c) One kcal (kilocalorie) is 4184 Joules.  How many kcal did the person burn in the previous part? \\ \vspace{2cm}
\item Suppose the energy consumed as a function of time in a system is $E(t) = a t + b$.  Which of the following is the power?
\begin{itemize}
\item A: $b$
\item B: $a$
\item C: $t$
\item D: $E(t)$
\end{itemize}
\item A man of mass 80 kg runs up a flight of stairs 20 m high in 10 seconds. (a) how much power is used to lift the man? (b) If the man’s body is 25\% efficient, how much power does he expend?
\end{enumerate}

\end{document}

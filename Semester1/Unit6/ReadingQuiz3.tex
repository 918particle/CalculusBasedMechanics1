\documentclass{article}
\usepackage{graphicx}
\usepackage[margin=1.5cm]{geometry}
\usepackage{amsmath}

\begin{document}
\twocolumn

\title{Warm Up: Energy I}
\author{Prof. Jordan C. Hanson}

\maketitle

\section{Memory Bank}

\begin{itemize}
\item $W = \vec{F} \cdot \Delta \vec{x}$ ... Definition of work
\item $\vec{F} = -k\Delta \vec{x}$ ... Hooke's Law, or the force of a spring
\item $W = \frac{1}{2}k(\Delta x)^2$ ... Work done to compress or stretch a spring by $\Delta x$.
\item $KE = \frac{1}{2}m v^2$ ... Definition of Kinetic Energy
\item $W = KE_f - KE_i$ ... Work-energy theorem
\item Gravitational potential energy: $U = mgh$, where $h$ is the height, $m$ is the mass, and $g$ is the gravitational constant.
\end{itemize}

\section{Work and Energy}

\begin{enumerate}
\item Suppose three springs with equal $k$ constants are connected \textit{in series} (back to back).  If the springs have original length $L_0$, what is the total length if a mass $m$ is hung from them? \\ \vspace{4cm}
\item Suppose a force $\vec{F} = 20\hat{i} + 40\hat{j}$ N acts on a mass, and displaces it by $\Delta \vec{x} = 10\hat{i}$ N.  (a) What is the work done?  (b) What is the change in kinetic energy? (c) If the mass is 40 kg, what is the final velocity after the given displacement is reached? \\ \vspace{4cm}
\item Suppose we lift a 25 kg object 10 meters above the ground.  (a) How much work does this require? (b) What is the gravitational potential energy? (b) What is the final speed at the ground if the object is dropped? \\ \vspace{3cm}
\item In the previous problem, (a) what happens to the final speed if the height is doubled? (b) What is the amount of work required?
\end{enumerate}

\end{document}

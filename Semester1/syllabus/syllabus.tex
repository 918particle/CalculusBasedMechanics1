\title{Syllabus for Calculus-Based Physics 1 (Mechanics): PHYS150}
\author{Dr. Jordan Hanson - Whittier College Dept. of Physics and Astronomy}
\date{\today}
\documentclass[10pt]{article}
\usepackage[margin=1.5cm]{geometry}
\usepackage{outlines}
\usepackage{hyperref}
\usepackage{graphicx}
\begin{document}
\maketitle

\begin{abstract}
The concepts of calculus-based mechanics will be presented via interactive problem-solving in an integrated lecture/laboratory format.  First, the concepts of displacement, velocity, and acceleration in one and two dimensions will be introduced, building up to Newton's Laws of motion.  Next, the concepts of friction and rotational motion will be added.  More complex problems will be introduced through the conservation of energy and linear momentum, followed by the rotational equivalents.  This course includes analytic textbook problems, peer instruction and group discussions, interactive computational exercises, and lab-based activities.  Special topics include the application of Newton's Law of Gravity to the Solar System, and modern physics research.
\end{abstract} \vspace{0.1cm}
\noindent
\textit{\textbf{Pre-requisites}: MATH141 (may be taken concurrently).} \\
\textit{\textbf{Course credits, Liberal Arts Categorization}: 4 Credits, COM1, NSIN, QR2} \\
\textit{\textbf{Regular course hours and location}: MWF, 1:30 pm - 2:50 pm, SLC 228.} \\
\textit{\textbf{Instructor contact information}: Discord: 918particle, Email: jhanson2@whittier.edu, Office: SLC 212.} \\
\textit{\textbf{Office hours}: Open-door policy, or make an appointment via Discord/email.} \\
\textit{\textbf{Attendance/Absence}: Students needing to reschedule midterms must notify the professor a few days in advance.} \\ 
\textit{\textbf{Late work policy}: Late work is generally not accepted, but is left to the discretion of the instructor.} \\
\textit{\textbf{Text}: OpenStax University Physics, vol. 1: \url{https://openstax.org/details/books/university-physics-volume-1}.} \\
\textit{\textbf{Homework}: Homework exercises will be managed by The ExpertTA.  This system costs \$35 and will provide hints, feedback, and automatically grade student work.  Please register: \url{https://reg.theexpertta.com/USB06CA-31E00E-4JC}.} \\
\textit{\textbf{Grading}: In-class warm-up exercises are marked only for attendance.  Homework sets are assigned weekly.  Two open-book midterm exams will be completed outside of class.  There is no final exam for the course.  A final project and class presentation will be completed near the end of the semester.  See Tab. \ref{tab:grading}.}
\begin{table}
\centering
\begin{tabular}{| c | c | c |}
\hline
\textbf{Assignment} & \textbf{Weight} & \textbf{Date} \\ \hline
Daily exercises & 10 \% & Mondays, Wednesdays, and Fridays, 1:30 pm - 2:50 pm \\ \hline
Homework sets and labs & 30 \% & Fridays, submitted online using The ExpertTA \\ \hline
First Midterm Exam & 20 \% & October 13th, 2025 (take-home style on Units 0-3) \\ \hline
Second Midterm Exam & 20 \% & December 8th, 2025 (take-home style on Units 4-8) \\ \hline
Final Project Presentation & 20 \% & December 3rd and 5th, 2025 (in class) \\ \hline
\end{tabular}
\caption{\label{tab:grading} These are the grade weights for each assignment. The final project presentation can take two forms.  \textbf{Option A}: A 10-15 minute traditional presentation with several minutes for questions.  \textbf{Option B}: A video in digital storytelling format using WeVideo, also 10-15 minutes long.}
\end{table} \\
\textit{\textbf{Grade Settings}: $\geq 60\%, <70\%$ = D, $\geq 70\%, <80\%$ = C, $\geq 80\%, <90\%$ = B, $\geq 90\%, <100\%$ = A. Pluses and minuses: 0-3\% minus, 3\%-6\% straight, 6\%-10\% plus (e.g. 79\% = C+, 91\% = A-).} \\
\textit{\textbf{ADA Statement on Disability Services}: Whittier College is committed to make learning experiences as accessible as possible. If you experience physical or academic barriers due to a disability, you are encouraged to contact Student Disability Services (SDS) to discuss options. To learn more about academic accommodations, email disabilityservices@whittier.edu, call (562) 907-4825, or go to SDS which is located on the ground floor of Wardman Library.} \\
\textit{\textbf{Academic Honesty Policy}: \url{http://www.whittier.edu/academics/academichonesty}} \\
\noindent
\textit{\textbf{Course Objectives}:}
\begin{itemize}
\item Develop skill in written and oral expression of technical ideas.
\item Improve performance in solving technical problems using physics and mathematics.
\item Develop skill in constructing mathematical models of mechanical systems.
\item Improve performance in the application of conceptual and logical thinking in technical scenarios.
\item To practice scientific experimentation, data analysis, and reporting of results.
\end{itemize}
\clearpage
\twocolumn
\noindent
\textit{\textbf{Course Outline}:}
\begin{outline}[enumerate]
\1 \textbf{Unit 0:} Review and preparation 
\2 Course introduction and organization, peer instruction modality
\2 Physical science review - \textbf{Chapters 1.1 - 1.7}
\3 Physical quantities
\3 Estimation and unit analysis
\2 Kinematics, I - \textbf{Chapters 2.1 - 2.4}
\3 \textit{Mathematics review: scalars and vectors}
\3 Distance, velocity, and time
\2 \textit{Problem sets:}
\3 \textit{Problem set 1: chapters 1.1 - 1.7}
\3 \textit{Problem set 2: chapters 2.1 - 2.4}
\1 \textbf{Unit 1:} One and two-dimensional kinematics
\2 Kinematics, II - \textbf{Chapters 3.1 - 3.6}
\3 Definitions of time, displacement, velocity, acceleration
\3 Kinematic equations with constant acceleration
\2 Kinematics, III - \textbf{Chapters 4.1 - 4.5}
\3 Displacement, velocity, and acceleration in multiple dimensions
\3 Projectile motion, uniform circular motion
\3 Relative motion
\2 \textit{Problem sets:}
\3 \textit{Problem set 3: chapters 3.1 - 3.6}
\3 \textit{Problem set 4: chapters 4.1 - 4.5}
\1 \textbf{Unit 2:} Forces and Newton's Laws
\2 Forces, I - \textbf{Chapters 5.1 - 5.7}
\3 Definition of force, free-body diagrams
\3 Newton's First Law
\3 Newton's Second Law
\3 Newton's Third Law
\2 Forces, II: applications - \textbf{Chapters 6.1 - 6.4}
\3 Normal forces, tension, other examples
\3 Friction
\3 Centripetal force
\3 Drag and terminal velocity
\2 \textit{Problem sets:}
\3 \textit{Problem set 5: chapters 5.1 - 5.7}
\3 \textit{Problem set 6: chapters 6.1 - 6.4}
\1 \textbf{Unit 3:} Work and Energy
\2 Energy, I - \textbf{Chapters 7.1 - 7.4}
\3 Definitions of work and kinetic energy
\3 Work-energy theorem, potential energy, energy conservation
\3 Energy consumption versus time: power
\2 \textit{Problem sets:}
\3 \textit{Problem set 7: chapters 7.1 - 7.4}
\1 \textbf{First Midterm} - due October 13th, 2025
\2 The midterm will be posted on Moodle on Friday, October 10th.
\2 The exam will be completed at home, with an open-book/open-note policy.  The format includes multiple choice, written exercises, and design problems.
\2 The exam is due in PDF form on Monday, October 13th, to be submitted via Moodle.
\1 \textbf{Unit 4:} Potential Energy, Conservative Forces, and Energy Conservation
\2 Energy, II - \textbf{Chapters 8.1 - 8.5}
\3 Potential energy
\3 Conservative forces
\3 Energy conservation
\3 \textit{Problem sets:}
\4 \textit{Problem set 8: chapters 8.1 - 8.5}
\1 \textbf{Unit 5:} Linear momentum
\2 Momentum, I - \textbf{Chapters 9.1 - 9.3}
\3 Mathematics review: systems of equations
\3 Definition of linear momentum
\3 Momentum conservation
\2 Momentum, II - \textbf{Chapters 9.4 - 9.7}
\3 Elastic and inelastic scattering
\3 Scattering in more than one dimension
\3 Center of mass
\3 Rocket propulsion
\2 \textit{Problem sets:}
\3 \textit{Problem set 9: chapters 9.1 - 9.3}
\3 \textit{Problem set 10: chapters 9.4 - 9.7}
\1 \textbf{Unit 6:} Fixed-axis rotation and Angular Momentum
\2 Fixed axis rotation - \textbf{Chapters 10.1 - 10.8}
\3 Rotational kinematics
\3 Rotational energy
\3 Moments of inertia, torque
\3 Newton's Laws with rotation
\3 Work and power with rotation
\2 Angular momentum - \textbf{Chapters 11.1 - 11.3}
\3 Rolling motion
\3 Angular momentum
\3 Angular momentum conservation
\2 \textit{Problem sets:}
\3 \textit{Problem set 11: chapters 10.1 - 10.8}
\3 \textit{Problem set 12: chapters 11.1 - 11.3}
\1 \textbf{Unit 7:} \textit{Special topics: statics and gravitation}
\2 Statics - \textbf{Chapters 12.1 - 12.4}
\3 Conditions for static equilibrium
\3 Stress, strain, elastic modulus
\3 Elasticity and plasticity
\2 Gravitation - \textbf{Chapters 13.1 - 13.5}
\3 Universal Law of Gravitation
\3 Gravity near Earth's surface
\3 Gravitational potential energy
\3 Satellites, orbits, Kepler's Laws and the Solar System
\2 \textit{Problem sets:}
\3 \textit{Problem set 13: chapters 12.1 - 12.4}
\3 \textit{Problem set 14: chapters 13.1 - 13.5}
\1 \textbf{Second Midterm} - due December 8th, 2025
\2 The midterm will be posted on Moodle on Friday, December 5th.
\2 The exam will be completed at home, with an open-book/open-note policy.  The format includes multiple choice, written exercises, and design problems.
\2 The exam is due in PDF form on Monday, December 8th, to be submitted via Moodle.
\1 \textbf{Final project presentations}
\2 Presented via option A or B (see Tab. \ref{tab:grading}).
\2 Given on December 3rd and 5th
\2 Project teams can be groups of 2 or 3 students
\2 Students will complete a DIY physics experiment
\2 \textbf{Option A}: A 10-15 minute traditional presentation with several minutes for questions.
\2 \textbf{Option B}: A video in digital storytelling format using WeVideo, also 10-15 minutes long.
\2 Students will receive training and access for the WeVideo tools, which Whittier College provides for free.
\2 Presentations will be graded according to the following breakdown:
\3 For presenting results emphasizing \textit{precision and attention to detail}, with \textit{experimentally verifiable} content, 30\%.
\3 For \textit{clearly communicating} the content, including legibility and sensible graphics, 30\%.
\3 For \textit{meeting but not exceeding} the time-requirement with appropriate amount of content, 30\%.
\3 For \textit{articulating the idea with a unique style}, 10\%.
\end{outline}
\begin{figure}[ht]
\centering
\includegraphics[width=0.17\textwidth]{figures/franklin.png}
\includegraphics[width=0.13\textwidth]{figures/alzate_ramirez.jpg} \\
\includegraphics[width=0.30\textwidth]{figures/1769_transit.png} \\
\includegraphics[width=0.15\textwidth]{figures/Abbot_dAuteroche.png}
\includegraphics[width=0.15\textwidth]{figures/joaquin_v_d_Leon.png}
\caption{\label{fig:franklin} \small (Top left) Benjamin Franklin, (Top right) Father Jos\'{e} Antonio Alzate y Ram\'{i}rez, (Middle) Astronomical data from 1769 Venus transit, (Bottom left) Jean-Baptiste Chapp\'{e} d'Auteroche, (Bottom right) Joaqu\'{i}n Vel\'{a}zquez de Le\'{o}n.}
\end{figure}
\begin{figure}[hb]
\centering
\includegraphics[width=0.375\textwidth]{figures/roman_lopez_salvatierra.png}
\includegraphics[width=0.375\textwidth]{figures/venus.png}
\caption{\label{fig:franklin2} \small (Top) Carlos Rom\'{a}n, Xavier L\'{o}pez, and Fransisco Salvatierra. (Bottom) The 2012 Venus transit, from Baja California.}
\end{figure}
\end{document}

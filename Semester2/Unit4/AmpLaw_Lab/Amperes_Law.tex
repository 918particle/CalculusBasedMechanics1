\documentclass[12pt]{article}
\usepackage[margin=1.5cm]{geometry}
\usepackage{amsmath}
\usepackage{graphicx}
\title{Amp\`{e}re's Law: $\vec{B} \cdot d\vec{l}$ measurement}

\begin{document}
\maketitle

\section{Introduction}

Let $\vec{B}$ represent magnetic field in a region of space containing a current $I$ enclosed by a \textit{closed path} comprised of line segments $d\vec{l}$, and let $\mu_0 = 4\pi \times 10^{-7}$ T m/A.  Amp\`{e}re's Law states that 

\begin{equation}
\oint\vec{B} \cdot d\vec{l} = \mu_0 I
\end{equation}

The type of integral on the left hand side is called a \textit{closed line-integral}.  If the \textit{closed path} is broken into segments, and $\theta$ is the angle between the line segment and $\vec{B}$ at each line segment position, the integral is a summation

\begin{equation}
\sum_i B dl_i \cos(\theta_i) = \mu_0 I
\end{equation}

Further, if the line segments are all the same length, the $dl$ does not depend on $i$ and may be taken out of the summation:

\begin{equation}
dl \sum_i B \cos(\theta_i) = \mu_0 I
\end{equation}

Finally, $B\cos(\theta_i)$ is the component of the B-field that is parallel to the direction of the local line segment.  Labelling it $B_{||}$ gives

\begin{equation}
dl \sum_i B_{||} = \mu_0 I \label{eq:main}
\end{equation}

If measurements of $B_{||}$ are taken around a closed loop and summed, the result should be proportional to the enclosed current.

\section{Wire coils}

We have measured the number of wires in the coils to be $N = 90\pm 5$.  Let us take this to be a given for this lab exercise.  Connect a power supply to the wire coil terminals with the standard red and black wires with banana connectors.  Using the current limiting knob, place a current of 0.75 A through the coils.  The Pasco magnetic field probe should detect a current significantly higher than the background field due to the Earth.

Turn off the current.  Insert the yellow page marked with 1 cm line segments onto the coil platform.  Each line segment is 1 cm and the gap is 2 cm.  At each 1 cm line segment, measure $B_{||}$ and record the result in a list of numbers.  Pay attention to units.  Record the sum of the $B_{||}$.  This forms a \textit{systematic bias} in the data, representing the calibration offset of the magnetic probe. \\ \vspace{4cm}

\section{The Measurement}

With the 0.75 A current activated, repeat the procedure of the prior section by recording the $B_{||}$ measurements.  Record a \textbf{statistical error} for each measurement.  Finally, record the sum of the measurements, \textbf{including the associated error.}  Subtract the \textbf{systematic bias} from the total. \\ \vspace{4cm}

\section{The Interpretation}

Solving Eq. \ref{eq:main} for the current (accounting for the number of turns in the coil) gives

\begin{equation}
I_{pred} = \frac{dl}{\mu_0} \sum_i B_{||} \label{eq:main2}
\end{equation}

Assume no error for $dl$, only account for the error in the B-field.  Make sure to first convert the sum and the error in the sum to Tesla before using Eq. \ref{eq:main2}.  The result is the \textit{predicted current} $I_{pred}$.  How do the predicted and measured current compare?  Compute the percent error between the predicted current and measured current below.  Do we observe a systematic offset? \textbf{Bonus: with the current on, try measuring around a loop that does not enclose the current.  What happens?} \\ \vspace{7cm}

\end{document}
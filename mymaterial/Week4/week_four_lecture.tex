\documentclass{beamer}
\usetheme{metropolis}
\usepackage{graphicx}
\usepackage{subfig}
\usepackage{hyperref}
\usepackage{tcolorbox}
\title{Calculus-Based Physics-1: Mechanics (PHYS150-01): Week 4}
\date{September 25th - September 29th, 2017}
\author{Jordan Hanson}
\institute{Whittier College Department of Physics and Astronomy}

\begin{document}
\maketitle

\section{Week 3 Review}

\begin{frame}{Week 3 Review}
\begin{enumerate}
\item Displacement, velocity and acceleration vectors \alert{as functions of time}
\begin{itemize}
\item Breaking into components
\item Derivatives of components
\end{itemize}
\item Combining free-fall and vector components: \alert{projectile motion}
\begin{itemize}
\item The independence of velocity components
\item \textbf{Lab-activity: testing component independence}
\end{itemize}
\item Relative motion and reference frames
\begin{itemize}
\item Relative motion in one-dimension
\item Relative motion in two-dimensions
\end{itemize}
\end{enumerate}
\end{frame}

\section{Week 3 Review Problems}

\begin{frame}{Week 3 Review Problems}
\small
A pilot is performing an airdrop maneuver, in which he must release a package of supplies to land on a beach.  The plane is traveling towards the beach at a speed of 100 kilometers per hour, with an altitude of 500 meters.  How far offshore must the pilot release the supplies such that the package lands on the sandy beach and not in the water?
\begin{itemize}
\item A: 280 m 
\item B: 410 m
\item C: 100 m
\item D: 170 m
\end{itemize}
\end{frame}

\begin{frame}{Week 3 Review Problems}
\small
Suppose the pilot is flying straight, adjusting for a cross-wind of 3 m/s.  How far to the side of the flight path of the plane does the package land, assuming the package is released 280 m from the shore?
\begin{itemize}
\item A: 10.1 seconds
\item B: 5.4 seconds
\item C: 3.2 seconds
\item D: 1.1 seconds
\end{itemize}
\end{frame}

\section{Week 4 Summary}

\begin{frame}{Week 4 Summary}
\begin{figure}
\centering
\includegraphics[width=0.5\textwidth]{figures/newton.png}
\caption{\label{fig:newton} A portrait of Sir Isaac Newton.}
\end{figure}
\end{frame}

\begin{frame}{Week 4 Summary}
\begin{enumerate}
\item Deep statements about physics: \textit{dynamics} and \textit{kinematics}
\begin{itemize}
\item \textbf{Lab activity}: Force, mass and stretching springs
\end{itemize}
\item Newton's \alert{First Law}
\begin{itemize}
\item \textbf{Lab activity}: force tables
\end{itemize}
\item Newton's \alert{Second Law}
\item Newton's \alert{Third Law}
\item Applications
\begin{itemize}
\item Free-body diagrams
\item Tension
\item Inclined surfaces
\item Restoring forces
\end{itemize}
\end{enumerate}
\end{frame}

\begin{frame}{Deep statements about physics: dynamics and kinematics}
\small
\textit{Kinematics} - A \alert{description} of the motion of particles and systems \\
\textit{Dynamics} - An \alert{explanation} of the motion of particles and systems \\
\vspace{0.25cm}
What causes an object to move?  \textbf{Forces}.  Forces exist as a result of the \alert{\textbf{interactions}} of objects or systems.\\
\vspace{0.25cm}
\rule{10cm}{0.4pt} \\
\vspace{0.25cm}
\textit{Evolution} - A \alert{description} of the change of biological species \\
\textit{Natural Selection} - An \alert{explanation} of change in biological species \\
\vspace{0.25cm}
What causes species to evolve?  \textbf{Natural selection}.  Natural selection exists because of \alert{election pressures}, \alert{numerous offspring}, and \alert{variation} among offspring.
\end{frame}

\begin{frame}{What is a force, in practice?}
A force has units of \textit{Newtons}, just like distance has units of \textit{meters}.  One Newton is the force required to make an object of mass 1 kilogram accelerate by 1 m/s$^2$.\\
\vspace{1cm}
A force must also be a \textit{vector}: if a force acts on a system in a certain direction, the object will accelerate in that direction. \\
\vspace{1cm}
Force has to be related to mass in some way.
\end{frame}

\begin{frame}{What is a force, in practice?}
\small
\begin{figure}
\centering
\includegraphics[width=0.9\textwidth]{figures/force1.png}
\caption{\label{fig:force1} (a) No interaction is stretching the spring.  (b) An interaction stretches the spring a distance $\Delta x$, and the spring pulls back.  (c) A device that can compare forces by comparing $\Delta x$ for different interactions (connecting to different weights, for example).}
\end{figure}
\end{frame}

\begin{frame}{What is a force, in practice?}
\textbf{Lab activity}: Force, mass, and stretched springs.
\begin{enumerate}
\item Obtain a set of weights, a force-meter (spring), and a ruler.
\item Hang a weight from the spring, and measure the extra distance the spring stretches.
\item Repeat with different weights, recording the stretched distances alongside the weights.
\item Compute the ratio of the mass of the weight to the stretched distance in each case.  What is the result?
\end{enumerate}
\end{frame}

\begin{frame}{What is a force, in practice?}
Thus, if a force causes \textit{a system with some mass} to accelerate, the force must be \textit{proportional to that mass}.  \alert{``If it is heavier, we mush push it harder, to obtain the same acceleration.''}  \\
\vspace{1cm}
Now, let's consider all the systems for which we have described the \textit{kinematics}, where we made no use of the concept of a force...
\end{frame}

\section{Newton's First Law}

\begin{frame}{Newton's First Law}
\begin{tcolorbox}[colback=white,colframe=red!40!blue,title=Newton's First Law]
\alert{A body at rest remains at rest or, if in motion, remains in motion at constant velocity unless acted on by a net external force.}
\end{tcolorbox}
\end{frame}

\begin{frame}{Newton's First Law}
\small
For most people in the late 15th and early 16th centuries, Newton's First Law was not intuitive.  ``When have you ever seen a thing move perpetually?''\\
\vspace{0.5cm}
The key is the last phrase: ``\textit{...unless acted on by a \alert{net} external force}.''  Nothing moves unless forced, and if the \alert{net} force is zero, the velocity does not change.  Thus, if some object has a constant velocity, then it remains at that velocity unless some force (friction, air-resistance, gravity, a wall) interrupts. \\
\vspace{0.5cm}
\url{https://openstaxcollege.org/l/21forcemotion}
\end{frame}

\begin{frame}{Newton's First Law}
\small
\textbf{Lab activity}: Force tables
\begin{enumerate}
\item Obtain a set of weights, and a force-table, with ring and pulley system.
\item Using knowledge of vectors, arrange weights on the pulleys such that the ring remains stationary in the center.
\item Double one of the weights, and find the angles the strings must make to keep the ring stationary in the center.
\item Define the force vectors as vectors with magnitudes equal to the masses of the weights, in the directions of the strings.  Do the vectors add to zero?
\item Why does the ring remain stationary even though there are three forces from strings acting on it?
\end{enumerate}
\end{frame}

\begin{frame}{Newton's First Law}
\begin{figure}
\centering
\includegraphics[width=0.6\textwidth]{figures/Table.pdf}
\caption{\label{fig:table} The force table setup includes a wheel with angles, strings and pulleys, and a central ring.}
\end{figure}
\end{frame}

\begin{frame}{Newton's First Law}
Newton's First Law may be thought of in terms of the following equation:
\begin{equation}
\boxed{F_{\rm net} = \sum_{\rm i} \vec{F}_{\rm i} = 0}
\label{eq:firstlaw}
\end{equation}
In the case of the force table ring, $\vec{F}_{\rm i} \neq 0$ but $\vec{F}_{\rm net} = 0$, so we observe no velocity.  We can also have a situation with constant velocity and $\vec{F}_{\rm net} = 0$  but $\vec{F}_{\rm i} \neq 0$.
\end{frame}

\begin{frame}{Newton's First Law}
A man slides a palette crate across a concrete floor of his shop.  He exerts a force of 60.0 N, and the box has a constant velocity of 0.5 m/s.  What force cancels his pushing force, and what is the magnitude of that force?
\begin{itemize}
\item A: wind, 60.0 N
\item B: friction: 60.0 N
\item C: friction: -60.0 N
\item D: weight: -60.0 N
\end{itemize}
\end{frame}

\begin{frame}{Newton's First Law}
\small
\textit{Newton's First Law and Inertial Reference Frames}.  If an object has a constant velocity in one frame of reference, it has a constant velocity in another frame of reference moving at a constant velocity with respect to the first frame.  Consider the prior problem, and a second man is walking in the opposite direction as the palette crate at a speed of 1 m/s.  What is the speed of the palette crate from his perspective?  What is the first man's force on the palette crate from his perspective?
\begin{itemize}
\item A: 1.5 m/s, 60.0 N
\item B: 1.5 m/s, -60.0 N
\item C: 0.5 m/s, 0.0 N 
\item D: -1.5 m/s, -60.0 N
\end{itemize}
\end{frame}

\section{Newton's Second Law}

\begin{frame}{Newton's Second Law}
From the prior problem, we see that \alert{Newton's First Law} holds even under relativity, for inertial reference frames.  Let's assume we observe a system from an intertial reference frame. \\
\vspace{0.5cm}
Let us also ignore any \textit{internal forces}: forces components of the system apply to each other.  Focusing on the external forces only, we make the following two observations: \\
\end{frame}

\begin{frame}{Newton's Second Law}
\begin{figure}
\centering
\includegraphics[width=0.9\textwidth]{figures/NewtonsSecond.png}
\caption{\label{fig:newton1} A force produces greater acceleration for less massive objects.}
\end{figure}
\end{frame}

\begin{frame}{Newton's Second Law}
\begin{figure}
\centering
\includegraphics[width=0.9\textwidth]{figures/NewtonsSecond2.png}
\caption{\label{fig:newton2} A larger force produces greater acceleration for a given mass.}
\end{figure}
\end{frame}

\begin{frame}{Newton's Second Law}
\begin{tcolorbox}[colback=white,colframe=red!40!blue,title=Newton's Second Law]
\alert{The net force on a system is equal to the mass of the system multiplied by the acceleration of the system: $\vec{F}_{\rm net} = m \vec{a}$}
\end{tcolorbox}
\end{frame}

\begin{frame}{Newton's Second Law}
A man slides a palette crate across a concrete floor of his shop.  He exerts a force of 60.0 N, and friction pushes against the crate with a force of 40.0 N.  What is the acceleration of the crate, if the crate is loaded with 50.0 kg of material?
\begin{itemize}
\item A: 0.1 m/s$^2$
\item B: 1.0 m/s$^2$
\item C: 0.5 m/s$^2$
\item D: 0.4 m/s$^2$
\end{itemize}
\end{frame}

\begin{frame}{Newton's Second Law}
A man slides a palette crate across a concrete floor of his shop.  He exerts a net force of 30.0 N, and the crate accelerates at 0.5 m/s$^2$.  What mass is loaded onto the crate?
\begin{itemize}
\item A: 40.0 kg
\item B: 50.0 kg
\item C: 60.0 kg
\item D: 70.0 kg
\end{itemize}
\end{frame}

\begin{frame}{Newton's Second Law}
\begin{figure}
\centering
\includegraphics[width=0.5\textwidth]{figures/NewtonsSecond3.png}
\caption{\label{fig:newton3} An example of a \textit{free body diagram}, which summarizes all external forces on a system.}
\end{figure}
\end{frame}

\begin{frame}{Newton's Second Law}
\begin{figure}
\centering
\includegraphics[width=0.5\textwidth,trim=0.75cm 0.5cm 0.75cm 0.5cm,clip=true]{figures/NetForce.pdf}
\caption{\label{fig:fbd} The \textit{free-body diagram} is just a vector summation problem.  The \textit{normal force}, \textbf{N}, acts against the force of gravity, \textbf{w}, according to Netwon's Third Law.}
\end{figure}
\end{frame}

\begin{frame}{Newton's Second Law}
Indiana Jones is running through the rainforest, and he wades into a pit of quicksand.  He has a mass of 70 kg, however the normal force is only 400.0 N.  He pushes forward with a force of 250.0 N, and the quicksand sucks him backwards with a force of 50.0 N.  What is the net force on Indiana?  The force of gravity is his mass times $g$, in the downward direction.
\begin{itemize}
\item A: (200.0, -290) N
\item B: (50.0, -690) N
\item C: (200.0, 290) N
\item D: (150.0, 690) N
\end{itemize}
\end{frame}

\begin{frame}{Newton's Second Law}
What is the net acceleration on Indiana, from the previous example?
\begin{itemize}
\item A: 1.0 m/s$^2$
\item B: 3.0 m/s$^2$
\item C: 4.0 m/s$^2$
\item D: 5.0 m/s$^2$
\end{itemize}
\end{frame}

\begin{frame}{Newton's Second Law}
Notice that if we define $\vec{p} = m\vec{v}$, we may write \\
\vspace{0.5cm}
\begin{equation}
\vec{F}_{\rm net} = \frac{d\vec{p}_{\rm net}}{dt} = m \frac{d\vec{v}}{dt} = m \vec{a}_{\rm net}
\end{equation} \\
\vspace{0.5cm}
In words: the force is the derivative of the \textit{momentum}, $\vec{p}$.
\end{frame}

\begin{frame}{Newton's Second Law}
A particle of mass $m$ is falling under the influence of gravity, but experiences a thrust force upwards: $\vec{F}_{\rm t} = kt$, making the net force $\vec{F}_{\rm net} = kt - mg$.  Express the velocity as a function of time, assuming the velocity is $v_{\rm 0}$ at $t=0$.
\begin{itemize}
\item A: $v(t) = \frac{1}{2} \frac{k}{m} t^2+v_{\rm 0}$
\item B:  $v(t) = \frac{k}{m} t - gt$
\item C:  $v(t) = \frac{k}{m} t - gt +v_{\rm 0}$
\item D: $v(t) = \frac{1}{2} \frac{k}{m} t^2 - gt +v_{\rm 0}$
\end{itemize}
\vspace{0.5cm}
We will return to the concept of \textit{momentum} in the next few chapters...
\end{frame}

\begin{frame}{Newton's Second Law}
There is a difference between \alert{\textit{mass}}, and \alert{\textit{weight}}.  \textit{Mass} is proportional to the number of atoms in an object.  \textit{Weight} is a force derived from Newton's second law (mass times acceleration), assuming the acceleration due to gravity is constant.  If an object has a mass of 1 kilogram, and the Earth's gravitational acceleration is $\approx 10$ m/s$^2$, then it \textit{weighs} $\approx 10.0$ N (10.0 kg m/s$^2$).
\end{frame}

\begin{frame}{Newton's Second Law}
Astronauts are walking on the moon, and lifting moon rocks into cannisters.  On Earth, the cannisters weighed 40.0 N each, and each astronaut is expected to carry two of them, loaded each with 65.0 kg of moon rock.  The acceleration due to gravity on the Moon is about 17\% of that on Earth.  What weight are these astronauts expected to carry?
\begin{itemize}
\item A: 230 N
\item B: 200 N
\item C: 1000 N
\item D: 170 N
\end{itemize}
\end{frame}

\section{Conclusion}

\section{Answers}

\begin{frame}{Answers}
\begin{columns}[T]
\begin{column}{0.5\textwidth}
\begin{itemize}
\item 280 m
\item 10.1 seconds
\item friction: -60.0 N
\item 1.5 m/s, 60.0 N
\item 0.4 m/s$^2$
\item 60 kg
\item (200.0, -290) N
\item 5.0 m/s$^2$
\item $v(t) = \frac{1}{2} \frac{k}{m} t^2 - gt +v_{\rm 0}$
\item 230 N
\end{itemize}
\end{column}
\begin{column}{0.5\textwidth}
\begin{itemize}
\item
\end{itemize}
\end{column}
\end{columns}
\end{frame}

\end{document}

\title{Syllabus for Calculus-Based Physics-1: Mechanics (PHYS150-01)}
\author{Dr. Jordan Hanson - Whittier College Dept. of Physics and Astronomy}
\date{\today}
\documentclass[10pt]{article}
\usepackage[a4paper, total={18cm, 27cm}]{geometry}
\usepackage{outlines}
\usepackage[sfdefault]{FiraSans}

\begin{document}
\maketitle

\begin{abstract}
The concepts of calculus-based mechanics will be presented within the context of interactive problem-solving.  First, the concepts of displacement, velocity, and acceleration in one and two dimensions will be introduced, building up to Newton's Laws of motion.  Next, the concepts of friction and rotational motion will be added.  More complex problems will be introduced through the conservation of energy and linear momentum, followed by the rotational equivalents.  The course work will include interactive computational exercises, analytic textbook problems, and lab-based activities.
\end{abstract}
\noindent
\textit{\textbf{Regular course hours}: Monday, Wednesday, and Friday from 12:50 - 14:20.} \\
\textit{\textbf{Office hours}: Mondays and Tuesdays at 15:00.} \\
\textit{\textbf{Text}: University Physics Volume One (openstax.org) -  http://cnx.org/content/col12031/latest} \\
\textit{\textbf{Grading}: There will be four 45-minute midterms, each examining conceptual understanding in step-by-step problems.  Each midterm is worth 12.5\% of the final grade.  The ExpertTA system (\$30.00 online) will manage the homework sets, which are due weekly and worth 20\% of the grade.  Interactive in-class activities will be worth 15\% of the final grade.  The final exam will be held on December 13th for a full class-period, and will be worth 15\% of the grade.}

\begin{outline}[enumerate]
\1 Week 1 - September 6th through September 8th - \textbf{Chapter 1, Chapters 2.1-2.3, 3.1}
\2 Estimations, approximations, unit-analysis, and coordinate systems
\2 Adding and subtracting vectors, displacement and translational motion
\2 \textit{Mathematics review: limits, derivatives, integrals}
\1 Week 2 - September 11th through September 15th - \textbf{Chapters 3.2-3.6}
\2 Instantaneous velocity and acceleration
\3 \textit{Mathematics review: single-variable differentiation}
\2 Accelerration due to gravity near the Earth's surface and other constant accelerations
\2 Common equations of motion for constant acceleration	
\1 Week 3 - September 18th through September 22nd - \textbf{Chapters 4.1-4.3,4.5}
\2 Velocity and acceleration vectors, vector components
\2 Combination of vector components and free-fall: projectile motion
\2 Relative motion, reference frames
\1 First midterm exam, beginning of week 4
\2 Estimations, approximations, unit-analysis, and basic calculus problems
\2 Displacement, velocity, and constant acceleration
\2 Vectors and relative motion
\1 Week 4 - September 25th through September 29th - \textbf{Chapters 5.1-5.7}
\2 Newton's First Law
\2 Newton's Second Law
\2 Newton's Third Law
\2 Examples and free-body diagrams
\1 Week 5 - October 2nd through October 6th - \textbf{Chapters 4.4, 6.2-6.4}
\2 Uniform circular motion, centripedal acceleration, centripedal force
\2 Frictional forces
\2 Drag
\clearpage
\1 Second midterm exam, beginning of week 6
\2 Newton's laws and free-body diagrams
\2 Circular motion
\2 Friction
\1 Week 6 - October 9th through October 13th - \textbf{Chapters 7.1-7.4, 8.1}
\2 Work, kinetic and potential energy
\2 Power
\3 \textit{Mathematics review: single-variable differentiation and integration}
\2 Potential energy
\1 Week 7 - October 16th through October 19th - \textbf{Chapters 8.2-8.3}
\2 Definitions of a conservative force
\2 Conservation of energy
\3 \textit{Mathematics review: the fundamental theorem of calculus in one variable}
\3 Graphical and other notions of energy conservation
\1 Mid-semester break, October 20th
\1 Week 8 - October 23rd through October 27th - \textbf{Chapters 9.1-9.6}
\2 Linear momentum and momentum conservation
\2 Scattering in one and two dimensions
\2 Center of mass
\1 Third midterm exam, beginning of week 9
\2 Work, kinetic energy, and conservative forces
\2 Momentum conservation and scattering
\1 Week 9 - October 30th through November 3rd - \textbf{Chapters 10.1-10.5}
\2 Rotational variables and rotational acceleration
\2 Moments of inertia and rotational kinetic energy
\3 \textit{Mathematics review: volume integrals}
\1 Week 10 - November 6th through November 10th - \textbf{Chapters 10.6 - 10.8}
\2 Torque and Newton's Laws in rotating systems
\2 Work and energy in rotating systems
\1 Week 11 - November 13th through November 17th - \textbf{Chapters 11.1-11.4}
\2 Rolling motion
\2 Angular momentum and angular momentum conservation
\1 Forth midterm exam, beginning of week 12
\2 Rotational motion, moments of inertia, rotational kinetic energy
\2 Torque and Newton's Laws in rotational form
\2 Angular momentum conservation
\1 Week 12 - November 20th and November 21st - \textbf{Review days}
\1 Week 13 - November 27th through December 1st  - \textbf{Programming demonstrations}
\2 Problems involving constant acceleration - \textit{lunar lander}
\2 Projectile motion - \textit{bellicose birds}
\1 Week 14 - December 4th through December 8th - \textbf{Further programming demonstrations and final review}
\2 Linear momentum conservation and scattering - \textit{planetoids}
\2 Preparing for the final exam
\2 Covering topics deemed necessary by classmates
\end{outline}
\end{document}
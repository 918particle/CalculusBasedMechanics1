\title{Project Overview and Assessment: Edwin Flores, Addison Mock, and Alex Trujillo}
\author{Dr. Jordan Hanson - Whittier College Dept. of Physics and Astronomy}
\date{\today}
\documentclass[10pt]{article}
\usepackage[a4paper, total={18cm, 27cm}]{geometry}
\usepackage{outlines}
\usepackage[sfdefault]{FiraSans}
\usepackage{hyperref}
\usepackage{graphicx}

\begin{document}
\maketitle

\begin{abstract}
This was a two-stage examination of forces and kinematics.  First, the researchers had to measure the constant of proportionality between force and displacement in an exercise band.  The measured k value was 95.4 N/m.  Using this value, the goal was to predict the horizontal range of a projectile launched with this force.  While the calculations appear to be in order, several systematics impeded a successful launch.  Friction between the balloon pouch and the wood, and the fact that the balloon stores energy in its changing shape are the two main issues that come to mind.  Third, the exercise band may not be linear in force versus distance.
\end{abstract}

\textit{Score} - \textbf{9 of 10 points.}

\textit{Project Assessment}
\begin{outline}[enumerate]
\1 Introduction of Concepts, Hypothesis
\2 There was a quantitative hypothesis with well-introduced concepts
\1 Explanation of the Experiment, with Diagram or Picture
\2 The presenters showed pictures and diagrams
\1 Presentation of Data and Systematics
\2 The presentation contained a thorough, quantitative derivation of the prediction, and showed that the data did not match.
\1 Conclusion
\2 The conclusion stuck with the data, rather than attempt to argue that data and hypothesis agreed.  The data did not match, but systematic errors were assessed.
\end{outline}
\end{document}

\documentclass{beamer}
\usetheme{metropolis}
\usepackage{graphicx}
\usepackage{subfig}
\usepackage{hyperref}
\usepackage{tcolorbox}
\title{Calculus-Based Physics-1: Mechanics (PHYS150-01): Week 7}
\date{October 16th - October 20th, 2017}
\author{Jordan Hanson}
\institute{Whittier College Department of Physics and Astronomy}

\begin{document}
\maketitle

\section{Week 6 Review}

\begin{frame}{Week 6 Review}
\begin{enumerate}
\item \alert{Work} has a scientifically precise definition
\begin{itemize}
\item Units
\item As a product of force and displacement vectors
\end{itemize}
\item Kinetic Energy and the \alert{Work-Energy Theorem}
\item Gravitational potential energy
\begin{itemize}
\item Potential energy
\item \textit{Simplifying otherwise complex calculations}
\item Potential energy near Earth's surface
\item ...in space
\end{itemize}
\item Definition of a \textbf{conservative force}
\begin{itemize}
\item Relationship between conservative forces and potential energy
\item Conservation of energy for conservative forces
\end{itemize}
\end{enumerate}
\end{frame}

\section{Week 6 Review Problems}

\begin{frame}{Week 6 Review Problem}
Recall that the \textit{gravitational potential energy} associated with an object of mass $m$ located a height $y$ above ground level is $U = mgy$.  What is $-dU/dy$?
\begin{itemize}
\item A: $mg$
\item B: $-mg$
\item C: $g$
\item D: $-g$
\end{itemize}
\end{frame}

\begin{frame}{Week 6 Review Problem}
What is the physical meaning of the quantity $-mg =-dU/dy$?
\begin{itemize}
\item A: The force of gravity 
\item B: The acceleration due to gravity
\item C: The mass
\item D: Change in kinetic energy
\end{itemize}
\end{frame}

\section{Week 7 Summary}

\begin{frame}{Week 6 Summary}
\begin{enumerate}
\item things and stuff
\end{enumerate}
\end{frame}

\section{Conclusion}

\section{Answers}

\begin{frame}{Answers}
\begin{columns}[T]
\begin{column}{0.5\textwidth}
\begin{itemize}
\item $-mg$
\item The force of gravity 
\end{itemize}
\end{column}
\begin{column}{0.5\textwidth}
\begin{itemize}
\item ...
\end{itemize}
\end{column}
\end{columns}
\end{frame}

\end{document}

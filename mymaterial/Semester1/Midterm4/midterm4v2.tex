\title{Midterm 4 for Calculus-Based Physics-1: Mechanics (PHYS150-01)}
\author{Dr. Jordan Hanson - Whittier College Dept. of Physics and Astronomy}
\date{November 27th, 2017}
\documentclass[10pt]{article}
\usepackage[a4paper, total={18cm, 27cm}]{geometry}
\usepackage{outlines}
\usepackage[sfdefault]{FiraSans}
\usepackage{graphicx}

\begin{document}
\maketitle

\section{Definition of Momentum}
\begin{enumerate}
\item Which of the following quantities has units of momentum?
\begin{itemize}
\item 4 kg m/s
\item 8 kg m/s$^2$
\item 10 N m
\item 3 m/s
\end{itemize}
\item Which two properties of a system must be true for momentum to be conserved?
\begin{itemize}
\item The mass must be constant, and there must be no dissipative forces.
\item There must be no net external force, and the velocity vectors must sum to zero.
\item The mass must be constant, and there must be no net external force.
\item There must be no internal forces and $dP/dt \neq 0$.
\end{itemize}
\item In which of the following situations will momentum be conserved?
\begin{itemize}
\item After a rocket has launched vertically upward, and has run out of fuel.
\item Two asteroids collide in space and merge into one asteroid.
\item A person floats downward at constant velocity using a parachute.
\item An object skids to a stop due to surface friction.
\end{itemize}
\item What is the momentum of a 1 kg meteorite traveling through space at 10 km/s? \\ \vspace{1cm}
\end{enumerate}
\section{Conservation of Momentum}
\begin{enumerate}
\item A 200 gram meteorite breaks into two smaller ones.  The two final velocities are observed to be equal, and in the same direction as the original velocity.  What are the masses of the two pieces, if momentum is conserved? \\ \vspace{1.5cm}
\item A proton and a neutron interact and stick together.  The mass of both particles is identical and cannot change.  If one has a velocity of $v_{\rm 1} = 10^{6}$ m/s, and the other has $v_{\rm 2} = 0$ m/s, what is the final velocity?  Draw a diagram of the initial state and the final state.  \\ \vspace{1.5 cm}
\end{enumerate}
\section{Classifying Interactions}
\begin{enumerate}
\item What quantity is conserved in an \textit{elastic} interaction but not in an \textit{inelastic} interaction?
\begin{itemize}
\item Acceleration (remains zero)
\item Momentum
\item Internal forces
\item Kinetic energy
\end{itemize}
\item If an interaction is \textit{totally} inelastic, what is the final velocity?
\begin{itemize}
\item Less than the initial velocity
\item Such that the kinetic energy decreases
\item Such that the final momentum is zero
\item The final velocity is zero
\end{itemize}
\item The notation $n \rightarrow n$ means many-to-many, or several particles scattering off of each other.  Which of the following is true of $n \rightarrow n$ interactions?
\begin{itemize}
\item They are usually elastic
\item They can be totally elastic
\item They must be inelastic
\item They can be totally inelastic
\end{itemize}
\item The notation $n \rightarrow 1$ means many-to-one, or several particles interacting and becoming one system.  Which of the following is true of $n \rightarrow n$ interactions?
\begin{itemize}
\item The are usually inelastic
\item They are always totally inelastic
\item They must be elastic
\item Momentum is not conserved if there are internal forces
\end{itemize}
\item A 1 kg particle has $v_{\rm 1} = 0$ m/s, and it interacts with a 2 kg particle with velocity $v_{\rm 2} = 4$ m/s.  If the collision is inelastic, what is the final velocity of the combined system? \\ \vspace{1.5cm}
\item A 2 kg particle has $v_{\rm 1} = -2$ m/s, and it interacts with a 2 kg particle with velocity $v_{\rm 2} = 2$ m/s.  If the collision is elastic, what is the final velocity of each particle? \\ \vspace{1.5 cm}
\end{enumerate}
\end{document}
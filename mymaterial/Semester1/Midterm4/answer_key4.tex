\title{Answer Key for Calculus-Based Physics-1: Mechanics (PHYS150-01)}
\author{Dr. Jordan Hanson - Whittier College Dept. of Physics and Astronomy}
\date{November 27th, 2017}
\documentclass[10pt]{article}
\usepackage[a4paper, total={18cm, 27cm}]{geometry}
\usepackage{outlines}
\usepackage[sfdefault]{FiraSans}
\usepackage{graphicx}

\begin{document}
\maketitle

\section{Definition of Momentum}
\begin{enumerate}
\item Which of the following quantities has units of momentum?
\begin{itemize}
\item 4 m/s
\item \textbf{8 kg m/s.}  \textit{Momentum is mass times velocity, so we need the answer with kg times m/s.}
\item 10 m/s$^2$
\item 3 N m
\end{itemize}
\item Which two properties of a system must be true for momentum to be conserved?
\begin{itemize}
\item \textbf{The mass must be constant, and there must be no net external force.}  \textit{The masses must be held constant, otherwise individual momenta could change independently of the total.  The change in total momentum is the net external force, so if there is a net external force then momentum will change.}
\item The mass must be constant, and there must be no internal forces.
\item There must be no net external force, and the velocity vectors must sum to zero.
\item There must be no internal forces and no net external force.
\end{itemize}
\item In which of the following situations will momentum be conserved?
\begin{itemize}
\item An object skids to a stop due to surface friction.
\item After a rocket has launched vertically upward, and has run out of fuel.
\item \textbf{A hockey puck glides along frictionless ice and rebounds off of a wall.}  \textit{The ice is frictionless and therefore no external net force.  The mass of the hockey puck does not change.  This is a $1 \rightarrow 1$ interaction.}
\item A sheet of paper falls to the ground at constant velocity and then rests on the ground.  \textit{One could also make a case for this answer.  There is no net force because the sheet moves at constant velocity.  It could be argued that it undergoes an inelastic collision with the Earth.}
\end{itemize}
\item What is the momentum of a 10 kg meteorite traveling through space at 1 km/s? \\ \\
\textbf{Beginning with the definition, $\vec{p} = m\vec{v}$, the magnitude of the momentum is 10 kg $\times$ 1000 m/s, or $10^{4}$ kg m/s.}
\end{enumerate}
\section{Conservation of Momentum}
\begin{enumerate}
\item A 100 gram meteorite breaks into two smaller ones.  The two final velocities are observed to be equal, and in the same direction as the original velocity.  What are the masses of the two pieces, if momentum is conserved? \\ \\
\textbf{$\vec{P}_{\rm f} = \vec{P}_{\rm i}$.  The initial momentum is the total mass $M$ multiplied by some initial velocity $v$ (in the positive x-direction).  The final velocities $v'$ are equal and also in the positive x-direction, and the masses of the pieces can be $m_{\rm 1}$ and $m_{\rm 2}$.  We have $Mv = m_{\rm 1} v' + m_{\rm 2} v' = (m_{\rm 1} + m_{\rm 2}) v'$.  Since $M = m_{\rm 1} + m_{\rm 2}$, we find $v = v'$.  The pieces of meteorite move at the same speed as the original.  However, this does not yet constrain the masses of the pieces, as long as $M = m_{\rm 1} + m_{\rm 2}$.  Suppose we are moving alongside the meteor as it splits, and we observe the pieces move apart very slowly in an equal and opposite fashion in the plus and minus y-direction.  In this case, each mass would have to be 50 grams, because momentum is conserved \textit{no matter what frame in which it is observed}.}
\item Two protons interact and scatter off of each other.  The mass of both particles cannot change.  If one proton has a velocity of $v_{\rm 1} = -10^{6}$ m/s, and the other has $v_{\rm 2} = 10^{6}$ m/s, what are the final velocities?  Draw a diagram of the initial state and the final state.  \\ \\
\textbf{$\vec{P}_{\rm f} = \vec{P}_{\rm i}$.  The initial total momentum is $\vec{P}_{\rm i} = m\vec{v}_{\rm 1} + m\vec{v}_{\rm 2} = m\vec{v}_{\rm 1} - m\vec{v}_{\rm 1} = 0$.  The final total momentum is therefore $0 = m\vec{v}_{\rm 1}' + m\vec{v}_{\rm 2}'$, implying that $\vec{v}_{\rm 2}' = -\vec{v}_{\rm 1}'$.  If the collision is \textit{inelastic}, kinetic energy decreases and the particles can have any velocity combination that sums to zero.  If it \textit{elastic}, then kinetic energy is conserved and $v_{\rm 1}^2 + v_{\rm 2}^2 = v_{\rm 1}'^2 + v_{\rm 2}'^2 \rightarrow v_{\rm 1}^2 = v_{\rm 1}'^2$ and $v_{\rm 2}^2 = v_{\rm 2}'^2$ ($v_{\rm 1} = -v_{\rm 2}$ and $\vec{v}_{\rm 2}' = -\vec{v}_{\rm 1}'$).  Taking the square roots, we find (for example) $v_{\rm 1}' = \pm v_{\rm 1}$.  The positive root corresponds to the case of no scattering (proton 1 passes proton 2 without changing direction).  The negative root corresponds to elastic scattering.  In this case, $v_{\rm 1}' = 10^6$ m/s and $v_{\rm 2}' = -10^6$ m/s.}
\end{enumerate}
\section{Classifying Interactions}
\begin{enumerate}
\item What quantity is conserved in an \textit{elastic} interaction but not in an \textit{inelastic} interaction?
\begin{itemize}
\item Gravitational potential energy
\item \textbf{Kinetic energy}  \textit{Kinetic energy is conserved in an elastic collision.}
\item Momentum
\item Internal forces
\end{itemize}
\item If an interaction is \textit{totally} inelastic, what is the final velocity?
\begin{itemize}
\item Less than the initial velocity
\item Such that the kinetic energy decreases
\item Such that the final momentum is zero
\item \textbf{The final velocity is zero}  \textit{All kinetic energy disappears.}
\end{itemize}
\item The notation $n \rightarrow n$ means many-to-many, or several particles scattering off of each other.  Which of the following is true of $n \rightarrow n$ interactions?
\begin{itemize}
\item \textbf{They must be at least partially elastic}  \textit{This is the best answer.  Interactions of this type usually conserve some fraction of the kinetic energy, if not all of it.  In other words $K_{\rm f} = K_{\rm i}$ or $K_{\rm f} \leq K_{\rm i}$.  An example is two particles strike each other and bounce away.  If one absorbs some energy, kinetic energy decreases.  The maximum decrease occurs when the two objects fully absorb the kinetic energy by becoming one object.}
\item \textbf{They can only be totally elastic}  \textit{This is a special case.}
\item They must be inelastic
\item They must be totally inelastic
\end{itemize}
\item The notation $n \rightarrow n$ means many-to-many, or several particles scattering off of each other.  Which of the following is true of $n \rightarrow n$ interactions?
\begin{itemize}
\item \textbf{They must be elastic}  \textit{This is a good answer, but these interactions don't \textit{have} to be completely elastic.}
\item They must be inelastic \textit{This cannot be the answer.  At least some of the particles would have to stick together to lower the kinetic energy.}
\item \textbf{They cannot be inelastic}  \textit{This is the best answer.}
\end{itemize}
\item A 1 kg particle has $v_{\rm 1} = 0$ m/s, and it interacts with a 2 kg particle with velocity $v_{\rm 2} = 2$ m/s.  If the collision is inelastic, what is the final velocity of the combined system? \\ \\
\textbf{The initial momentum is due entirely to the second particle: $\vec{P}_{\rm i} = 4\hat{i}$ kg m/s.  The final momentum of the combined system is  $\vec{P}_{\rm f} = (m_{\rm 1} + m_{\rm 2})v\hat{i}$ kg m/s = 4 kg m/s.  The total mass $(m_{\rm 1} + m_{\rm 2})$ is 3 kg, so v = $4/3$ kg m/s.  This answer makes sense because it is smaller than 2 m/s, the initial velocity.}
\item A 1 kg particle has $v_{\rm 1} = -1$ m/s, and it interacts with a 1 kg particle with velocity $v_{\rm 2} = 1$ m/s.  If the collision is elastic, what is the final velocity of each particle? \\ \\
\textbf{The intuitive answer is correct: each particle rebounds in the opposite direction with the same speed.  To show this, note that the initial total momentum is zero, so the final velocities must be equal and opposite (equal masses): $v_{\rm 1}' = -v_{\rm 2}'$.  Kinetic energy is conserved, and the masses are equal, so $v_{\rm 1}^2 + v_{\rm 2}^2 = v_{\rm 1}'^2 + v_{\rm 2}'^2$.  The magnitudes of the initial velocities are both 1 m/s, and the magnitudes of the final velocites are equal by momentum conservation.  Thus, $2v_{\rm 1}^2 = 2v_{\rm 1}'^2 \rightarrow v_{\rm 1} = \pm v_{\rm 1}'$.  Choosing the correct roots yields $v_{\rm 1}' = 1$ m/s and $v_{\rm 2}' = -1$ m/s.}
\end{enumerate}
\end{document}
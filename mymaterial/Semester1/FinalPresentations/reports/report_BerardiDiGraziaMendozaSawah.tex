\title{Project Overview and Assessment: Kristen Berardi, Cole DiGrazia, Aaron Mendoza, and Naya Sawah}
\author{Dr. Jordan Hanson - Whittier College Dept. of Physics and Astronomy}
\date{\today}
\documentclass[10pt]{article}
\usepackage[a4paper, total={18cm, 27cm}]{geometry}
\usepackage{outlines}
\usepackage[sfdefault]{FiraSans}
\usepackage{hyperref}
\usepackage{graphicx}

\begin{document}
\maketitle

\begin{abstract}
This was a challenging project.  The concept of the shear modulus of wood was involved, and the researchers did a nice job of combining wood data to make a decent estimate of this number.  The hypothesis was stated as the force required to bend wood to some ``compression distance'' that would lead to fracture.  Had the researchers secured the other end of the ruler to the table, the compression distance would likely have been reached by the bending caused by the sandbags.  However, the other part of the setup involved (subtely) drag forces and pressure on newspaper.  I believe the compression distance was not reached because of impulse considerations; the pressure on the paper didn't keep it fully stationary, but drag did not have time to build up either.
\end{abstract}

\textit{Score} - \textbf{9 of 10 points.}

\textit{Project Assessment}
\begin{outline}[enumerate]
\1 Introduction of Concepts, Hypothesis
\2 The hypothesis was quantitative, and involved researching shear modulii.
\1 Explanation of the Experiment, with Diagram or Picture
\2 The experiment was explained sufficiently in words and diagrams.
\1 Presentation of Data and Systematics
\2 There were 10 trials, but no quantitative data was recorded other than the weights (120 N, and 220 N) of the sandbags.
\1 Conclusion
\2 The conclusion matched the data, and systematic errors were considered (although not necessarily the subtle issue described in the abstract).
\end{outline}
\end{document}

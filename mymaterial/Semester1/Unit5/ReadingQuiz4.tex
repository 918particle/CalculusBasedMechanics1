\documentclass{article}
\usepackage{graphicx}
\usepackage[margin=1.5cm]{geometry}
\usepackage{amsmath}

\begin{document}

\title{Warm Up Exercises: Circular Motion}
\author{Prof. Jordan C. Hanson}

\maketitle

\section{Memory Bank}

\begin{itemize}
\item Force of drag, in air or other gas: $F_D = \frac{1}{2}C \rho A v^2$.
\item In the above formula, $C$ is an empirical constant, $\rho$ is the density of the air or gas, $A$ is the area of the object, and $v$ is the object's velocity.
\item Circular motion position with angular velocity $\omega = \Delta \theta / \Delta t$:
\begin{equation}
\vec{r}(t) = r\cos(\omega t)\hat{i} + r\sin(\omega t)\hat{j}
\end{equation}
\item $a_{\rm C} = r \omega^2$ ... Centripetal force.
\item $v = r\omega$ ... Radial velocity.
\item $\vec{s} = - k \Delta \vec{x}$ ... Spring force.
\end{itemize}

\section{Circular Motion}
\begin{enumerate}
\item Suppose a system is rotating about the origin with a radius $r = 1.0$ m, and angular speed $\omega = \pi$ radians per second. (a) How long does the object take to complete one rotation?  (b) What is the angular velocity in rotations per minute? (c) Where is the system at $t = 0.5,~1.5,~2.5, ...$ seconds? \\ \vspace{2cm}
\item In the prior problem, if the mass of the system is 30 kg, what is $F_{\rm C} = m ~ a_{\rm C}$? \\ \vspace{1cm}
\end{enumerate}

\section{Spring Forces}
\begin{enumerate}
\item Suppose a 0.25 kg mass is hung from a spring, and the spring stretches 0.1 m.  What is the spring constant, $k$? \\ \vspace{1cm}
\item Assume the spring constant from the prior problem, but now assume the mass is stretching the spring along a 45 degree incline plane (no friction).  What is the new $\Delta x$?
\end{enumerate}

\end{document}

\documentclass{article}
\usepackage{graphicx}
\usepackage[margin=1.5cm]{geometry}
\usepackage{amsmath}

\begin{document}

\title{Forces and Inclines \textit{with Friction}}
\author{Prof. Jordan C. Hanson}

\maketitle

\section{Review of Friction Force}
The friction force is $f = \mu N$.  That is, the force is directly proportional to the \textit{normal force}, and the constant of proportionality is $\mu$. If the object is not moving, then $\mu$ is the static coefficient of friction.  If the object is \textit{sliding}, then $\mu$ is the kinetic coefficient of friction.  If $N = mg$, then $f = \mu m g$ and the direction of $f$ will be in the direction opposing motion.
\section{Inclined Surfaces with Static Friction}
Place a mass on the ruler and incline the ruler at some small angle.  Draw a free body diagram below summarizing the weight, friction, and normal forces on the mass.  Use the free body diagram to show that $\mu = \tan(\theta)$, if $\mu$ is the static coefficient of friction. \\ \vspace{3cm}
\section{Measurement of $\mu$}
For several masses, measure $\mu$ by measuring the largest possible $\theta$ such that the mass does not slide down the ruler.  Create a plot of $\mu$ versus mass.  Does $\mu$ depend on mass?
\end{document}

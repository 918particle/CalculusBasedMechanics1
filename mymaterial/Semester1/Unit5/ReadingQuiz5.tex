\documentclass{article}
\usepackage{graphicx}
\usepackage[margin=1.5cm]{geometry}
\usepackage{amsmath}

\begin{document}

\title{Warm Up Exercises: Springs, Inclines, and Homework Exercises with Friction}
\author{Prof. Jordan C. Hanson}

\maketitle

\section{Memory Bank}

\begin{itemize}
\item $\vec{s} = - k \Delta \vec{x}$ ... Spring force.
\item $w_x = mg \sin\theta$ ... Weight down the incline, for incline planes.
\item $w_y = mg \cos\theta$ ... Weight perpendicular to surface, for incline planes.
\item $f = \mu N$ ... Force of friction.
\end{itemize}

\section{Force of Friction}
\begin{enumerate}
\item Suppose 10 sled dogs pull a dogsled across snow.  The waxed wood of the sled runners has a coefficient of kinetic friction against dry snow of 0.08.  The combined weight of the sled and rider is 250 kg. (a) What is the acceleration, if each dog pulls with a force of 40 N? (b) How long does it take for the system to reach 8 m/s, if initial speed is zero? (c) Proceeding at 8 m/s, how long would it take for the system to travel 10 km? \\ \vspace{2.5cm}
\end{enumerate}

\section{Spring Forces and Inclines}
\begin{enumerate}
\item Suppose a 0.5 kg mass is hung from a spring, and the spring stretches 0.5 m.  (a) What is the spring constant, $k$? (b) Assume the spring constant from the prior problem, but now assume the mass is stretching the spring along a 30 degree incline plane (no friction).  What is the new $\Delta x$? (c) Now assume there is a \textit{static} coefficient of friction of 0.1 between the mass and the plane.  What is the new $\Delta x$?
\end{enumerate}

\end{document}

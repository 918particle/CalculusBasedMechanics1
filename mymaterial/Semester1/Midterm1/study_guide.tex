\title{Study Guide for Midterm 1}
\author{Dr. Jordan Hanson - Whittier College Dept. of Physics and Astronomy}
\date{\today}
\documentclass[10pt]{article}
\usepackage[margin=1.5cm]{geometry}
\usepackage{outlines}
\usepackage{graphicx}
\usepackage{amsmath}

\begin{document}
\maketitle

\section{Memory Bank}

\begin{itemize}
\item Unit conversions: 1 km = 1000 m, 1 m = 100 cm, 1 hr = 3600 s, 1 year = $\pi \times 10^7$ s, 1 g/cm$^3$ = 1000 kg/m$^3$.
\item $\vec{x} = a \hat{i} + b\hat{j}$ ... Component form of a two-dimensional vector.
\item $|\vec{x}| = \sqrt{a^2+b^2}$ ... Pythagorean theorem for obtaining vector magnitude.
\item $\theta = \tan^{-1}(b/a)$ ... Obtaining the angle between vector and x-axis.
\item $a = |\vec{x}|\cos(\theta)$ ... Obtaining the x-component with trigonometry.
\item $b = |\vec{x}|\sin(\theta)$ ... Obtaining the y-component with trigonometry.
\item $\Delta x = \vec{x}_f - \vec{x}_i$ ... Definition of displacement.
\item $\vec{v} = \frac{\Delta \vec{x}}{\Delta t} = \frac{\vec{x}_f - \vec{x}_i}{t_f-t_i}$ ... Definition of velocity.
\item $\vec{a} = \frac{\Delta \vec{v}}{\Delta t} = \frac{\vec{v}_f - \vec{v}_i}{t_f-t_i}$ ... Definition of acceleration.
\item $x(t) = x_i + v t$ ... Velocity is the slope of position versus time.
\item $x(t) = \frac{1}{2} a t^2 + v_i t + x_i$ ... With constant acceleration, position is quadratic.  If $a=0$ this becomes the prior function.
\item $v(t) = v_i + a t$ ... With constant acceleration, acceleration is the slope of velocity.
\item $v^2 = v_i^2 + 2 a \Delta x$ ... The kinematic equation without time, assuming constant acceleration.
\item $\vec{v}(t) = \frac{d\vec{x}}{dt}$ ... General calculus-based definition of velocity.
\item $\vec{a}(t) = \frac{d\vec{v}}{dt}$ ... General calculus-based definition of acceleration.
\item General set of 2D kinematic equations, assuming gravity provides constant vertical negative acceleration.
\begin{align}
\vec{x}(t) &= (x_i + v_{x,i} t) \hat{i} \\
\vec{y}(t) &= (-\frac{1}{2}g t^2 + v_{i,y} t + y_i) \hat{j} \\
\vec{v}_y &= (v_{i,y} - g t) \hat{j} \\
\vec{a} &= -g \hat{j} \\
T_{tof} &= \frac{2 v_0\sin(\theta_0)}{g} \\
R &= \frac{v_0^2\sin(\theta_0)}{g} \\
v_{x,i} &= v_0 \cos(\theta) \\
v_{y,i} &= v_0 \sin(\theta)
\end{align}
\end{itemize}

\clearpage

\section{Chapter 1: Estimations and Unit Analysis}

\begin{enumerate}
\item Nerve fibers are often observed to make nerve signals propagate at a speed of 100 m/s.  Estimate the reaction time of a person, if they touch something hot.  That is, the signal must travel from their finger touching a hot surface, to the spinal chord, and back to the finger to make it move. \\ \vspace{1.5cm}
\item A distance of 1 AU is the distance from the Earth to the Sun, and is equal to $\approx 1.5 \times 10^8$ km.  Jupiter is about 5 AU from the Sun.  How many kilometers between Jupiter and the Sun? \\ \vspace{1.5cm}
\item The speed of sound is measured to be 342 m/s on a certain day. What is this measurement in kilometers per hour? \\ \vspace{1.5cm}
\item A two \textit{liter} bottle of water has a volume of $2000$ cm$^3$.  What is this volume in m$^3$?  \textit{Should it be a large or small number?} \\ \vspace{1.5cm}
\end{enumerate}

\section{Chapter 2: }

\end{document}
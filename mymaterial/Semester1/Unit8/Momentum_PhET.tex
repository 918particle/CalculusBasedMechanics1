\documentclass{article}
\usepackage{graphicx}
\usepackage[margin=1.5cm]{geometry}
\usepackage{amsmath}
\usepackage{hyperref}

\begin{document}
\small

\title{PhET Activity: Smashing Things Together (Momentum)}
\author{Prof. Jordan C. Hanson}

\maketitle

\section{Introduction}

\textit{Momentum} is a vector quantity defined as the product of the mass of a system and the velocity of a system:
\begin{equation}
\vec{p} = m\vec{v}
\end{equation}
Looking at the change of both sides, assuming the mass is constant:
\begin{equation}
\frac{\Delta\vec{p}}{\Delta t} = m \frac{\Delta\vec{v}}{\Delta t} = m\vec{a}
\end{equation}
Note that the change in velocity is the acceleration, so Newton's 2nd Law has appeared:
\begin{equation}
\boxed{
\frac{\Delta\vec{p}}{\Delta t} = \vec{F}_{\rm Net}
}
\end{equation}

\section{Questions}

\begin{enumerate}
\item A system of mass 1 kg has a velocity with magnitude 10 m/s at a 45 degrees with respect to the x-axis.  Write the momentum vector. \\ \vspace{1cm}
\item Two systems each of mass 1 kg have velocities $\vec{v}_1 = 1\hat{i}$ m/s and $\vec{v}_2 = -1\hat{i}$ m/s, respectively. (a) What are the momenta vectors? (b) What is the total momentum $\vec{p}_1 + \vec{p}_2$? (c) In your own words, what will happen if these systems come in contact? \label{q:2} \\ \vspace{2cm}
\end{enumerate}

\section{PhET Simulation Tutorial: Collision Lab}

\begin{enumerate}
\item Navigate to the following PhET simulation: \url{https://phet.colorado.edu/en/simulations/collision-lab}.  Load the introduction tab to learn the controls.  The mass of each system can be tuned with the sliding knobs at the bottom center.  Click the values tab in the gray display at right.  This control displays the magnitudes of the velocity and the momentum.  Note that the units of momentum are kg m/s.  Clicking the ``More Data'' box in the lower left enables control over position and velocity.
\item Use the PhET simulation to create the scenario in Question \ref{q:2}. (a) Qualitatively, was your answer in part (c) correct? (b) What is the sum of the momentum vectors before and after the collision?
\end{enumerate}

\clearpage

\section{Exercises with the Collision Lab: Momentum Conservation}

Complete the following exercises to show that the total momentum (the sum the momentum vectors) cannot change.

\begin{enumerate}
\item Set $m_1 = 1$ kg, and $m_2 = 1$ kg.  Set $\vec{v}_1 = 1\hat{i}$ m/s, and $\vec{v}_2 = 0\hat{i}$ m/s.  (a) What are the initial momenta, $\vec{p}_{1,i}$ and $\vec{p}_{2,i}$, before the collision? (b) What is the total momentum $\vec{P}_{\rm tot}$ before the collision? (c) What will be the final momenta after the collision, $\vec{p}_{1,f}$ and $\vec{p}_{2,f}$, assuming $\vec{P}_{\rm tot}$ cannot change? (d) Run the PhET simulation and verify your predictions.  \\ \vspace{2cm}
\item Set $m_1 = 1$ kg, and $m_2 = 1$ kg.  Set $\vec{v}_1 = 1\hat{i}$ m/s, and $\vec{v}_2 = 0\hat{i}$ m/s.  Set the ``Elasticity'' parameter to 0\%.  This will cause the two systems to attach and become \textit{one system} during and after the collision.  (a) What are the initial momenta, $\vec{p}_{1,i}$ and $\vec{p}_{2,i}$, before the collision? (b) What is the total momentum $\vec{P}_{\rm tot}$ before the collision? (c) What will be the final momentum $\vec{P}_{\rm tot}$ after the collision? (d) What will be the final velocity of the system?  (e) Run the PhET simulation and verify your predictions.  \\ \vspace{2cm}
\item Set $m_1 = 1$ kg, and $m_2 = 1$ kg.  Set $\vec{v}_1 = 1\hat{i}$ m/s, and $\vec{v}_2 = -1\hat{i}$ m/s.  Set the ``Elasticity'' parameter to 0\%.  This will cause the two systems to attach and become \textit{one system} during and after the collision.  (a) What are the initial momenta, $\vec{p}_{1,i}$ and $\vec{p}_{2,i}$, before the collision? (b) What is the total momentum $\vec{P}_{\rm tot}$ before the collision? (c) What will be the final momentum $\vec{P}_{\rm tot}$ after the collision? (d) What will be the final velocity of the system?  (e) Run the PhET simulation and verify your predictions.  \\ \vspace{2cm}
\end{enumerate}

\section{Proof: Total Momentum Cannot Change}

We will now prove \textit{momentum conservation}, the idea that the sum of two momenta in the initial state is equal to the sum of two momenta in the final state.  Define system 1, with $m_1$ and $\vec{v}_1$.  Define system 1, with $m_2$ and $\vec{v}_2$.  Complete the proof by following the steps below.

\begin{itemize}
\item Begin with Newton's 3rd Law.  Write down, in vector form, the idea that during the collision the force imparted by system 1 on system 2 is equal and opposite to the force imparted by system 2 on system 1: \\ \vspace{0.5cm}
\item Use Newton's 2nd Law to trade $\vec{F}$ for $m\vec{a}$, as appropriate.  Maintain the subscripts that refer to systems 1 and 2. \\ \vspace{0.5cm}
\item Trade $\vec{a}$ for $\Delta\vec{v}/\Delta t$, the definition of acceleration, and move all terms to one side of the equation. \\ \vspace{0.5cm}
\item Convince yourself that
\begin{equation}
\frac{\Delta\vec{x}_1}{\Delta t} + \frac{\Delta\vec{x}_1}{\Delta t} = \frac{\Delta\left(\vec{x}_1 + \vec{x}_2\right)}{\Delta t} \label{eq:1}
\end{equation}
\item Use Eq. \ref{eq:1} to factor the previous step.  Notice that we now have shown that the change in the sum of vectors is zero.  What does this mean for that sum of vectors?  State your conclusion in your own words below:
\end{itemize}

\end{document}
\documentclass{article}
\usepackage{graphicx}
\usepackage[margin=1.5cm]{geometry}
\usepackage{amsmath}

\begin{document}

\title{Monday Reading Assessment: Unit 2}
\author{Prof. Jordan C. Hanson}

\maketitle

\section{Chapter 4}

\begin{enumerate}
\item Imagine a system propagating through 3D space with a velocity vector $\vec{v} = (2t-1)\hat{i} + 2\hat{j} + (-3t+2)\hat{k}$.  Is the object accelerating?  Why or why not? \\ \vspace{3cm}
\item Suppose a car is being pulled out of a ditch by a truck.  The car moves up the ditch at an angle of 45 degrees.  Which of the following is the likely acceleration vector?  Draw a diagram of the acceleration vector as a visual tool.
\begin{itemize}
\item A: $\vec{a} = (2\hat{i}+2\hat{j})$ m/s$^2$
\item B: $\vec{a} = (2\hat{i}+2\hat{j})$ m/s
\item C: $\vec{a} = (2\hat{i}-2\hat{j})$ m/s$^2$
\item D: $\vec{a} = (-2\hat{i}+2\hat{j})$ m/s$^2$
\end{itemize}
\item Imagine you are launching a projectile with a cannon that can tilt to any angle between 0 and 90 degrees.  What angle maximizes the \textit{range}?  Why? \\ \vspace{3cm}
\item If an object has a constant x-component of velocity and suddenly experiences an acceleration at an angle of 70 degrees with respect to the x-direction, does the x-component of velocity change?  Why or why not?
\end{enumerate}

\end{document}

\documentclass{article}
\usepackage{graphicx}
\usepackage[margin=1.5cm]{geometry}
\usepackage{amsmath}

\begin{document}

\title{Warm Up: Objects in Free-Fall, Projectiles}
\author{Prof. Jordan C. Hanson}

\maketitle

\section{Memory}

\begin{enumerate}
\item $y(t) = -\frac{1}{2}g t^2 + v_{i,y} t + y_i$ ... Vertical displacement.
\item $v_y(t) = -g t + v_{i,y}$ ... Vertical velocity.
\item $R = (v_i^2 \sin(2\theta))/g$ ... The range formula.
\end{enumerate}

\section{Objects in Free-Fall, Projectiles}

\begin{enumerate}
\item Use the \textit{second} equation in the Memory Bank to show that the total time spent in the air of a projectile launched from the origin is 
\begin{equation}
T = \frac{2 v_i \sin\theta}{g} \label{eq:1}
\end{equation}
\item Suppose we have a device that, when launching a marble straight upwards ($\theta = 90$ degrees) from the origin, it produces a measured flight time of $T$.  (a) What is $v_i$ in terms of the other variables? (b) Suppose that we now aim that device at $\theta = 45$ degrees.  Derive an algebraic expression for where it will land, that is, the range $R$. (c) If $T = 2$ seconds, and $g = 9.81$ m s$^{-2}$, what is $R$?\footnote{Hint: $T = 2v_i/g$, if we've done the derivation correctly.  Use this to find the \textbf{general} result for the range, then plug in numbers.}
\end{enumerate}

\end{document}

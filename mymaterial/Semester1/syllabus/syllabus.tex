\title{Syllabus for Calculus-Based Physics-1: Mechanics (PHYS150-02)}
\author{Dr. Jordan Hanson - Whittier College Dept. of Physics and Astronomy}
\date{\today}
\documentclass[10pt]{article}
\usepackage[a4paper, total={18cm, 27cm}]{geometry}
\usepackage{outlines}
\usepackage[sfdefault]{FiraSans}
\usepackage{hyperref}
\begin{document}
\maketitle

%Typically 10 problems per week, assigned on Monday and collected the following Monday. Please
%follow the link http: // goeta. link/ USB06CA-F1461C-1X2 to access the online homework system. The system re-
%quires $32.50 for access (remember that the textbook is free). The online system will give clues to struggling students
%and provide the professor with useful class statistics to aid in class management.


\begin{abstract}
The concepts of calculus-based electromagnetism will be presented within the context of interactive problem-solving.  The course will begin with the concepts of electric charge, electrostatics, electric potential and applications to DC circuits.  The course will proceed with the addition of magnetism, induction, and AC circuits.  The course will conclude with the electromagnetic spectrum and electromagnetic waves, time permitting.  The course work will include analytic textbook problems, interactive computational exercises, group-designed projects, and lab-based activities.
\end{abstract}
\noindent
\textit{\textbf{Pre-requisites}: Calculus I (MATH141) (may be concurrent).} \\
\textit{\textbf{Course credits, Liberal Arts Categorization}: 4 Credits, COM1} \\
\textit{\textbf{Regular course hours}: Monday, Wednesday and Friday from 15:00 - 16:30 in SLC 228} \\
\textit{\textbf{Instructor contact information}: jhanson2@whittier.edu, tel. 562.907.5130} \\
\textit{\textbf{Office hours}: Mondays, 16:30-17:30, and Tuesdays from 13:00-16:00 in SLC 212.} \\
\textit{\textbf{Attendance/Absence}: In-class activities will serve as attendance (see \textbf{Grading}). Students needing to reschedule
midterms and exams should notify the professor a reasonable time beforehand}.\\ 
\textit{\textbf{Late work policy}: Acceptance of late work is left to the discretion of the instructor.} \\
\textit{\textbf{Text}: University Physics Volume One - \url{https://openstax.org/details/books/university-physics-volume-1}} \\
\textit{\textbf{Grading}: There will be three tests, each examining conceptual understanding in step-by-step problems. Each test is
worth 15\% of the final grade. The weekly online homework is worth 20\% of the grade. Interactive in-class activities will be worth 10\% of the final grade. Lab groups will present results of a group project worth 10\% of the grade. The final exam will be held on December 10th from 13:00-15:00, and will be worth 15\% of the grade.} \\
\textit{\textbf{Grade Settings}: $<60\%$ = F, $\geq 60\%, <70\%$ = D, $\geq 70\%, <80\%$ = C, $\geq 80\%, <90\%$ = B, $\geq 90\%, <100\%$ = A.  Pluses and minuses: 0-3\% minus, 3\%-6\% straight, 6\%-10\% plus (e.g. 79\% = C+, 91\% = A-)} \\
\textit{\textbf{Homework Sets}: Typically 10 problems per week, assigned on Monday and collected the following Monday. Please follow the link \url{http://goeta.link/USB06CA-9A264D-1X1} to access the online homework system. The system requires \$32.50 for access (remember that the textbook is free). The online system will give clues to struggling students and provide the professor with useful class statistics to aid in class management.} \\
\textit{\textbf{Bonus Essay}: Students may submit an essay on the history of scientific developments covered in the course, due at
the end of the semester. The essay must be 10 pages, address scientific arguments and results, and must include
references. The grade of this paper will replace the lowest midterm grade, if it would raise the final course grade.} \\
\textit{\textbf{ADA Statement on Disability Services}: The Americans with Disabilities Act (ADA) is a federal anti-discrimination statute that provides comprehensive civil rights protection for persons with disabilities. Among other things, this legislation requires that all students with disabilities be guaranteed a learning environment that provides for reasonable accommodation of their disabilities. If you believe you have a disability requiring an accommodation, please contact Disability Services: disabilityservices@whittier.edu, tel. 562.907.4825.} \\
\textit{\textbf{Academic Honesty Policy}: \url{http://www.whittier.edu/academics/academichonesty}} \\
\textit{\textbf{Course Objectives}:}
\begin{itemize}
\item To practice written expression of quantitative and numerical ideas and arguments.
\item To practice expression of quantitative and numerical ideas and arguments.
\item Improvement of numerical analysis and problem solving.
\item Improvement of problem solving via computer simulations.
\item To practice the analysis of scientific data and results.
\end{itemize}
\textit{\textbf{Course Outline}:}
\begin{outline}[enumerate]
\1 Unit 0: Review of pre-requisite course, 150
\2 Estimation, approximation, kinematics and Newton's Laws
\2 Work, energy and power
\2 Momentum, linear and angular
\1 Unit 1: Electrostatics - \textbf{Chapters 5-6}
\2 The Coulomb force
\2 Electric and gravitational fields
\2 Gauss' Law and symmetries
\1 Unit 2: Electric potential and capacitance - \textbf{Chapters 7-8}
\2 Electric potential (voltage) and potential energy
\2 Capacitance: stored charge and energy
\2 Dielectric materials and batteries
\1 First midterm exam, end of Unit 2
\1 Unit 3: Current, resistance, and DC circuits - \textbf{Chapters 9-10}
\2 Current and resistivity, resistance
\2 Ohm's law
\2 Electromotive force
\2 Resistors in series and parallel, Kirchhoff's rules
\2 RC Circuits
\1 Unit 4: Magnetism 1 - \textbf{Chapters 11-12}
\2 Magnetism and magnetic field lines
\2 Charged particles in magnetic fields
\2 Current-carrying conductors in magnetic fields
\2 The Hall effect
\2 Applications
\2 The Biot-Savart Law
\2 Amp\`{e}re's Law
\1 Second midterm exam, end of Unit 4
\1 Unit 5: Field Induction - \textbf{Chapters 13-14}
\2 Faraday's Law and Lenz's Law
\2 Motional EMF and induced fields
\2 Electric generators
\2 Mutual inductance, self-inductance
\2 RL circuits, RLC circuits, transformers
\1 Unit 6: Electromagnetic waves - \textbf{Chapter 16}
\2 Maxwell's equations
\2 Electromagnetic waves and energy
\2 The electromagnetic spectrum
\1 Third midterm exam, end of Unit 6
\1 Unit 7 - \textbf{Cumulative Review, group presentations, and final exam}
\end{outline}
\end{document}

\title{Syllabus for Calculus-Based Physics-1: Mechanics (PHYS150-02)}
\author{Dr. Jordan Hanson - Whittier College Dept. of Physics and Astronomy}
\date{\today}
\documentclass[10pt]{article}
\usepackage[a4paper, total={18cm, 27cm}]{geometry}
\usepackage{outlines}
\usepackage[sfdefault]{FiraSans}
\usepackage{hyperref}
\begin{document}
\maketitle

\begin{abstract}
The concepts of calculus-based mechanics will be presented within the context of interactive problem-solving.
First, the concepts of displacement, velocity, and acceleration in one and two dimensions will be introduced, build-
ing up to Newton’s Laws of motion. Next, the concepts of friction and rotational motion will be added. More complex
problems will be introduced through the conservation of energy and linear momentum, followed by the rotational
equivalents. The course work will include interactive computational exercises, analytic textbook problems, and
lab-based activities.
\end{abstract}

\noindent
\textit{\textbf{Pre-requisites}: Calculus I (MATH141) (may be concurrent).} \\
\textit{\textbf{Course credits, Liberal Arts Categorization}: 4 Credits, COM1} \\
\textit{\textbf{Regular course hours}: Monday, Wednesday and Friday from 15:00 - 16:30 in SLC 228} \\
\textit{\textbf{Instructor contact information}: jhanson2@whittier.edu, tel. 562.907.5130} \\
\textit{\textbf{Office hours}: Mondays, 16:30-17:30, and Tuesdays from 13:00-16:00 in SLC 212.} \\
\textit{\textbf{Attendance/Absence}: In-class activities will serve as attendance (see \textbf{Grading}). Students needing to reschedule
midterms and exams should notify the professor a reasonable time beforehand}.\\ 
\textit{\textbf{Late work policy}: Acceptance of late work is left to the discretion of the instructor.} \\
\textit{\textbf{Text}: University Physics Volume One - \url{https://openstax.org/details/books/university-physics-volume-1}} \\
\textit{\textbf{Grading}: There will be three tests, each examining conceptual understanding in step-by-step problems. Each test is
worth 15\% of the final grade. The weekly online homework is worth 20\% of the grade. Interactive in-class activities will be worth 10\% of the final grade. Lab groups will present results of a group project worth 10\% of the grade. The final exam will be held on December 10th from 13:00-15:00, and will be worth 15\% of the grade.} \\
\textit{\textbf{Grade Settings}: $<60\%$ = F, $\geq 60\%, <70\%$ = D, $\geq 70\%, <80\%$ = C, $\geq 80\%, <90\%$ = B, $\geq 90\%, <100\%$ = A.  Pluses and minuses: 0-3\% minus, 3\%-6\% straight, 6\%-10\% plus (e.g. 79\% = C+, 91\% = A-)} \\
\textit{\textbf{Homework Sets}: Typically 10 problems per week, assigned on Monday and collected the following Monday. Please follow the link \url{http://goeta.link/USB06CA-9A264D-1X1} to access the online homework system. The system requires \$32.50 for access (remember that the textbook is free). The online system will give clues to struggling students and provide the professor with useful class statistics to aid in class management.} \\
\textit{\textbf{Bonus Essay}: Students may submit an essay on the history of scientific developments covered in the course, due at
the end of the semester. The essay must be 10 pages, address scientific arguments and results, and must include
references. The grade of this paper will replace the lowest midterm grade, if it would raise the final course grade.} \\
\textit{\textbf{ADA Statement on Disability Services}: The Americans with Disabilities Act (ADA) is a federal anti-discrimination statute that provides comprehensive civil rights protection for persons with disabilities. Among other things, this legislation requires that all students with disabilities be guaranteed a learning environment that provides for reasonable accommodation of their disabilities. If you believe you have a disability requiring an accommodation, please contact Disability Services: disabilityservices@whittier.edu, tel. 562.907.4825.} \\
\textit{\textbf{Academic Honesty Policy}: \url{http://www.whittier.edu/academics/academichonesty}} \\
\textit{\textbf{Course Objectives}:}
\begin{itemize}
\item To practice written expression of quantitative and numerical ideas and arguments.
\item To practice expression of quantitative and numerical ideas and arguments.
\item Improvement of numerical analysis and problem solving.
\item Improvement of problem solving via computer simulations.
\item To practice the analysis of scientific data and results.
\\
\end{itemize}
\textit{\textbf{Course Outline}:}
\begin{outline}[enumerate]
\1 Unit 0 - \textbf{Chapters 1.1 - 1.7, 2.1 - 2.4}
\2 Introduction to iClicker, class procedures, reading syllabus.
\2 \textit{Warm-up activity: gas mileage, speed, calories and energy.}
\2 Monday reading quiz: chapter 2.2
\2 Wednesday reading quiz: chapter 2.3
\2 Friday: article bonus
\2 Topics covered: 
\3 Unit-analysis, approximation, and coordinate systems.
\3 Adding and subtracting vectors, displacement and translational motion.
\3 Vectors and scalars.
\3 Adding and subtracting vectors.
\3 Multiplying vectors.
\1 Unit 1 - \textbf{Chapters 3.1 - 3.6}
\2 Monday reading quiz: chapter 3.1
\2 Wednesday reading quiz: chapter 3.2
\2 Friday: article bonus
\2 Topics covered:
\3 Displacement, velocity, and acceleration vectors.
\3 Differentiation and integration of functions of the form $f(x) = a x^n + b$.
\3 Motion with constant acceleration in 1D.
\1 Unit 2 - \textbf{Chapters 4.1 - 4.3}
\2 Monday reading quiz: chapter 4.1
\2 Wednesday reading quiz: chapter 4.2
\2 Friday: article bonus
\2 Topics covered:
\3 Displacement, velocity, and acceleration in 2D.
\3 Projectile motion.
\1 \textbf{First midterm: October 2nd, 2019 during normal class time.}  This exam is worth 15\% of the class grade, and will cover Units 0-2. The emphasis is on working with units, vectors, and applying kinematic equations in 1D.
\1 Unit 3 - \textbf{Chapters 4.4, 5.1 - 5.4}
\2 Monday reading quiz: chapter 4.4
\2 Wednesday reading quiz: chapter 5.3
\2 Friday: article bonus
\2 Topics covered:
\3 Uniform rotational motion.
\3 Newton's First Law.
\3 Newton's Second Law.
\1 Unit 4 - \textbf{Chapters 5.5 - 5.7}
\2 Monday reading quiz: chapter 5.5
\2 Wednesday reading quiz: chapter 5.7
\2 Friday: article bonus
\2 Topics covered:
\3 Newton's Third Law.
\3 Common forces: normal and tension forces.
\3 Incline planes and free-body diagrams.
\1 Unit 5 - \textbf{Chapters 6.1 - 6.4}
\2 Monday reading quiz: chapter 6.2
\2 Wednesday reading quiz: chapter 6.3
\2 Friday: article bonus
\2 Topics covered:
\3 Frictional forces.
\3 Centripetal force.
\3 Drag forces.
\1 Unit 6 - \textbf{Chapters 7.1 - 7.4}
\2 Monday reading quiz: chapter 7.1
\2 Wednesday reading quiz: chapter 7.3
\2 Friday: article bonus
\2 Topics covered:
\3 Work and energy.
\3 Work-Energy theorem.
\3 Power.
\1 \textbf{Second midterm: November 4th, 2019 during normal class time.} This exam is worth 15\% of the class grade, and will cover Units 3-6. The emphasis will be on Newton's Second Law, free-body diagrams and friction, and the definition of work.
\1 Unit 7 - \textbf{Chapters 8.1 - 8.5}
\2 Monday reading quiz: chapter 8.1
\2 Wednesday reading quiz: chapter 8.3
\2 Topics covered:
\3 Potential energy.
\3 Conservative forces.
\3 Conservation of energy.
\1 Unit 8 - \textbf{Chapters 9.1, 9.3, and 9.4.} \textit{Time-permitting, also 9.5.}
\2 Monday reading quiz: chapter 9.1
\2 Wednesday reading quiz: chapter 9.3
\2 Friday: article bonus
\2 Topics covered:
\3 Linear momentum
\3 Collisions: elastic and inelastic
\1 Unit 9 - \textbf{Chapters 10.1 - 10.4, 10.6}
\2 Monday reading quiz: chapter 10.1
\2 Wednesday reading quiz: chapter 10.4
\2 Friday: article bonus
\2 Topics covered:
\3 Kinematics of rotating systems.
\3 Rotational kinetic energy.
\3 Torque.
\1 Unit 10 - \textbf{Chapters 11.2 - 11.3, 13.1-13.5}
\2 Monday reading quiz: chapter 11.2
\2 Wednesday reading quiz: chapter 13.1
\2 Topics covered:
\3 Angular momentum and conservation of angular momentum.
\3 Newton's Law of Gravity, \textit{Time permitting.}
\1 \textbf{Third midterm: November 25th, 2019 during normal class time.}  This exam is worth 15\% of the class grade, and will cover Units 7-10. The emphasis is on potential energy, inelastic collisions involving linear momentum, torque and moments of inertia, and angular momentum conservation.
\1 Unit 11 - \textbf{Class presentations and Final Review}
\2 No reading quizzes
\2 Group presentations:
\3 Worth 10\% of the course grade.
\3 Given as a group.
\3 10-15 minute presentation with slides or board work.
\3 Final exam reviews will be given between Dec. 6th - 9th, for the convenience of the students.
\end{outline}
\end{document}
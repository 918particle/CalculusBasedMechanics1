\documentclass{article}
\usepackage{graphicx}
\usepackage[margin=1.5cm]{geometry}
\usepackage{amsmath}

\begin{document}

\title{Work and Energy Lab with Pulleys: }
\author{Prof. Jordan C. Hanson}

\maketitle

\section{Introduction}

Recall that the friction force is given by

\begin{equation}
f = \mu N
\end{equation}

Recall also that we found the frictional coefficient to be $\mu = m_2/m_1$ in a previous lab, where $m_2$ was a mass dangled from a string over a pulley, and $m_1$ was dragged across the table.  Repeat this measurement with the set of weights to verify that $\mu \approx 0.2$.

\section{Setup}

You will need the following objects:

\begin{itemize}
\item A pulley, and a clamp to attach it to the table.
\item A set of metal weights and a wooden block.
\item A ruler and a timer (can be the stopwatch on the phone).
\end{itemize}

\section{Measurements}

Arrange a pulley at the edge of the table using the clamp.  The block shoud have a string attached to allow weights to be hung from it.  Place the block on the table, and feed the string through the pulley with the hook and weights on the other side.  \textbf{Measure} $\mu$ and record it below.  Using a free body diagram, show that the acceleration of the block on the table is

\begin{equation}
a = g(\mu - m_2 / m_1)
\end{equation}

Knowing the acceleration means you can know the final velocity if you measure the time the top block acceleration: $v_f = a t$.  From a fixed distance $\Delta x$ away from the pulley, arrange weights such that the block accelerates to the pulley.  \textbf{Compute the kinetic energy gained by the top block} using $v_f = a t$, and $KE_f = \frac{1}{2} m_1 v_f^2$. \\ \vspace{3cm}

\section{Gravitational Potential Energy}

Compare the value $m_2 g \Delta y$, the change in \textit{gravitational potential energy} of the lower block to the energy gained by the top block.  If the string is taught the entire time, it should be that $\Delta x = \Delta y$, but it's good to check this.  How close are the two energy results?
\end{document}

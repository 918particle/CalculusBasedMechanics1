\documentclass{article}
\usepackage{graphicx}
\usepackage[margin=1.5cm]{geometry}
\usepackage{amsmath}

\begin{document}

\title{Monday Reading Assessment: Unit 1 and kinematics}
\author{Prof. Jordan C. Hanson}

\maketitle

\section{Chapter 3.1}

\begin{enumerate}
\item In your own words: does a car odometer measure distance traveled or displacement?  Why or why not?  \\ \vspace{2cm}
\item Suppose a system has a velocity vector $\vec{v}(t) = a t \hat{i} + b \hat{j}$, where $t$ is the time.  What would be the correct formula for $\vec{v}(t)$ if the system simply doubled in velocity, regardless of the time? \\ \vspace{2cm}
\item Suppose a system has a velocity vector $\vec{v}(t) = a t \hat{i} + b \hat{j}$, where $t$ is the time.  What would be the correct formula for $\vec{v}(t)$ if the system reversed the direction of its velocity, regardless of the time? \\ \vspace{2cm}
\item Suppose a system has a speed $v(t) = a t + b$, where $t$ is the time.  What are the units of the constant $a$?  If $b=4$ m/s, and $v(4) = 8$ m/s, what is the value of $a$? \\ \vspace{2cm}
\item If you go between two toll-road checkpoints and your average velocity is found to be higher than the speed limit, is it possible that your \textit{instantaneous} speed was always below the speed limit? \\ \vspace{3cm}
\end{enumerate}

\end{document}

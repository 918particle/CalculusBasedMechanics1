\title{Midterm 3 for Calculus-Based Physics-1: Mechanics (PHYS150-01)}
\author{Dr. Jordan Hanson - Whittier College Dept. of Physics and Astronomy}
\date{November 6th, 2017}
\documentclass[10pt]{article}
\usepackage[a4paper, total={18cm, 27cm}]{geometry}
\usepackage{outlines}
\usepackage[sfdefault]{FiraSans}
\usepackage{graphicx}

\begin{document}
\maketitle

\section{Definition of Work}
\begin{enumerate}
\item \textit{In each of the following three questions, determine whether work is being performed \textbf{on the briefcase} by the man}.
\begin{itemize}
\item A man is walking horizontally, carrying a briefcase at a constant height. (a) No work is done on briefcase (b) Positive work is done on briefcase (c) Negative work is done on briefcase
\item A man stands still, and holding a briefcase in a fixed position.  (a) No work is done on briefcase (b) Positive work is done on briefcase (c) Negative work is done on briefcase
\item A man stands still, and lowers the briefcase to a height less than the original height.  (a) No work is done on briefcase (b) Positive work is done on briefcase (c) Negative work is done on briefcase
\item A man raises the briefcase to a height greater than the original height.  (a) No work is done on briefcase (b) Positive work is done on briefcase (c) Negative work is done on briefcase
\end{itemize}
\item For this problem, use the work formula $W = \vec{F}_{\rm Net} \cdot \vec{x}$.  (a) Draw the correct free-body diagram for a crate being pushed horizontally against friction by some applied force $\vec{f}$.  (b) Calculate the work done by $\vec{F}_{\rm Net}$ on a 200 kg palette crate through a distance $\vec{x} = 10$ m on a surface with coefficient of kinetic friction of 0.05, if $\vec{f} = 150\hat{i}$ N (in the horizontal direction).  \\ \vspace{2cm}
\item A crane lifts a shipping container from a ship to the dock.  The shipping container is at a height of 30 meters above the water initially, and ends 4 meters above the water after the move is complete.  The shipping container has a mass of 4500 kg.  (a) Draw the correct free-body diagram.  (b) What is the work done on the shipping container, in kJ?  Should it be positive or negative? (\textit{Recall that the work done lifting an object a height $h$ against gravity is $W = mgh$}).\\ \vspace{2cm}
\end{enumerate}
\section{Kinetic Energy}
\begin{enumerate}
\item In a particular radioactive isotope, \textit{thermal neutrons} are emitted with kinetic energy $KE = \frac{1}{2}mv^2$.  If the mass of a neutron is $1.7\times 10^{-27}$ kg, and the velocity is $v = 3\times 10^6$ m/s, (a) what is the kinetic energy? (b) If 1 electron-Volt equals $1.6 \times 10^{-19}$ Joules, how many electron-Volts of energy do these neutrons have? \\ \vspace{2cm}
\item Not every particle in the radiation is a neutron.  Suppose one is an \textit{alpha-particle}, which has four times the mass of a neutron.  If we detect an alpha particle that has the same speed as a neutron, and the neutron has kinetic energy of 5 MeV ($5 \times 10^{6}$ electron-Volts), what is the kinetic energy of the alpha particle? \\ \vspace{1.75cm}
\end{enumerate}
\section{Work-Energy Theorem}
\begin{enumerate}
\item An archer's bow acts like a spring with a spring constant of $k=350$ N/m.  If an arrow is loaded into this bow, how much energy is required to draw it back 0.5 m?
\begin{itemize}
\item 4 J
\item 40 J
\item 400 J
\item 4000 J
\end{itemize}
\item Assume all of the energy in the drawn bow in the previous question is released into the kinetic energy of the arrow (mass = 0.02 kg).  If the arrow is fired from a height of 1.5 m, how far does it travel before it lands?
\begin{itemize}
\item 3 m
\item 30 m
\item 300 m
\item 3000 m
\end{itemize}
\end{enumerate}
\section{Gravitational Potential Energy}
\begin{enumerate}
\item The height of Freedom Tower in New York City is 1776 feet, or or 541 meters.  Suppose we drop a penny off of the top!  (a) What is the \textit{potential energy} in J before we drop it, if it has a mass of 2.5 grams?
\begin{itemize}
\item 1.35 J
\item 13.5 J
\item 27 J
\item 54 J
\end{itemize}
(b) What velocity does the penny have when it hits the street, assuming no drag force?
\begin{itemize}
\item 1 m/s
\item 10 m/s
\item 100 m/s
\item 300 m/s
\end{itemize}
\end{enumerate}
\section{Conservative Forces}
\begin{enumerate}
\item (a) Which of these forces is \textit{conservative}?
\begin{itemize}
\item The spring force (Hooke's Law)
\item Kinetic friction
\item Drag due to air
\item Stoke's Law (drag in viscous substances)
\end{itemize}
(b) In your own words, define a \textit{conservative} force.
\end{enumerate}
\end{document}
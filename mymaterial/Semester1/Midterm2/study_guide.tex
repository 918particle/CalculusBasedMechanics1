\title{Study Guide for Midterm 2}
\author{Dr. Jordan Hanson - Whittier College Dept. of Physics and Astronomy}
\date{\today}
\documentclass[10pt]{article}
\usepackage[margin=1.5cm]{geometry}
\usepackage{outlines}
\usepackage{graphicx}
\usepackage{amsmath}

\begin{document}
\maketitle

\section{Memory Bank}

\begin{itemize}
\item Unit conversions: 1 km = 1000 m, 1 m = 100 cm, 1 hr = 3600 s, 1 year = $\pi \times 10^7$ s, 1 g/cm$^3$ = 1000 kg/m$^3$.
\item $\vec{x} = a \hat{i} + b\hat{j}$ ... Component form of a two-dimensional vector.
\item $|\vec{x}| = \sqrt{a^2+b^2}$ ... Pythagorean theorem for obtaining vector magnitude.
\item $\theta = \tan^{-1}(b/a)$ ... Obtaining the angle between vector and x-axis.
\item $a = |\vec{x}|\cos(\theta)$ ... Obtaining the x-component with trigonometry.
\item $b = |\vec{x}|\sin(\theta)$ ... Obtaining the y-component with trigonometry.
\item $x(t) = x_i + v t$ ... Velocity is the slope of position versus time.
\item $x(t) = \frac{1}{2} a t^2 + v_i t + x_i$ ... With constant acceleration, position is quadratic.  If $a=0$ this becomes the prior function.
\item $v(t) = v_i + a t$ ... With constant acceleration, acceleration is the slope of velocity.
\item $v^2 = v_i^2 + 2 a \Delta x$ ... The kinematic equation without time, assuming constant acceleration.
\item $\vec{F}_{net} = 0$ ... Newton's First Law, an object with no net force stays at constant velocity, or zero velocity.
\item $\vec{F}_{net} = m\vec{a}$ ... Newton's Second Law.
\item $\vec{F}_{AB} = -\vec{F}_{BA}$ ... Newton's Third Law.
\item $\vec{w} = - mg \hat{j}$ ... Weight force.
\item $\vec{N} = +mg\hat{j}$ ... Normal force, when the object is on a flat surface.
\item $N = mg\cos\theta$, $w_x = -mg\sin\theta$, $w_y = -mg\cos\theta$ ... Incline plane forces.
\item $f = \mu N$, $F_D = \frac{1}{2}C\rho A v^2$, $F_D = 6\pi r \eta v$ ... friction, drag in air, drag in viscous fluids.
\item $stress = Y \times strain$, or $F/A = Y (\Delta x / L)$ ... Young's Modulus and elasticity.
\item $s = r \theta$ ... Definition of a \textit{radian}, with arc length $s$ and angle $\theta$.
\item $v = r\omega$, $a = r\alpha$ ... Angular velocity, angular acceleration.
\item $a_C = v^2/r = r\omega^2$ ... Centripetal acceleration.
\item $F_C = m a_C = mv^2/r = mr\omega^2$ ... Centripetal force.
\item $\vec{F}_G = G m_1 m_2/r^2 ~~ \hat{r}$ ... Newton's Law of Gravity.
\end{itemize}

\section{Chapter 4: Dynamics, Force and Newton's Laws of Motion}

\begin{enumerate}
\item A motorcycle can produce an acceleration of 3.0 m/s$^2$ while traveling at 90.0 km/h. At that speed the forces resisting motion, including friction and air resistance, total 400 N.  What is the magnitude of the force the motorcycle exerts backward on the ground to produce its acceleration if the mass of the motorcycle with rider is 220 kg? \\ \vspace{2.5cm}
\item A rugby player is being pushed backward by an opponent who is exerting a force of 800 N on him.  The mass of the losing player plus equipment is 90.0 kg, and he is accelerating at -1.20 m/s$^2$.  (a) What is the force of friction between the losing player’s feet and the grass? (b) What force does the winning player exert on the ground to move forward if his mass plus equipment is 110 kg? \\ \vspace{2.5cm}
\item Two muscles in the back of the leg pull upward on the Achilles tendon, as shown in Fig. \ref{fig:muscle}. (These muscles are called the medial and lateral heads of the gastrocnemius muscle.) Find the magnitude and direction of the total force on the Achilles tendon.  What type of movement could be caused by this force? \\ \vspace{2.5cm}
\begin{figure}[hb]
\centering
\includegraphics[width=0.2\textwidth]{figures/muscle.png}
\caption{\label{fig:muscle} The achilles tendon and the two muscles that connect to it.}
\end{figure}
\end{enumerate}

\clearpage

\section{Chapter 5: Further Applications of Newton's Laws, Friction, Drag, and Elasticity}

\begin{enumerate}
\item A team of eight dogs pulls a sled with waxed wood runners on wet snow ($\mu_k = 0.1$). Each dog is 19.0 kg, and the loaded sled with its rider has a mass of 210 kg.  (a) Calculate the magnitude of the acceleration starting from rest if each dog exerts an average force of 185 N backward on the snow. (b) What is the magnitude of the acceleration once the sled starts to move? \\ \vspace{2.0cm}
\item Calculate the maximum deceleration of a car that is heading down a $6^{\circ}$ slope under the following road conditions. You may assume that the weight of the car is evenly distributed on all four tires and that the coefficient of static friction is involved (no slipping).  Calculate for a car: (a) On dry concrete ($\mu_s = 1.0$). (b) On wet concrete ($\mu_s = 0.7$). \\ \vspace{2.5cm}
\item \textbf{Stokes’ Law}.  Suppose a steel ball bearing (density $7.8\times 10^3$ kg/m$^3$, diameter 3.0 mm) is dropped in a container of motor oil. It takes 12 seconds to fall a distance of 0.60 m. Calculate the viscosity of the oil. \\ \vspace{2cm}
\item (a) By how much does a 65.0-kg mountain climber stretch her 0.800-cm diameter nylon rope when she hangs 35.0 m below a rock outcropping? The Young's Modulus is $5 \times 10^9$ N m$^{-2}$. Is the answer reasonable? \\ \vspace{1cm}
\end{enumerate}

\section{Chapter 6: Uniform Circular Motion and Gravitation}

\begin{enumerate}
\item In lacrosse, a ball is thrown from a net on the end of a stick by rotating the stick and forearm about the elbow. If the angular velocity of the ball about the elbow joint is 30.0 radians/second and the ball is 1.30 m from the elbow joint, what is the velocity
of the ball? \\ \vspace{1cm}
\item An ordinary workshop grindstone has a radius of 7.50 cm and rotates at 6500 rev/min.  Calculate the magnitude of the centripetal acceleration at its edge in m/s$^2$ and convert to g's. \\ \vspace{1cm}
\item What is the ideal banking angle for a gentle turn of 1.20 km radius on a highway with a 105 km/h speed limit, assuming everyone travels at the limit?
\end{enumerate}

\end{document}
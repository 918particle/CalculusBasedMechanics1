\title{Practice Problems for 2nd Midterm for Calculus-Based Physics-1: Mechanics (PHYS150-01)}
\author{Dr. Jordan Hanson - Whittier College Dept. of Physics and Astronomy}
\date{October 16th, 2017}
\documentclass[10pt]{article}
\usepackage[a4paper, total={18cm, 27cm}]{geometry}
\usepackage{outlines}
\usepackage[sfdefault]{FiraSans}
\usepackage{graphicx}

\begin{document}
\maketitle

\section{Vectors and Newton's Laws}
For each of the exercises below, $\vec{a} = 3\hat{i} + 4\hat{j}$, and $\vec{b} = 6\hat{i}+8\hat{j}$.
\begin{enumerate}
\item Calculate the magnitude of $\vec{a}$: $|\vec{a}| = \sqrt{3^2+4^2} = 5$.
\item Calculate the magnitude of $\vec{b}$: $|\vec{b}| = \sqrt{4^2+8^2} = 10$.
\item Calculate the dot product $\vec{a} \cdot \vec{b}$: $3*6 + 4*8 = 50$ (Notice the dot product gives a number).
\item Using $\vec{a}\cdot\vec{b} = |\vec{a}||\vec{b}|\cos\theta$, get the angle $\theta$ between the vectors: $\vec{a}\cdot\vec{b}/(|\vec{a}||\vec{b}|) = \cos\theta = 50/(5*10) = 1$.  So $\cos\theta = 1$, therefore $\theta = 0$.
\end{enumerate}
\section{Newton's Laws, and Circular Motion}
\begin{enumerate}
\item The centripetal acceleration is $a_{\rm C} = v^2/r = r\omega^2$, and the centripetal force is $F_{\rm C} = mv^2/r = mr\omega^2$.  Show that if a a person is swinging a rock attached to a line in a circle over their head (think of a sling, or bolas), that the rock must have speed $v = \sqrt{rg}$ in order stay aloft, if the circular trajectory of the rock is perpindicular to the ground: \\ \\
The gravitational force on the stone pulls down, creating tension in the rope, which is eventually sideways pulling the stone with centripetal force.  Thus, $T = mg = mv^2/r$. Solving for $v$, we have $v^2 = rg$ or $v = \sqrt{rg}$.  
\end{enumerate}
\section{Frictional Forces}
\begin{enumerate}
\item We did a lab to measure $\mu_{\rm k}$, the coefficient of static friction.  Show that the free-body diagram yields the following equation $\mu_{\rm k} = \frac{m_{\rm p}}{m_{\rm B}} < 1$: \\ \\
The pulley transmits the gravitational force (which points down) into tension which pulls the block sideways.  The tension is therefore $m_{\rm P} g$, where $m_{\rm P}$ is the mass on the pulley.  The frictional force is $\mu_{\rm B}N$, where $N$ is the normal force.  The normal force is $m_{\rm B}g$, where $m_{\rm B}$ is the mass of the block.  If these forces are in balance, we have $\mu_{\rm B}m_{\rm B}g = m_{\rm P} g$ so $\mu = \frac{m_{\rm P}}{m_{\rm B}}$.  The mass on the pulley was always smaller because the block mass included the wood and weights added to it.
\end{enumerate}
\end{document}
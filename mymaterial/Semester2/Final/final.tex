\title{Final for Calculus-Based Physics: Electricity and Magnetism (PHYS180)}
\author{Dr. Jordan Hanson - Whittier College Dept. of Physics and Astronomy}
\date{\today}
\documentclass[10pt]{article}
\usepackage[a4paper, total={18cm, 27cm}]{geometry}
\usepackage{outlines}
\usepackage{graphicx}
\usepackage{amsmath}
\begin{document}
\maketitle

\section{Equations and constants}

\begin{enumerate}
\item Coulomb force: $\vec{F}_C = k \frac{q_1 q_2}{r^2}\hat{r}$.
\item Centripetal force: $\vec{F} = \frac{mv^2}{r}$
\item Definition of electric field: $\vec{F}_C = q\vec{E}$.
\item Voltage and electric field, one dimension, uniform field: $|E| = - \frac{\Delta V}{\Delta x}$.
\item Charge and capacitance: $Q = CV$.
\item Definition of current: $I = \Delta Q / \Delta t$.
\item Parallel plate capacitor: $C = \frac{\epsilon_0 A}{d}$.
\item Ohm's Law: $V = IR$.
\item Adding resistors \textit{in series}: $R_{tot} = R_1 + R_2$ \textit{in parallel}: $R_{tot}^{-1} = R_1^{-1} + R_2^{-1}$.
\item Adding capacitors \textit{in parallel}: $C_{tot} = C_1 + C_2$ \textit{in series}: $C_{tot}^{-1} = C_1^{-1} + C_2^{-1}$.
\item Electrical power: $P = IV = I^2 R = V^2/R$.
\item Magnetic dipole moment: $\vec{\mu} = I \vec{A}$, where $\vec{A}$ is the area vector. $\mu = N I A$ if there are $N$ loops.
\item Torque on a magnetic dipole: $\tau = \vec{\mu} \times \vec{B}$.  The magnitude is $\tau = \mu B \sin(\theta)$.
\item Hall voltage: $emf = B l v$.
\item Definition of magnetic flux: $\phi_m = \vec{B} \cdot \vec{A}$.  The units are T m$^2$, which is called a Weber, or Wb.
\item Faraday's Law: $emf = -N \frac{\Delta \phi}{\Delta t}$.
\item Faraday's Law using \textbf{Inductance}, M: $emf = -M \frac{\Delta I}{\Delta t}$.
\item Typically, we refer to \textit{mutual inductance} between two objects as $M$, and \textit{self inductance} as $L$.
\item Magnetic permeability: $\mu_0 = 4\pi \times 10^{-7}$ T m A$^{-1}$
\item Units of inductance: V s A$^{-1}$, which is called a Henry, or H.
\item Coulomb constant: $k = 8.9876 \times 10^{9}$ N m$^2$ C$^{-2}$.
\item Fundamental charge: $q_e = 1.602 \times 10^{-19}$ C.
\item Speed of light: $\approx 3 \times 10^{8}$ m/s.
\item Permittivity of free space: $\epsilon_0 = 8.85 \times 10^{-12}$ N$^{-1}$ C$^2$ m$^{-2}$.
\end{enumerate}

\clearpage

\begin{enumerate}
\item Consider Fig. \ref{fig:ring} below.  A ring of charge with radius $R$ is situated in the xy-plane.  The charge is positive, and it is distributed evenly across the ring.  We write $\Delta q = \lambda R \Delta\theta$, to mean that there is $\lambda$ Coulombs per unit length.  If $\Delta\theta$ were to extend to $2\pi$ (all the way around the circle), then the total charge is $Q = \lambda (2 \pi R)$.  (a) By symmetry, where should the electric field be zero?
\begin{figure}[ht]
\centering
\includegraphics[width=0.22\textwidth]{ring.png}
\caption{\label{fig:ring} A ring of charge situated in the xy-plane.}
\end{figure}
\item As $z \rightarrow \infty$ in Fig. \ref{fig:ring}, what happens to the field?
\begin{itemize}
\item A: The field-strength increases.
\item B: The field-strength remains constant.
\item C: The field-strength decreases.
\item D: The field-strength is exactly zero.
\end{itemize}
\item Suppose the actual function for the E-field $\vec{E}(z)$ is
\begin{equation}
\vec{E}(z) = \frac{1}{4\pi \epsilon_0} \frac{q z}{\left( z^2 + R^2 \right)^{3/2}} \hat{z} \label{eq:eq1}
\end{equation}
To see what happens when $z$ is much larger than $R$, try setting $R=0$.  What is the result in Eq. \ref{eq:eq1} if $R=0$? \\ \vspace{1cm}
\item To what charge distribution does this expression correspond (the limit that $R \rightarrow 0$)? \\ \vspace{1cm}
\item (a) What is the final kinetic energy of a proton accelerated through 1 kV? (b) Suppose protons are placed into a linear accelerator with 100 voltages that each provide 10 kV potential.  What is the final kinetic energy in eV?  (c) What is the final speed of the proton? \\ \vspace{1.5cm}
\item Suppose two parallel plate capacitors are added in parallel.  One has an area of 1.0 mm$^2$, and a plate separation of 0.1 mm, and the other has area 0.5 mm$^2$ and separation 0.2 mm.  What is the total capacitance of the system? \\ \vspace{1cm}
\clearpage
\item A DC winch motor is rated at 20.00 A with a voltage of 115 V. When the motor is running at its maximum power, it can lift an object with a weight of 4900.00 N a distance of 10.00 m, in 30.00 s, at a constant speed. (a) What is the power consumed by the motor? (b) What is the power used in lifting the object? Ignore air resistance. (c) Assuming that the difference in the power consumed by the motor and the power used lifting the object are dissipated as heat by the resistance of the motor, estimate the resistance of the motor? \\ \vspace{2.5cm}
\item Suppose a a battery is connceted in series with a resistor.  The $\epsilon$, or emf of the battery is 14 V and the resistance is 50$\Omega$.  The current is measured to be 266 mA.  What is the \textit{internal resistance} of the battery? \\ \vspace{1.5cm}
\item Consider Fig. \ref{fig:lorentz}, in which a DC power generator is depicted inside a 0.05 T B-field.  Suppose the area of the loop is $10^{-2}$ m$^{2}$, the voltage in the circuit is 24 V, and the circuit resistance is 50 $\Omega$.  Also assume that there is just one loop of wire in the rotor.  What is the \textit{maximum torque} the system could achieve? \\ \vspace{2cm}
\begin{figure}[ht]
\centering
\includegraphics[width=0.55\textwidth]{commute.jpeg}
\caption{\label{fig:lorentz} An illustration of how a power generator works.  This version uses DC current and a commutator.}
\end{figure}
\item What would the maximum torque be if there were $N = 100$ turns of wire? \\ \vspace{1cm}
\item Consider Fig. \ref{fig:acgen}.  Suppose that the angle between the area vector and the magnetic field is $\theta = \omega t$.  (a) Show that
\begin{equation}
\phi(t) = BA\cos(\omega t) \label{eq:ac}
\end{equation}
(b) Given Eq. \ref{eq:ac}, it turns out that the voltage generated in the loop is proportional to $\sin(\omega t)$ and $\omega$ itself.  That is,
\begin{equation}
\epsilon(t) = BA\omega \sin(\omega t)
\end{equation}
What is the voltage at a time $t = 1/240$ seconds, $\omega = 120\pi$ Hz, $B = 0.1$ T, and $A = 0.01$ m$^2$? (c) At what time is the voltage zero? \\ \vspace{3cm}
\begin{figure}[hb]
\centering
\includegraphics[width=0.35\textwidth]{acGen.jpeg}
\caption{\label{fig:acgen} A schematic of the concept of an AC generator.}
\end{figure}
\item Suppose the AC generator in Fig. \ref{eq:ac} has $V_0 = 12$ V so that $\epsilon(t) = V_0 \sin(\omega t)$.  If the AC generator pushes current through a resistance $R = 50\Omega$, what is the average power generated?
\end{enumerate}

\end{document}
\documentclass{beamer}
\usetheme{metropolis}
\usepackage{graphicx}
\usepackage{subfig}
\usepackage{tcolorbox}
\title{Calculus-Based Physics-2: Electricity, Magnetism, and Thermodynamics (PHYS180-02): Unit 1}
\author{Jordan Hanson}
\institute{Whittier College Department of Physics and Astronomy}

\begin{document}
\maketitle

\begin{frame}
\centering
\includegraphics[width=0.95\textwidth]{figures/hyper.pdf}
\end{frame}

\section{Summary}

\begin{frame}{Unit 1 Summary}
\textbf{Reading: Chapters 5-6}
\begin{enumerate}
\item Charge, conductors and insulators
\item Coulomb's law and electric Fields
\item Electric fields of charge distributions
\item Gauss's Law
\end{enumerate}
\end{frame}

\section{Charge, Conductors and Insulators}

\begin{frame}{Charge, Conductors and Insulators}
\small
Charge the following intrinsic properties: \\ \vspace{0.25cm}
\begin{enumerate}
\item Charge is conserved globally (charge cannot be created nor destroyed).  Mass has the same property.
\item Charge is conserved locally (if we pull charge out of the system, charge will flow into the system).
\item Charge is quantized, with an electron (for example) having the fundamental negative unit, and a proton (for example) having the fundamental positive unit.
\item The laws of physics are the same for positive and negative charges.
\item The two kinds of charge emit fields that attract each other; fields emitted by charges of the same type repel such charges.
\end{enumerate}
\end{frame}

\begin{frame}{Charge, Conductors and Insulators}
\textbf{Benjamin Franklin and the Leyden Jar}.  (Good paper topic).
\begin{figure}
\centering
\includegraphics[width=0.3\textwidth]{figures/leyden.png}
\caption{\label{fig:leyden} A Leyden jar was an early version of a capacitor.  Benjamin Franklin guessed that one type of charge moves and another remains stationary, explaining several behaviors of charged objects.}
\end{figure}
\end{frame}

\begin{frame}{Charge, Conductors and Insulators}
The rest of the properties of charge are connected to the development of the structure of the atom, and we will return to this topic at the end of the semeter.
\begin{figure}
\centering
\includegraphics[width=0.5\textwidth]{figures/atom.png}
\caption{\label{fig:atom} A sketch of our current atomic paradigm.}
\end{figure}
\end{frame}

\begin{frame}{Charge, Conductors and Insulators}
Suppose an ion is composed of six protons, eight neutrons, and five electrons.  What is the net charge?
\begin{itemize}
\item A: +1
\item B: 0
\item C: -1
\item D: -2
\end{itemize}
\end{frame}

\begin{frame}{Charge, Conductors and Insulators}
A rod with a positive charge is held next to a \textit{conductor} (an object were charge can move around freely).  Which of the following is true?
\begin{itemize}
\item A: The charges in the conductor all remain in place because charge is conserved.
\item B: The negative charges in the conductor move toward the positive charges in the rod.
\item C: The positive charges remain in place but the negative charges move away from the rod.
\item D: The positive charges move toward the rod and the negative charges remain in place.
\end{itemize}
\end{frame}

\begin{frame}{Charge, Conductors and Insulators}
An \textit{insulator} with a net positive charge is held next to an \textit{insulator} with a net negative charge.  Which of the following is true?
\begin{itemize}
\item A: The charges in the conductor all remain in place, and the force is attractive.
\item B: The charges in the conductor all move around until the force is attractive.
\item C: The charges in the conductor all remain in place, and the force is repellent.
\item D: The charges in the conductor all move around until the force is repellent.
\end{itemize}
\end{frame}

\begin{frame}{Coulomb’s Law and Electric Fields}
The boundary conditions of problems can vary depending on the materials involved: \\ \vspace{0.5cm}
\textbf{Insulator}: A material in which there are no free charges available to conduct electricity.  Charges may be fixed in position within an insulator. \\
\textbf{Conductor}: A material in which there are free charges available to conduct electricity.  Charges may not be fixed in position within a conductor. \\
\textbf{Semi-conductor}: A material in which there are free charges available to conduct electricity if certain requirements are met.
\end{frame}

\section{Activity: PhET Charges and Fields}

\begin{frame}{Activity: PhET Charges and Fields}
At your tables, go to the following URL: \\ \vspace{0.2cm}
\url{https://phet.colorado.edu/en/simulation/charges-and-fields} \\ \vspace{0.2cm}
Click on the java app to get it running.  Notice the following:
\begin{enumerate}
\item This is a 2D coordinate space, and you can activate the grid lines at right, by clicking \textit{grid.}
\item Clicking \textit{values} gives you the measurement scale.
\item Click \textit{electric field}, or make sure it is activated.
\item Verify the length scale with the \textbf{ruler tool}, shaped like a tape measure.  It can be dragged from the box at right.
\end{enumerate}
\end{frame}

\begin{frame}{Activity: PhET Charges and Fields}
\small
\url{https://phet.colorado.edu/en/simulation/charges-and-fields} \\ \vspace{0.2cm}
\textbf{Click and drag a positive charge into the 2D coordinate system.  This is analagous to charging an insulator.}
\begin{enumerate}
\item Drag the yellow tool at the bottom into the space, and use it to measure the field strength.  Notice the units are in V/m and m.
\item Write down in your notes the field strength versus distance.  Use 25 cm distance increments, and record 15 data points in two columns.
\item Copy the data to Excel. Let $r$ be the distance. In a third column, multiply the field strength by $r^{2}$.
\item In a fourth column, compute the base-10 logarithm of $r^2$ times the field strength.
\end{enumerate}
\end{frame}

\begin{frame}{Activity: PhET Charges and Fields}
\small
\url{https://phet.colorado.edu/en/simulation/charges-and-fields} \\ \vspace{0.2cm}
\textbf{Click and drag a positive charge into the 2D coordinate system.  This is analagous to charging an insulator.}
\begin{enumerate}
\item Plot $r^2 E$ vs. $r$ and estimate the slope.  How close to zero do you think it is?  What are some sources of error that contribute to the uncertainty in the slope?
\item Repeat this same exercise, but instead of measuring field strength versus \textit{distance}, measure it in one location, versus \textit{charge.} Take 15 data points in two columns and plot the results in Excel.  What is the slope of the line?  Notice the units of charge are nC.
\end{enumerate}
\end{frame}

\begin{frame}{Charge, Conductors and Insulators}
\centering
\textbf{\alert{Charge: the constant of proportionality between the strength of a \textit{field} and the force a field exerts on an \textit{object}.}} \\
\hrulefill
\small
\begin{columns}[T]
\begin{column}{0.5\textwidth}
\alert{Gravity}
\begin{enumerate}
\item Force: $\vec{F} = G \frac{m M}{r^2} \hat{r}$
\item Parameters: $r$ is absolute distance between two objects with masses $m$ and $M$, and the direction is $\hat{r}$
\item \textit{Charge} of one object: $m$
\item \textit{Field felt by that object}: $\vec{G} = G \frac{M}{r^2} \hat{r}$
\item $\vec{F} = m \vec{G}$
\end{enumerate}
\end{column}
\begin{column}{0.5\textwidth}
\alert{Electricity}
\begin{enumerate}
\item Force: $\vec{F} = k \frac{q Q}{r^2} \hat{r}$
\item Parameters: $r$ is absolute distance between two objects with electric charges $q$ and $Q$, and the direction is $\hat{r}$
\item \textit{Charge} of one object: $q$
\item \textit{Field felt by that object}: $\vec{E} = G \frac{Q}{r^2} \hat{r}$
\item $\vec{F} = q \vec{E}$
\end{enumerate}
\end{column}
\end{columns}
\end{frame}

\begin{frame}{Charge, Conductors and Insulators}
\centering
\textbf{\alert{Charge: the constant of proportionality between the strength of a \textit{field} and the force a field exerts on an \textit{object}.}} \\
\hrulefill \\
\small
In the field paradigm, objects with charges \textit{emanate} fields, causing other objects with charge to experience force. \\
\hrulefill \\
\begin{columns}[T]
\begin{column}{0.5\textwidth}
\alert{Gravity} \\
How many \textit{types of charge}, or how many charges, exist under the force of gravity? \\
\textbf{One.} We call it mass.
\end{column}
\begin{column}{0.5\textwidth}
\alert{Electricity} \\
How many \textit{types of charge}, or how many charges, exist under the force of electricity? \\
\textbf{Two.} We call one positive, and one negative.
\end{column}
\end{columns}
\end{frame}

\begin{frame}{Charge, Conductors and Insulators}
\centering
\textbf{\alert{Charge: the constant of proportionality between the strength of a \textit{field} and the force a field exerts on an \textit{object}.}} \\
\hrulefill \\
\small
In the field paradigm, objects with charges \textit{emanate} fields, causing other objects with charge to experience force. \\
\hrulefill \\
In the field paradigm, gravity has one charge (mass), and electricity has two charges (positive and negative). \\
\hrulefill \\
\textbf{There is one fundamental fact that is puzzling.} What about Newton's 2nd law?  Acceleration is not a field, it is a kinematic function.
\begin{equation}
\vec{F}_{\rm net} = m \vec{a}
\end{equation}
Aparently there are \textit{two kinds of mass}: \textbf{inertial} and \textbf{gravitational}.  
\end{frame}

\begin{frame}{Charge, Conductors and Insulators}
\small
\textit{Equivalence principle:} \\ \hrulefill \\
There are \textit{two kinds of mass}: \textbf{inertial} and \textbf{gravitational}, with \textbf{equal value} for a given object. \\ \vspace{0.5cm}
\url{https://en.wikipedia.org/wiki/Equivalence_principle} \\
\hrulefill \\
There is no similar principle for charge.  If the electric force on a charged object is calculated, that force must still be inserted into \textbf{Newton's 2nd Law} to obtain the acceleration, and the inertial mass must be known.
\end{frame}

\section{Coulomb’s Law and Electric Fields}

\begin{frame}{Coulomb’s Law and Electric Fields}
\textbf{Coulomb's Law} describes the force between charges. \\ \vspace{0.5cm}
\begin{tcolorbox}[colback=white,colframe=black!100!black,title=Coulomb's Law]
\alert{The electric force, or \textbf{Coulomb force}, between two electrically charged systems with charges $q_{\rm 1}$ and $q_{\rm 2}$ separated by a distance $r$ is
\begin{equation}
\vec{F}_{\rm C} = \frac{1}{4\pi\epsilon_{\rm 0}} \frac{q_{\rm 1} q_{\rm 2}}{r^2} \hat{r} \label{eq:C}
\end{equation}
In Eq. \ref{eq:C}, $\hat{r} = \vec{r}/|\vec{r}|$, and $\epsilon_{\rm 0} = 8.85418782\times 10^{-12} N^{-1} m^{-2} C^2$, called the \textit{perimittivity of free space.}}
\end{tcolorbox}
\end{frame}

\begin{frame}{Coulomb’s Law and Electric Fields}
\begin{tcolorbox}[colback=white,colframe=black!100!black,title=Coulomb Field]
\alert{The electric field corresponding to Eq. \ref{eq:C}, experienced by a charge $q$ and generated by a charge $Q$ is 
\begin{equation}
\vec{E}_{\rm C} = \frac{1}{4\pi\epsilon_{\rm 0}} \frac{Q}{r^2} \hat{r} \label{eq:Cf}
\end{equation}
In Eq. \ref{eq:Cf}, $r$ remains the separation between $q$ and $Q$.}
\end{tcolorbox}
Thus we have: $\vec{F}_{\rm C} = q \vec{E}_{\rm C}$. \\ \vspace{0.5cm}
The SI Unit of charge is the Coulomb, which is equal to the amount of charge in a "current" of 1 amp for 1 second (more on this later).  \textbf{The charge of an electron is $1.6\times 10^{-19}$ Coulombs, or C.}
\end{frame}

\begin{frame}{Coulomb’s Law and Electric Fields}
Suppose a charge $+q$ experiences the Coulomb field of another charge of $-2q$, separated by a distance $r$.  Which of the following is true, if the charges are in free space?
\begin{itemize}
\item A: The charge $+q$ accelerates towards the other charge, and the charge $-2q$ remains stationary, because it is larger.
\item B: The charge $-2q$ accelerates towards the other charge, and the charge $+q$ remains stationary, because it is the positive charge.
\item C: The charge $-2q$ accelerates towards the other charge, and the charge $+q$ remains stationary, because it is smaller.
\item D: The charges accelerate towards each other.
\end{itemize}
\end{frame}

\begin{frame}{Coulomb’s Law and Electric Fields}
Newton's Third Law still applies, but in \textit{materials} the positive charges are stationary.
\begin{figure}
\centering
\includegraphics[width=0.9\textwidth]{figures/third.png}
\caption{\label{fig:third} Newton's Third Law applies to the Coulomb force, as it does for all forces.}
\end{figure}
\end{frame}

\begin{frame}{Coulomb’s Law and Electric Fields}
Suppose a charge $2q$ is at rest at the origin (0,0).  Where should we place a charge $-2q$ such that the field is zero at (0,3)?
\begin{itemize}
\item A: (0,3)
\item B: (0,6)
\item C: (3,0)
\item D: (6,0)
\end{itemize}
\end{frame}

\begin{frame}{Coulomb’s Law and Electric Fields}
\small
The Coulomb force equation gives a vector, and so does the corresponding electric field.  Like a gravitational field, this effect has a vector at each point in space, so we refer to the Coulomb force and the Coulomb field as \textit{vector fields}. \\ \vspace{0.5cm}
\textbf{Vector field}: An assignment of a vector to each point in a subset of space.
\end{frame}

\begin{frame}{Coulomb’s Law and Electric Fields}
\small
\begin{columns}[T]
\begin{column}{0.5\textwidth}
Which of the following is true of vectors $\vec{v}_{\rm i}$ in the lower left-hand corner of the figure at right?
\begin{itemize}
\item A: They are probably $\vec{v}_{\rm i} = -\hat{i}-\hat{j}$
\item B: They are probably $\vec{v}_{\rm i} = \hat{i}+\hat{j}$
\item C: They are probably $\vec{v}_{\rm i} = -\hat{i}+\hat{j}$
\item D: They are probably $\vec{v}_{\rm i} = \hat{i}-\hat{j}$
\end{itemize}
\end{column}
\begin{column}{0.5\textwidth}
\begin{figure}
\includegraphics[width=\textwidth]{figures/vectorField.png}
\caption{\label{fig:field} A vector field of vectors $\vec{v}_{\rm i}$.  Let $\hat{j}$ represent up, and $\hat{i}$ represent right.}
\end{figure}
\end{column}
\end{columns}
\end{frame}

\begin{frame}{Coulomb’s Law and Electric Fields}
\small
\begin{columns}[T]
\begin{column}{0.5\textwidth}
Which of the following is true of vectors $\vec{v}_{\rm i}$ in the upper left-hand corner of the figure at right?
\begin{itemize}
\item A: They are probably $\vec{v}_{\rm i} = -\hat{i}-\hat{j}$
\item B: They are probably $\vec{v}_{\rm i} = \hat{i}+\hat{j}$
\item C: They are probably $\vec{v}_{\rm i} = -\hat{i}+\hat{j}$
\item D: They are probably $\vec{v}_{\rm i} = \hat{i}-\hat{j}$
\end{itemize}
\end{column}
\begin{column}{0.5\textwidth}
\begin{figure}
\includegraphics[width=\textwidth]{figures/vectorField.png}
\caption{\label{fig:field2} A vector field of vectors $\vec{v}_{\rm i}$.  Let $\hat{j}$ represent up, and $\hat{i}$ represent right.}
\end{figure}
\end{column}
\end{columns}
\end{frame}

\begin{frame}{Coulomb’s Law and Electric Fields}
\small
\begin{columns}[T]
\begin{column}{0.5\textwidth}
What is the angle of the E-field at point (1,1) in Fig. \ref{fig:netfield2} at right?
\begin{itemize}
\item A: 0 deg
\item B: 45 deg
\item C: 90 deg
\item D: 135 deg
\end{itemize}
What is the fastest way to solve this problem?
\begin{itemize}
\item A: Blind luck
\item B: Do the algebra
\item C: Symmetry
\item D: Numerical estimation
\end{itemize}
\end{column}
\begin{column}{0.5\textwidth}
\begin{figure}
\includegraphics[width=\textwidth]{figures/NetField2.pdf}
\caption{\label{fig:netfield2} Two charges create a field for a hypothetical \textit{test charge}.}
\end{figure}
\end{column}
\end{columns}
\end{frame}

\begin{frame}{Coulomb’s Law and Electric Fields}
\small
\begin{columns}[T]
\begin{column}{0.5\textwidth}
\textbf{Group board exercise}: What is the angle of the net electric field for the \textit{test charge} at the point (1,1) in Fig. \ref{fig:netfield1}? \\ \vspace{0.5cm}
(Professor: do one example first)
\end{column}
\begin{column}{0.5\textwidth}
\begin{figure}
\includegraphics[width=\textwidth]{figures/NetField1.pdf}
\caption{\label{fig:netfield1} Three charges create a field for a hypothetical \textit{test charge}.}
\end{figure}
\end{column}
\end{columns}
\end{frame}

\section{The Superposition Principle and Symmetry}

\begin{frame}{Coulomb’s Law and Electric Fields}
The forces of $N$ fixed charges on a test charge $Q$ create a net force, where the individual forces simply add like vectors.  This is known as the \textbf{superposition principle}.
\begin{align}
\vec{F}_{\rm C,Net} &= \frac{1}{4\pi\epsilon_{\rm 0}} Q \sum_{i = 1}^N \frac{q_i}{r_i^2}\hat{r}_i = Q \vec{E}_{\rm C,Net} \\
\vec{E}_{\rm C,Net} &= \frac{1}{4\pi\epsilon_{\rm 0}} \sum_{i = 1}^N \frac{q_i}{r_i^2}\hat{r}_i
\end{align}
\end{frame}

\begin{frame}{Coulomb’s Law and Electric Fields}
For the expressions of fields built from the superposition principle, let's adopt a notation:
\begin{equation}
\vec{E}_{\rm C,Net}(P) = \frac{1}{4\pi\epsilon_{\rm 0}} \sum_{i = 1}^N \frac{q_i}{r_i^2}\hat{r}_i \label{eq:P}
\end{equation}
Equation \ref{eq:P} represents the field at a \textit{position} $P = P(x,y,z)$, relative to the positions $\vec{r}_i$ of the source charges.
\end{frame}

\begin{frame}{Coulomb’s Law and Electric Fields}
\small
\begin{columns}[T]
\begin{column}{0.5\textwidth}
\textbf{Table exercise:} Calculate $\vec{E}_{\rm C,Net}(P)$, if $P = (1,1)$. \\ \vspace{0.5cm}
\textbf{Table exercise:} Calculate $\vec{E}_{\rm C,Net}(P)$, if $P = (-1,-1)$. \\ \vspace{0.5cm}
\textbf{Group discussion:} What does it mean if $P = (1,0)$? \\ \vspace{0.5cm}
\end{column}
\begin{column}{0.5\textwidth}
\begin{figure}
\includegraphics[width=\textwidth]{figures/NetField4.pdf}
\caption{\label{fig:netfield4} Two charges create a field for a hypothetical \textit{test charge}.}
\end{figure}
\end{column}
\end{columns}
\end{frame}

\begin{frame}{Coulomb’s Law and Electric Fields}
\small
\begin{columns}[T]
\begin{column}{0.5\textwidth}
The following problem is an example of solving for a field analytically, and \textit{testing various limits}.  Upon taking limits results are often simple and intuitive. \\ \vspace{0.5cm}
Two charges $+q$ are on the fixed in an insulator on the x-axis.  Solve for the E-field at $P = (0,0,z)$. \\ \vspace{0.5cm}
Show that the general solution is
\begin{equation}
\vec{E}(z) = \frac{1}{4\pi\epsilon_0} \frac{2qz}{\left(z^2+\left(\frac{d}{2}\right)^2\right)^{3/2}} \hat{k}
\end{equation}
\end{column}
\begin{column}{0.5\textwidth}
\begin{figure}
\includegraphics[width=\textwidth]{figures/twoChargesZ.png}
\caption{\label{fig:twoChargesZ} Solve for the E-field as a function of $z$, $d$, and $q$.}
\end{figure}
\end{column}
\end{columns}
\end{frame}

\begin{frame}{Coulomb’s Law and Electric Fields}
\small
\begin{columns}[T]
\begin{column}{0.5\textwidth}
Show that the general solution is
\begin{equation}
\vec{E}(z) = \frac{1}{4\pi\epsilon_0} \frac{2qz}{\left(z^2+\left(\frac{d}{2}\right)^2\right)^{3/2}} \hat{k}
\end{equation}
\textit{Take the following two limits:} \\ 1) $z \gg d$ and 2) $z=0$.  What are the results? \\ \vspace{0.5cm}
Keep these results in mind, because we are about to start drawing \textbf{vector fields,} in order to visualize the algebra.
\end{column}
\begin{column}{0.5\textwidth}
\begin{figure}
\includegraphics[width=\textwidth]{figures/twoChargesZ.png}
\caption{\label{fig:twoChargesZ2} Solve for the E-field as a function of $z$, $d$, and $q$.}
\end{figure}
\end{column}
\end{columns}
\end{frame}

\begin{frame}{Coulomb’s Law and Electric Fields}
\textbf{PhET Simulation of E-fields from Charges}: \\ \vspace{0.5cm}
\url{https://phet.colorado.edu/en/simulation/charges-and-fields}
\begin{enumerate}
\item Create the situation in the prior problem, in Fig. \ref{fig:twoChargesZ2}.
\item Use the yellow sensor object to determine the local direction of the E-field at various points along the z-axis.
\begin{itemize}
\item Do the results match the limit $z\gg d$?
\item Do the results match the limit $z = 0$, halfway between the charges?
\item Where is the field maximal?
\end{itemize}
\item Make sure you can see above and below the charges, and repeat steps 1 and 2 for negative z-values.  What do you find?
\end{enumerate}
\end{frame}

\begin{frame}{Coulomb’s Law and Electric Fields}
\small
\textbf{PhET Simulation of E-fields from Charges}: \\ \vspace{0.5cm}
Build E-fields with the following properties, by adding single charges.  Let the \textit{z-axis be upwards}, and let the \textit{x-axis be to the right.}
\begin{enumerate}
\item Build an electric field that has \textbf{reflection symmetry} across the z-axis, with at least five charges.
\item Build an electric field that has \textit{radial symmetry} about the origin, with at least six charges.
\item Build an electric field that would be the same if I rotated the picture by 90 degrees (\textbf{4-fold symmetry}) with at least four charges, some negative and some positive.
\item Build an electric field that would be the same if I rotated the picture by 45 degrees (\textbf{8-fold symmetry}) with at least eight charges, some negative and some positive.
\end{enumerate}
\end{frame}

\begin{frame}{Coulomb’s Law and Electric Fields}
\small
\textbf{PhET Simulation of E-fields from Charges}: \\ \vspace{0.5cm}
\alert{The lesson is that the E-field has the \textit{symmetry properties} of the \textit{charge distribution}}.
\end{frame}

\begin{frame}{Coulomb’s Law and Electric Fields}
When we connect the vectors in a vector field, the results are figures like Fig. \ref{fig:lines}.  Fields by convention originate from positive charges and terminate on negative ones.
\begin{figure}
\includegraphics[width=0.8\textwidth]{figures/lines.png}
\caption{\label{fig:lines} Field-line diagrams.  The density of lines indicates electric field strength.}
\end{figure}
\end{frame}

\begin{frame}{Coulomb’s Law and Electric Fields}
\small
Different PhET simulations and programs to illustrate field lines:
\begin{enumerate}
\item \textbf{\url{http://www.flashphysics.org/electricField.html}}
\item \textit{\url{https://phet.colorado.edu/en/simulation/electric-hockey}}
\item \textit{\url{https://phet.colorado.edu/en/simulation/efield}}
\end{enumerate}
Homework bonus point: use (1) to draw the electric field of a water molecule.  Make sure it has the correct number of positive and negative charges, in the correct positions.
\end{frame}

\section{E-Fields of Charge Distributions}

\begin{frame}{E-Fields of Charge Distributions}
Welcome to calculus! Let $k = 1/(4\pi\epsilon_0)$.
\begin{align}
\vec{E}(P) &= k \sum_{i = 1}^N \left(\frac{q_i}{r_i^2}\right) \hat{r} \label{eq:point} \\
\vec{E}(P) &= k \int_{line} \left(\frac{\lambda dl}{r^2}\right) \hat{r} \label{eq:line} \\
\vec{E}(P) &= k \int_{surface} \left(\frac{\sigma dA}{r^2}\right) \hat{r} \label{eq:surface} \\
\vec{E}(P) &= k \int_{volume} \left(\frac{\rho dV}{r^2}\right) \hat{r} \label{eq:volume}
\end{align}
The functions $\lambda$, $\sigma$, and $\rho$ are just charge densities.  They decribe where charge is, and how much there is.  
\end{frame}

\begin{frame}{E-Fields of Charge Distributions}
\begin{figure}
\includegraphics[width=0.45\textwidth]{figures/disk.png}
\caption{\label{fig:disk} We are going to work examples like this together.}
\end{figure}
\textbf{Observe on board.}
\end{frame}

\begin{frame}{E-Fields of Charge Distributions}
\begin{figure}
\includegraphics[width=0.4\textwidth]{figures/disk.png}
\caption{\label{fig:disk2} We are going to work examples like this together.}
\end{figure}
Result:
\begin{equation}
\boxed{
\vec{E} = k\left(2\pi\sigma - \frac{2\pi\sigma z}{\sqrt{R^2 + z^2}} \right)\hat{k}}
\end{equation}
\end{frame}

\begin{frame}{E-Fields of Charge Distributions}
\begin{equation}
\boxed{
\vec{E} = k\left(2\pi\sigma - \frac{2\pi\sigma z}{\sqrt{R^2 + z^2}} \right)\hat{k}} \label{eq:disk}
\end{equation}
Which of the following is true of Eq. \ref{eq:disk}?
\begin{itemize}
\item A: Taking the limit $R \rightarrow \infty$ yields a constant field.
\item B: Taking the limit $z \rightarrow 0$ yields a constant field.
\item C: The charge distribution has radial symmetry, so the field cannot have horizontal components.
\item D: A, B, and C
\end{itemize}
\end{frame}

\begin{frame}{E-Fields of Charge Distributions}
In the limit that $R \rightarrow \infty$,
\begin{equation}
\vec{E} = 2\pi\sigma k \hat{k} = \frac{\sigma}{2\epsilon_0} \hat{k} \label{eq:disk3}
\end{equation}
Equation for the electric field of a uniform infinite disk.
\end{frame}

\begin{frame}{E-Fields of Charge Distributions}
Imagine two infinite disks with equal uniform charge distributions, some distance apart.  One has positive charge, the other negative charge.  What is the E-field between them?
\begin{itemize}
\item A: 0
\item B: $\frac{\sigma}{2\epsilon_0}$
\item C: $\frac{\sigma}{\epsilon_0}$
\item D: $\frac{\sigma}{4\epsilon_0}$
\end{itemize}
\end{frame}

\begin{frame}{E-Fields of Charge Distributions}
Imagine two infinite disks with equal uniform charge distributions, some distance apart.  Both have positive charge.  What is the E-field between them?
\begin{itemize}
\item A: 0
\item B: $\frac{\sigma}{2\epsilon_0}$
\item C: $\frac{\sigma}{\epsilon_0}$
\item D: $\frac{\sigma}{4\epsilon_0}$
\end{itemize}
\end{frame}

\begin{frame}{E-Fields of Charge Distributions}
Other interesting charge distributions:
\begin{itemize}
\item A line of charge with length $L$ and total charge $Q = \lambda L$, where $P = (0,0,z)$ above midpoint:
\begin{equation}
\vec{E}(z) = \frac{1}{4\pi\epsilon_0} \frac{\lambda L}{z\sqrt{z^2 + \frac{1}{4} L^2}} \hat{k} \label{eq:line2}
\end{equation}
\item Equation \ref{eq:line2}, but with $L \rightarrow \infty$:
\begin{equation}
\vec{E}(z) = \frac{1}{4\pi\epsilon_0} \frac{2\lambda}{z} \hat{k}
\end{equation}
\end{itemize}
\begin{itemize}
\item What quantity does $\lambda L$ represent?
\item Why is the field strictly in the $\hat{k}$ direction?
\item What happens if $L \ll z$? (Professor demonstrate on board).
\end{itemize}
\end{frame}

\begin{frame}{E-Fields of Charge Distributions}
Other interesting charge distributions:
\begin{itemize}
\item A ring of radius $R$ and total charge $Q = 2\pi R\lambda$, where $P = (0,0,z)$ above midpoint:
\begin{equation}
\vec{E}(z) = \frac{1}{4\pi\epsilon_0} \frac{2\pi R \lambda z}{\left(z^2 + R^2\right)^{3/2}} \hat{k} \label{eq:ring}
\end{equation}
\item Equation \ref{eq:ring}, but with $z \gg R$:
\begin{equation}
\vec{E}(z) = \frac{1}{4\pi\epsilon_0} \frac{2\pi R \lambda }{z^2} \hat{k} \label{eq:ring2}
\end{equation}
\end{itemize}
\end{frame}

\begin{frame}{E-Fields of Charge Distributions}
Let $Q_{\rm tot} = 2\pi R \lambda$.  That makes Eq. \ref{eq:ring2}
\begin{equation}
\vec{E}(z) = \frac{1}{4\pi\epsilon_0} \frac{Q_{\rm tot}}{z^2} \hat{k}
\end{equation}
This is identical to the electric field of what charge distribution?  (Think back to the definition of the electric field).
\begin{itemize}
\item A: A plane with charge density $Q_{\rm tot}/A$, where $A$ is the area
\item B: A line with total charge $Q_{\rm tot}$
\item C: A dipole of charge $\pm Q_{\rm tot}$
\item D: A point charge $Q_{\rm tot}$
\end{itemize}
\end{frame}

\begin{frame}{E-Fields of Charge Distributions}
As we shall see, the last few results follow from a notion known as \alert{Gauss's Law}.  First we need two concepts:
\begin{itemize}
\item The \textbf{area vector} of a surface
\item The concept of flux
\end{itemize}
\end{frame}

\section{Gauss's Law}

\begin{frame}{Gauss's Law}
Let $\vec{A}$ be a vector that:
\begin{itemize}
\item has a magnitude equal to the area of a surface
\item has a direction that is orthogonal to the surface
\end{itemize}
If a surface has area $A$, then $\vec{A} = A \hat{n}$, where $\hat{n}$ is normal, and pointed outward orthogonally from the surface.  What does \textit{outward} mean?
\end{frame}

\begin{frame}{Gauss's Law}
\begin{figure}
\centering
\includegraphics[width=0.5\textwidth]{figures/areaVector.png}
\caption{\label{fig:n} The convention for the area vector direction is outward, not inward.}
\end{figure}
\end{frame}

\begin{frame}{Gauss's Law}
\begin{figure}
\centering
\includegraphics[width=0.5\textwidth]{figures/patch.png}
\caption{\label{fig:patch} We may think of a surface $S$ as the sum of many infinitesimal square patches $dS_i$, each with an area vector $\vec{dA}_i$ equal in magnitude but not direction.}
\end{figure}
\end{frame}

\begin{frame}{Gauss's Law}
A rectangular surface has length $a$, width $b$, and it is located in the x-y plane.  What is the area vector of the rectangular surface?
\begin{itemize}
\item A: $b^2 \hat{k}$
\item B: $a^2 \hat{i}$
\item C: $ab \hat{k}$
\item D: $ab \hat{j}$
\end{itemize}
\end{frame}

\begin{frame}{Gauss's Law}
\small
\begin{figure}
\centering
\includegraphics[width=0.5\textwidth]{figures/coin.png}
\caption{\label{fig:coin} A circular patch, with an external electric field.  How many electric field lines would pass through the circular patch if it was tilted to 90$^{\circ}$ from the field?  How about 0$^{\circ}$?}
\end{figure}
This behavior indicates a \textbf{dot-product} is working (zero and maximal over a span of 90 degrees).  But a dot-product of which two quantities?
\end{frame}

\begin{frame}{Gauss's Law}
Electric flux:
\begin{equation}
\boxed{
\Phi = \vec{E} \cdot \vec{A}} = E A \cos\theta
\end{equation}
Assumptions:
\begin{itemize}
\item $\theta$ is the angle between the area vector and the field
\item The electric field is uniform over this surface
\item The surface is flat
\end{itemize}
These assumptions do not hold for any arbitrary configuration of charge and surfaces.  However, if we zoom in closely enough to an individual patch, it does. \\ \vspace{0.5cm}
The units of electric flux are N C$^{-1}$ m$^2$.
\end{frame}

\begin{frame}{Gauss's Law}
How can we obtain the flux of more complex electric fields through more complex surfaces?  Simple: zoom in, get the flux from a patch, add it to the total:
\begin{equation}
\Phi = \sum_i^N \vec{E}_i \cdot d\vec{A}_i
\end{equation}
In situations like this, the summation usually becomes an integral if we make $dA$ small and $N$ very large.  That is, break the surface into many small patches.  But now we are dealing with two vectors, $\vec{E}$ and $d\vec{A}$...
\end{frame}

\begin{frame}{Gauss's Law}
\begin{equation}
\boxed{
\Phi = \int_{\rm S} \vec{E} \cdot d\vec{A}} \label{eq:flux}
\end{equation}
Equation \ref{eq:flux} is known as a \textit{surface integral.}  We can do these!  Let's try an easy one.  Let $\vec{E} = E_0 x \hat{k}$, and $S$ be a square of width $x_0$ and height $y_0$ in the x-y plane. \\ \vspace{0.5cm}
\textbf{Group board exercise:} Compute the electric flux $\Phi$, and check the units to make sure they are correct.
\end{frame}

\begin{frame}{Gauss's Law}
$\Phi = \frac{1}{2} E_0 y_0 x_0^2$.  (This has units of electric field times area).  Which of the following would increase the flux?
\begin{itemize}
\item A: If $x_0$ or $y_0$ were to grow larger
\item B: If $E_0$ were to grow larger
\item C: If $\vec{E}$ were directed at an angle to the z-axis
\item D: Both A and B
\end{itemize}
\end{frame}

\begin{frame}{Gauss's Law}
Imagine now that there is another, identical surface \textit{above} the first surface, but it is \textit{upside-down}.  The electric field passes through both surfaces.  What is the total flux, the sum of the fluxes through both surfaces?
\begin{itemize}
\item A: $\Phi = \frac{1}{2} E_0 y_0 x_0^2$
\item B: $\Phi = E_0 y_0 x_0^2$
\item C: $0$
\item D: $\Phi = -\frac{1}{2} E_0 y_0 x_0^2$
\end{itemize}
\end{frame}

\begin{frame}{Gauss's Law}
The answer is zero because, for the \textit{upside-down} surface, the area vector is in the direction $-\hat{k}$.  In fact, this result applies to any \textbf{closed surface}:
\begin{figure}
\centering
\includegraphics[width=0.4\textwidth]{figures/cube.png}
\caption{\label{fig:cube} The flux through a closed surface due to an external field is zero.}
\end{figure}
\end{frame}

\begin{frame}{Gauss's Law}
We can encapsulate this idea in the following equation:
\begin{equation}
\boxed{
\Phi = \oint_{\rm S} \vec{E}_{\rm ext} \cdot d\vec{A} = 0} \label{eq:ext}
\end{equation}
In general:
\begin{equation}
\boxed{
\Phi = \oint_{\rm S} \vec{E} \cdot d\vec{A}} \label{eq:general}
\end{equation}
However this implies something interesting about closed surfaces.  If the flux is non-zero through a closed surface, the field cannot be an external field.  It has to be an \textit{internal field}.
\end{frame}

\begin{frame}{Gauss's Law}
How do you make an electric field \textit{internal} to a closed surface?  How do you make an electric field in general?  \textbf{Charge.}  Charge is the origin of any electric field.
\begin{equation}
\boxed{
\Phi = \oint_{\rm S} \vec{E} \cdot d\vec{A} \propto Q} \label{eq:general2}
\end{equation}
If the charge $Q$ is outside the surface, then by definition the field is external to the surface.  If the surface \textit{encloses the charge}, then the total flux is non-zero, and\footnote{Formal proof relies on the \textit{divergence theorem}, from Calculus III.}
\begin{equation}
\boxed{
\frac{Q_{\rm enc}}{\epsilon_0} = \oint_{\rm S} \vec{E} \cdot d\vec{A}} \label{eq:gauss}
\end{equation}
\end{frame}

\begin{frame}{Gauss's Law}
\textbf{Group exercise:} A charge $Q$ is at the origin.  Compute the electric field via Gauss's Law, at $P = (0,0,R)$. \\ \vspace{0.5cm}
\textbf{Group exercise:} A line of charge with length $L$ and $Q = \lambda L$ runs along the x-axis with the center at the origin.  Compute the electric field via Gauss's Law, at $P = (0,0,R)$. \\ \vspace{0.5cm}
\textbf{Group exercise:} A plane of charge situated in the x-y plane with radius $R$ and charge per unit area $\sigma$.  Compute the electric field via Gauss's Law, at $P = (0,0,\epsilon)$, where $\epsilon \ll R$. \\ \vspace{0.5cm}
\end{frame}

\begin{frame}{Gauss's Law}
Imagine a spherical shell conductor with the charge distributed along the surface.  Which of the following is true of the electric field?
\begin{itemize}
\item A: The electric field is positive and varies in direction inside the conductor.
\item B: The electric field is positive and points toward the origin inside the conductor.
\item C: The electric field is negative and points toward the origin inside the conductor.
\item D: The electric field is zero inside the conductor.
\end{itemize}
\end{frame}

\begin{frame}{Gauss's Law}
\begin{figure}
\centering
\includegraphics[width=0.6\textwidth]{figures/gaussianConductor.png}
\caption{\label{fig:cond1} Imagine a Gaussian surface inside a conductor, just below the surface charge distribution.}
\end{figure}
\end{frame}

\begin{frame}{Gauss's Law}
\begin{figure}
\centering
\includegraphics[width=0.6\textwidth]{figures/gaussianConductor2.png}
\caption{\label{fig:cond2} If the Gaussian surface is as pictured, and there are no other charges, then the field is $E = \sigma/\epsilon_0$.}
\end{figure}
\end{frame}

\begin{frame}{Gauss's Law}
\begin{figure}
\centering
\includegraphics[width=0.45\textwidth]{figures/charge3D2.png}
\caption{\label{fig:charge3D2} Imagine a charge distribution that varies with radius.  How would we determine the electric field?}
\end{figure}
\end{frame}

\begin{frame}{Gauss's Law}
\begin{figure}
\centering
\includegraphics[width=0.6\textwidth]{figures/charge3D.png}
\caption{\label{fig:charge3D} We can show that if the charge is evenly distributed (imagine an atom) then Gauss' law will tell us the radial dependence.}
\end{figure}
\textbf{Observe on board.} What if the charge now varies with radius? 
\end{frame}

\section{Conclusion}

\begin{frame}{Unit 3 Summary}
\textbf{Reading: Chapters 5-6}
\begin{enumerate}
\item Charge, Conductors and Insulators
\item Coulomb's Law and Electric Fields
\item E-fields of Charge Distributions
\item Gauss's Law
\end{enumerate}
\end{frame}

\end{document}

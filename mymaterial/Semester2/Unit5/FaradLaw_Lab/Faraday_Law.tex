\documentclass[12pt]{article}
\usepackage[margin=1.5cm]{geometry}
\usepackage{amsmath}
\usepackage{graphicx}
\title{Faraday's Law: Magnetic oscillation measurement}

\begin{document}
\maketitle

\section{Introduction}

Let $\mathcal{E}$ be the emf induced in a conductor with $N$ coils or loops by a changing magnetic flux $\phi = \vec{B} \cdot \vec{A}$ with respect to time.  Faraday's Law states that

\begin{equation}
\mathcal{E} = -N \frac{d\phi}{dt}
\end{equation}

The minus sign indicates Lenz's Law, which states that the emf will correspond to a current flowing in the circuit that \textit{opposes} the change in flux.  Suppose we had a magnetic field that oscillated in strength, with amplitude $B_0$, angular frequency $\omega$, and angular phase $\phi_0$:

\begin{equation}
B(t) = B_0 \cos(\omega t + \phi_0)
\end{equation}

Further, assume that this field is oscillating in a solenoid with $N$ turns and radius $r$.  The flux would then depend on time:

\begin{equation}
\phi = B_0 \pi r^2 \cos(\omega t + \phi_0)
\end{equation}

Taking the first derivative and multiplying by $-N$ gives

\begin{equation}
\mathcal{E} = N \omega B_0 \pi r^2 \sin(\omega t + \phi_0) \label{eq:main}
\end{equation}

Define $\mathcal{E}_0 = N B_0 \pi r^2$, so that (what are the units of $\mathcal{E}_0$?)

\begin{equation}
\mathcal{E} = \mathcal{E}_0 \omega \sin(\omega t + \phi_0) \label{eq:main2}
\end{equation}

In the following lab, Eq. \ref{eq:main2} will be an approximation for the magnetic field.  The $\vec{B}$-field will originate from an oscillating bar magnet on a spring, dipping in and out of a solenoid.  The voltage expressed in Eq. \ref{eq:main2} will hopefully appear on the oscilloscope, and we are trying to predict and measure $\mathcal{E}_0$.

\section{Setup}

The necessary items will be:

\begin{enumerate}
\item Oscilloscope and power chord, with settings of 10 mV/division vertical, and 500 ms/division horizontal.  Set the trigger to manual, with a threshold of about 10 mV.
\item A solenoid with exactly 80 turns, and radius of $\approx 8$ cm.
\item A coaxial to alligator-clip conversion cable.
\item A bench clamp (black)
\item Three metal rods
\item Two right-angle rod connectors (black plastic)
\item One Vernier LabPro measurement unit
\item One magnetic probe attachment
\item One spring
\item One bar-magnet
\item Strips of tape
\end{enumerate}

Place the solenoid next to the oscilloscope, with one end pointing towards the ceiling, and the other towards the table.  Connect the coaxial cable to channel one, and connect the alligator clips to the solenoid.  The choice of red/black will only change the sign of $\mathcal{E}(t)$.  Connect the clamp to the table and use the other hardware to place a horizontal bar $\approx 5-10$ cm above the top of the solenoid.  Tape the bar magnet to one end of the spring, and hang the bar magnet/spring assembly over the solenoid.  Raise the bar magnet until the spring is fully compressed, and release.  The oscilloscope should trigger, drawing a sinusoidal signal.

\section{Predicting $\mathcal{E}_0$}

Remove the solenoid temporarily, and connect the Vernier LabPro and magnetic probe attachment.  Click on the LoggerPro icon on the Windows desktop and check that the program is configured to measure B-fields. Start the magnet oscillations, and use the probe to measure the B-field versus time.  Orient the white dot to point upwards, and hold the probe at the average position of the magnet as it oscillates.  Record the magnetic field versus time in LoggerPro and transfer the data to Excel.  Plot the B-field versus time below, with the correct units: \\ \vspace{3cm}

\section{The Root-Mean-Square}

The root-mean-square is a statistic used to quantify the size of a series of measurements that has a mean close to zero.  It is defined as

\begin{equation}
x_{rms} = \sqrt{\frac{1}{N}\sum_i x_1^2 + x_2^2 + ... + x_N^2} \label{eq:rms}
\end{equation}

Equation \ref{eq:rms} is close to the definition of the standard deviation encountered in previous labs.  Compute the rms of the B-field in the solenoid using the data in Excel: \\

$B_{rms} = $ \\

\textbf{As a first approximation}, let's treat $B_{rms}$ as the $B_0$ in Eq. \ref{eq:main}.  Now count the number of turns in the solenoid (it should be close to 80).  Record that as your measurement of $N$.  Measure the radius of the wires of the solenoid.  For the moment we are ignoring the statistical errors of these quantities.  Insert these measured values into $\mathcal{E}_0 = N B_0 \pi r^2$, and record your value for $\mathcal{E}_0$ below, with the correct units: \\

$\mathcal{E}_0 = $

\section{RMS voltage from Faraday's Law}

Now reinsert the solenoid under the oscillating bar magnet and trigger on the voltage waveform.  Using the gray measurement button at right, open the menu for measurements.  Select channel 1, and twist the gray highlighted knob to choose RMS for channel 1, and click add measurement.  Similarly, add a measurement for the frequency of the waveform.  Record both numbers below, converting the frequency to angular frequency: \\

$\mathcal{E}_{rms} = $ \\ \\
$~~~~~~\omega = $ \\

Now compare the \textit{predicted} $\mathcal{E}_0 \omega$ to the \textit{measured} $\mathcal{E}_{rms}$.  How close can you get them?  What sources of error are there, and how can this comparison be made more accurate?  Draw the voltage waveform below, with correct depiction of the height and period in volts and seconds, respectively: \\ \vspace{3cm}

\section{Voltage versus Frequency}

Now oscillate the bar magnet with your hand instead of the spring.  Although it's less precise, you can vary the frequency if you are oscillating the bar magnet.  Using the oscilloscope, make a measurement of 10 different $\mathcal{E}_{rms}$ values at 10 different $\omega$ values.  Remember that you'll have to convert frequency to $\omega$.  Make a plot of $\mathcal{E}_{rms}$ versus $\mathcal{E}_0 \omega$ below, with the correct units.  Is it linear?  Why or why not?\\ \vspace{3cm}

\end{document}
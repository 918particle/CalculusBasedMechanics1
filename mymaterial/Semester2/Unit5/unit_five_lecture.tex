\documentclass{beamer}
\usetheme{metropolis}
\usepackage{graphicx}
\usepackage{subfig}
\usepackage{tcolorbox}
\title{Calculus-Based Physics-2: Electricity, Magnetism, and Thermodynamics (PHYS180-02): Unit 5}
\author{Jordan Hanson}
\institute{Whittier College Department of Physics and Astronomy}

\begin{document}
\maketitle

\section{Unit 4 Review}

\begin{frame}{Unit 4 Review}
Suppose a bundle of wires is carrying current along what we call the $\hat{z}$ direction.  Each wire runs along the z-axis and they are close enough to ignore the fact that the volume of each wire prevents it from being exactly on the z-axis.  One wire carries +2.0 A, another carries +1.5 A, and a third carries -0.5 A.  What is the B-field strength at a distance of 1 cm away in the x-y plane?
\begin{itemize}
\item A: 6 Gauss
\item B: 0.6 Gauss
\item C: 6 Tesla
\item D: 0.6 Tesla
\end{itemize}
\end{frame}

\begin{frame}{Unit 4 Review}
Suppose a loop of current exists in the x-y plane, and a uniform B-field is in the $\hat{z}$ direction.  Which of the following will occur?
\begin{itemize}
\item A: The loop will not rotate - there is no torque.
\item B: The loop will rotate 180 degrees - there is torque.
\item C: The loop will rotate 90 degrees - there is torque.
\item D: The loop will rotate -90 degrees - there is negative torque.
\end{itemize}
\end{frame}

\section{Summary}

\begin{frame}{Summary}
\textbf{Reading: Chapters 13 and 14} \\ \vspace{0.5cm}
\textit{This weekend:}
\begin{enumerate}
\item 13.1-2: Faraday's and Lenz's Law
\item 13.3: Motional EMF
\item 13.4: Induced E-fields
\end{enumerate}
\textit{Next week:} Chapter 14.1-3
\end{frame}

\section{Faraday's Law and Lenz's Law}

\begin{frame}{Faraday's Law}
\begin{figure}
\centering
\includegraphics[width=0.9\textwidth]{figures/farad.png}
\caption{\label{fig:farad1} Not only does moving charge create B-fields, but B-fields can create moving charge.  Study each of the cases above, and (Professor) define the concept of \textit{magnetic flux}.}
\end{figure}
\end{frame}

\begin{frame}{Faraday's Law}
\begin{figure}
\centering
\includegraphics[width=0.9\textwidth]{figures/farad2.png}
\caption{\label{fig:farad2} In addition to a moving magnetic field, \textit{other circuits} can make current flow in a circuit.  The effect must have something to do with \textit{changing} magnetic fields.}
\end{figure}
\end{frame}

\begin{frame}{Faraday's Law}
\begin{tcolorbox}[colback=white,colframe=black!40!black,title=Faraday's Law]
\alert{The emf $\epsilon$ induced is the negative change in the magnetic flux $\Phi_m$ per unit time. Any change in the magnetic field
or change in orientation of the area of the coil with respect to the magnetic field induces a voltage (emf).
\begin{align}
\phi_m &= \int_S \vec{B} \cdot d\vec{A} \\
\epsilon &= - \frac{d\phi_m}{dt}
\label{eq:farad}
\end{align}}
\end{tcolorbox}
\textit{The unit of magnetic flux is the Webter, or 1 Wb = 1 T m$^2$.}
\end{frame}

\begin{frame}{Faraday's Law}
\small
\textbf{Example:}
A square coil has sides 0.25 m long and is tightly wound with 200 turns of wire. The resistance of the coil 5.0 Ohms. The coil is placed in a spatially uniform magnetic field that is directed perpendicular to the face of the coil and whose magnitude is decreasing by −0.040 T/s. (a) What is the magnitude of the emf induced in the coil? (b) What is the magnitude of the current circulating through the coil?
\begin{figure}
\centering
\includegraphics[width=0.3\textwidth]{figures/loop1.png}
\caption{\label{fig:loop1} A 200 turn loop in a B-field.}
\end{figure}
\end{frame}

\begin{frame}{Faraday's Law}
\begin{tcolorbox}[colback=white,colframe=black!40!black,title=Lenz's Law]
\alert{The direction of the induced emf drives current around a wire loop to always oppose the change in magnetic flux that
causes the emf.}
\end{tcolorbox}
\end{frame}

\begin{frame}{Faraday's Law}
\small
\textbf{Example:}
A magnetic field B is directed outward perpendicular to the plane of a circular coil of radius r = 0.50 m.  The field is cylindrically symmetrical with respect to the center of the coil, and its magnitude decays exponentially according to
\begin{equation}
B(t) = B_0 \exp(-a t)
\end{equation}
with $B_0 = 1.5$ T and $a = 5.0$ s$^{-1}$.  (a) Calculate the emf induced in the coil at the times $t_0 = 0$, $t_1 = 0.05$, and $t_2 = 1.0$ seconds. (b) Determine the current in the coil if the resistance is 10 Ohms. 
\end{frame}

\section{Conclusion}

\begin{frame}{Summary}
\textbf{Reading: Chapters 13 and 14} \\ \vspace{0.5cm}
\textit{This weekend:}
\begin{enumerate}
\item 13.1-2: Faraday's and Lenz's Law
\item 13.3: Motional EMF
\item 13.4: Induced E-fields
\end{enumerate}
\textit{Next week:} Chapter 14.1-3
\end{frame}

\section{Answers - Chapter 13 and Unit 4 Review}

\begin{frame}{Answers}
\small
\begin{columns}[T]
\begin{column}{0.5\textwidth}
\begin{itemize}
\item B
\item A
\end{itemize}
\end{column}
\begin{column}{0.5\textwidth}
\begin{itemize}
\item ...
\end{itemize}
\end{column}
\end{columns}
\end{frame}

\section{Answers - Chapter 14}

\begin{frame}{Answers}
\small
\begin{columns}[T]
\begin{column}{0.5\textwidth}
\begin{itemize}
\item ...
\end{itemize}
\end{column}
\begin{column}{0.5\textwidth}
\begin{itemize}
\item ...
\end{itemize}
\end{column}
\end{columns}
\end{frame}

\end{document}

\documentclass{beamer}
\usetheme{metropolis}
\usepackage{graphicx}
\usepackage{subfig}
\usepackage{tcolorbox}
\title{Calculus-Based Physics-2: Electricity, Magnetism, and Thermodynamics (PHYS180-02): Homework 1 Solutions}
\author{Jordan Hanson}
\institute{Whittier College Department of Physics and Astronomy}

\begin{document}
\maketitle

\section{Chapter 1}

\begin{frame}{Chapter 1: 52, 54, 65, 79, 89.}
\textbf{Exercise 52:} Use the change in length cause by linear thermal expansion.
\begin{align}
\Delta L &= L_0 \alpha \Delta T \\
\Delta L &= (12\times 10^{-6})(35)(10)~m \\
\Delta L &\approx 4 \times 10^{-3}~m = 4~mm
\end{align}
\end{frame}

\begin{frame}{Chapter 1: 52, 54, 65, 79, 89.}
\textbf{Exercise 54:} Use the change in length cause by volumetric thermal expansion (water is a liquid). In this case it's a volumetric expansion but only in the upward direction (rigid walls).
\begin{align}
\Delta L &= L_0 \beta \Delta T \\
\Delta L &= (210\times 10^{-6})(1)(1000)~m \\
\Delta L &\approx 21 ~ cm
\end{align}
\end{frame}

\begin{frame}{Chapter 1: 52, 54, 65, 79, 89.}
\textbf{Exercise 65:} Use the definition of specific heat capacity:
\begin{align}
Q &= mc \Delta T \\
\frac{Q}{m\Delta T} &= c \\
c &\approx 0.4 ~ \frac{J}{kg~C^{\circ}}
\end{align}
\end{frame}

\begin{frame}{Chapter 1: 52, 54, 65, 79, 89.}
\small
\textbf{Exercise 79:} Part (a), use $m=\rho V$.  Part (b), use $Q = mL_f$ (latent heat of fusion).  Part (c), convert power per meter squared to power by multiplying by iceberg surface area.  Then, multiply by 12 hours in seconds to get Joules absorbed per day.  Divide answer in part (b) by this to obtain days to melt iceberg.
\begin{align}
m &= \rho V \\
m &\approx 1.5\times 10^{15}~kg
\end{align}
Next,
\begin{align}
Q &= m L_f \\
Q &\approx 5\times 10^{20}~J
\end{align}
\end{frame}

\begin{frame}{Chapter 1: 52, 54, 65, 79, 89.}
\small
\textbf{Exercise 79:} Part (a), use $m=\rho V$.  Part (b), use $Q = mL_f$ (latent heat of fusion).  Part (c), convert power per meter squared ($I$) to power by multiplying by iceberg surface area ($xy$).  Then, multiply by 12 hours in seconds ($\Delta t$) to get Joules absorbed per day.  Divide answer in part (b) by this to obtain days to melt iceberg.
\begin{align}
Q &= m L_f \\
Q &\approx 5\times 10^{20}~J
\end{align}
\begin{equation}
N_{days} = \frac{Q_{total}}{Ixy\Delta t} \approx 50~years
\end{equation}
The surprising thing is that the iceberg is still floating out there somewhere.  The latent heat of fusion of ice is that large.
\end{frame}

\begin{frame}{Chapter 1: 52, 54, 65, 79, 89.}
\small
\textbf{Exercise 89:} Use the standard heat conduction equation and plug in the givens:
\begin{equation}
P = \frac{kA\Delta T}{\Delta x} \approx \frac{8.4\times 10^{-2}(1.2)\times 10^2(1.3)(10)}{1.3\times 10^{-1}} ~ W \approx 1~kW
\end{equation}
So I think this is about 1 kiloWatt heater.  (If you said more than one because it's slightly higher than 1 kW I give you the point).
\end{frame}

\section{Chapter 2}

\begin{frame}{Chapter 2: 9, 21, 29, 36, 48, 59}
\small
\textbf{Exercise 9:} Use the fact that the energy per molecule is $\frac{3}{2}k_B T$, and equate this with $\frac{1}{2}mv_{rms}^2$.  This is like saying that the average kinetic energy is equal to the average thermal energy times the number of degrees of freedom (3).  We get
\begin{equation}
v_{rms} = \sqrt{\frac{3k_B T}{m}}
\end{equation}
The smaller the mass, the higher the average speed.  Planets with larger mass might contain these molecules with gravity, but the Earth cannot.
\end{frame}

\begin{frame}{Chapter 2: 9, 21, 29, 36, 48, 59}
\small
\textbf{Exercise 21:} This type of problem requires division of the mass in grams by the atomic weight in grams per mole (grams/(grams per mole) = mole). To convert from moles to molecules, multiply by $N_A$. Answers:
\begin{enumerate}
\item 0.018 moles
\item 0.23 moles
\item $\approx 10^{22}$ and $\approx 10^{23}$
\end{enumerate}
\end{frame}

\begin{frame}{Chapter 2: 9, 21, 29, 36, 48, 59}
\small
\textbf{Exercise 29:} This problem is a two-step problem.  First, note that the volume and temperature of the bicycle tire remain constant.  The ideal gas law for the initial and final state gives
\begin{align}
p_iV &= n_i R T \\
p_fV &= n_f R T
\end{align}
Dividing these equations, we get
\begin{equation}
\frac{p_f}{p_i} = \frac{n_f}{n_i}
\end{equation}
Let's define $\Delta n$ such that $n_f = n_i -\Delta n$ (the number of moles decreases in the tire).  Now we can substitute:
\begin{equation}
p_f = p_i\left(1-\frac{\Delta n}{n_i}\right)
\end{equation}
All that remains is to compute $n_i$ and $\Delta n$.
\end{frame}

\begin{frame}{Chapter 2: 9, 21, 29, 36, 48, 59}
\small
\textbf{Exercise 29:} All that remains is to compute $n_i$ and $\Delta n$. We know $n_i$ from the initial state of the tire:
\begin{equation}
n_i = \frac{p_iV}{RT}
\end{equation}
We know $\Delta n$ because it has to be the atoms that come out of the tire and are measure to have volume $V_{out}=100$ cm$^3$ at $p_{atm} = 1$ atm:
\begin{equation}
\Delta n = \frac{p_{atm}V_{out}}{RT}
\end{equation}
So the ratio $\Delta n/n_i$ is just $\frac{p_{atm}V_{out}}{p_iV}$, so the final answer is
\begin{equation}
p_f = p_i\left(1-\frac{p_{atm}V_{out}}{p_iV}\right) = 7~atm \left(1-\frac{1~atm~100~cc}{7~atm~2000~cc}\right)
\end{equation}
The final pressure is about 99\% of its original value.
\end{frame}

\begin{frame}{Chapter 2: 9, 21, 29, 36, 48, 59}
\small
\textbf{Exercise 36:} For part (a) just assume that the average kinetic energy is equal to $\frac{1}{2}m\bar{v^2}$, and that the velocities are uniformly random $v^2 = 3 v_x^2$.  The ball is 42 grams. $\frac{1}{2}m\bar{v^2} = 2.3$ J, and we have $v_x^2 \approx 110/3$ m$^2$ s$^{-1} \approx 36.7$ m$^2$ s$^{-1}$. For part (b) using $pV = \frac{1}{3}N m \bar{v^2}$ with $N=1$ and $\bar{v^2}$ from part (a) yields a pressure of 1.5 Pa.  (c) Assumption number 2 is not followed in this problem, since the velocity has to be constantly regenerated with the raquet rather than assuming it is uniform.
\end{frame}

\begin{frame}{Chapter 2: 9, 21, 29, 36, 48, 59}
\small
\textbf{Exercise 48:} To give molecules the given energy $K$, assert that $K = \frac{3}{2}k_B T$, and solve for $T$.  The result is $T \approx 3 \times 10^9$ Kelvin.  Does this make sense?  That's hotter than the surface of the Sun, right?  Well, the \textit{surface} of the sun is $5\times 10^3$ Kelvin, but the \textit{core} where the fusion is must be much higher.  Also, in plasmas where researchers typically try to create fusion, the densities of molecules are so low that this is not a bizarrely high amount of \textit{overall} energy, but very high \textit{energy density.}
\end{frame}

\begin{frame}{Chapter 2: 9, 21, 29, 36, 48, 59}
\small
\textbf{Exercise 59:} Use $Q = n c_V \Delta T$, solving for $c_V$.  Compare to $c_V = \frac{d}{2}R$, and we find that $d = 6.02$.  So this molecule cannot be just monatomic, but polyatomic ($d>5$), or diatomic with only one of the vibrational degrees of freedom.  The point is, not monatomic.
\end{frame}

\end{document}

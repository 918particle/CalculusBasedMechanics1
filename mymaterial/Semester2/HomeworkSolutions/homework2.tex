\documentclass{beamer}
\usetheme{metropolis}
\usepackage{graphicx}
\usepackage{subfig}
\usepackage{tcolorbox}
\title{Calculus-Based Physics-2: Electricity, Magnetism, and Thermodynamics (PHYS180-02): Homework 2 Solutions}
\author{Jordan Hanson}
\institute{Whittier College Department of Physics and Astronomy}

\begin{document}
\maketitle

\section{Chapter 3}

\begin{frame}{Chapter 3: C1, Ex. 20, 22, 25, 27, 28, 29, 38, 39, 40}
\textbf{C1:} (a) Opening a carbonated beverage represents work being done by the system on the environment, if the system is the can. (b) If the system is the tire, then filling it represents the environment doing work on the system. (c) If the system is the gas can, then this scenario is the environment doing work ont he system.
\end{frame}

\begin{frame}{Chapter 3: C1, Ex. 20, 22, 25, 27, 28, 29, 38, 39, 40}
\textbf{Exercise 20:} Integrating the isobaric case yields $V = \frac{R}{p}T+C_1$.  Separating the isochoric case into $\frac{dp}{p} = \frac{dT}{T}$ yields $p = C_2 T$.  Combining results, we can show that $pV = RT(1+C_1 C_2 / R)$.  Letting $b = (1+C_1 C_2 / R)$, we find that $pV = bRT$.  Also, assuming the ideal gas law and then checking the two derivative relations also works.
\end{frame}

\begin{frame}{Chapter 3: C1, Ex. 20, 22, 25, 27, 28, 29, 38, 39, 40}
\textbf{Exercise 22:} Isobaric case: $W = p\Delta V = 2.00 \times 2.00$ liter-atmospheres.  This is 200 J.
\end{frame}

\begin{frame}{Chapter 3: C1, Ex. 20, 22, 25, 27, 28, 29, 38, 39, 40}
\textbf{Exercise 25:} $W = p \Delta V$, so $\Delta V = W/p$.  The fractional volume increase is $\Delta V/V_i$.  Thus, $\Delta V/V_i = W/p/V_i = 500/0.8/20.0$ J/liter/atmosphere.  Converting units, we find $\Delta V/V_i = 0.31$, or a 31 percent increase in volume.
\end{frame}

\begin{frame}{Chapter 3: C1, Ex. 20, 22, 25, 27, 28, 29, 38, 39, 40}
\textbf{Exercise 27:} Isothermal case $W = nRT\ln(V_f/V_i)$. But the idea gas law says $nRT = pV$, so $W = pV\ln(V_f/V_i)$.  We also know that $V_f = 4 V_i$, so $W = pV \ln(4)$.
\end{frame}

\begin{frame}{Chapter 3: C1, Ex. 20, 22, 25, 27, 28, 29, 38, 39, 40}
\textbf{Exercise 28:} AB: 2 liter atmospheres. ADB: 4 liter atmospheres.  ACB: 4 liter atmospheres.  ADCB: 6 liter atmospheres
\end{frame}

\begin{frame}{Chapter 3: C1, Ex. 20, 22, 25, 27, 28, 29, 38, 39, 40}
\textbf{Exercise 29:} (a) the enclosed area is $\pi r^2/2$, and the radius (in x-direction) is 1 liter.  The area has to have units of liter atmospheres.  Thus the area is $\pi/2$ liter atmospheres or 157 Joules.  (b) Reversing direction means the work done is -157 Joules by Newton's 3rd law.
\end{frame}

\begin{frame}{Chapter 3: C1, Ex. 20, 22, 25, 27, 28, 29, 38, 39, 40}
\textbf{Exercise 38:} (a) If the temperature remains constant, then the internal energy change is 0 J. (b) $\Delta E = 0$, so $Q = W$ and $Q = 250$ J.
\end{frame}

\begin{frame}{Chapter 3: C1, Ex. 20, 22, 25, 27, 28, 29, 38, 39, 40}
\textbf{Exercise 39:} $\Delta E = Q - W$, so 80 J = Q - 500 J.  Thus, $Q = 580$ J.
\end{frame}


\begin{frame}{Chapter 3: C1, Ex. 20, 22, 25, 27, 28, 29, 38, 39, 40}
\textbf{Exercise 40:} Similar to the earlier problem, $W = pV \ln(4)$.  However, now we need to recognize that $\Delta E = 0$, so $Q = W$.  $Q = pV\ln(4)$.
\end{frame}

\end{document}

\title{Study Guide for Midterm 1 for Calculus-Based Physics: Electricity, Magnetism, and Thermodynamics}
\author{Dr. Jordan Hanson - Whittier College Dept. of Physics and Astronomy}
\date{\today}
\documentclass[10pt]{article}
\usepackage[a4paper, total={18cm, 27cm}]{geometry}
\usepackage{outlines}
\usepackage[sfdefault]{FiraSans}
\usepackage{hyperref}
\begin{document}
\maketitle

\begin{enumerate}
\item \textbf{Thermal expansion of materials.}  Recall that the \textit{thermal expansion coefficient}, $\alpha$, relates the linear expansion of an object of length $L_0$ to the change in length $\Delta L$ and the change in temperature $\Delta T$ by $\Delta L = \alpha L_0 \Delta T$.
\begin{enumerate}
\item The coefficient values for aluminum, brass, and copper are $25 \times 10^{-6}$ 1/$^{\circ}$C, $19 \times 10^{-6}$ 1/$^{\circ}$C, and $17 \times 10^{-6}$ 1/$^{\circ}$C, respectively.  What will be the fractional difference in length changes for brass and copper relative to aluminum for the same temperature change? \\ \vspace{2cm}
\item If the temperature of the needle is changed by 1000 degrees Celsius, and it happens to be made of brass, and it is originally 200 cm long, what will be the change in the length due to thermal expansion? \\ \vspace{2cm}
\item Recall that the volumetric expansion coefficient of water is $210 \times 10^{-6}$ 1/$^{\circ}$C.  What would be the change in sea level if the sea is warmed by 1 degrees Celsius, if the depth of the sea is on average 5 km? \\ \vspace{2cm}
\end{enumerate}
\item \textbf{Specific Heat Capacity and Latent Heat of Fusion.}  Recall that the specific heat capacity $c$ relates the heat $Q$ required to warm a solid substance with mass $m$ by a temperature change $\Delta T$ like $Q = m c \Delta T$.  Recall also that the latent heat of fusion $L_{\rm f}$ required to melt a substance of mass $m$ by introducing a total heat $Q$ is $Q = m L_{\rm f}$.
\begin{enumerate}
\item The specific heat capacity of a substance is 0.5 \textit{calories}/gram/degree Celsius, and the latent heat is 40 \textit{calories}/gram.  Suppose we have a 50 gram block of this substance.  What heat is required to warm it from 300 K to 400 K, when it begins to melt?  How much heat is required to melt it? \\ \vspace{2cm}
\end{enumerate}
\item \textbf{Error analysis.}  Suppose two temperatures are measured to be $T_{\rm 1} = 100 \pm 5$ degrees Celsius, and $T_{\rm 2} = 125 \pm 15$ degrees Celsius (accounting for random statistical errors).  What is the temperature ratio $\Delta T = T_{\rm 2}/T_{\rm 1}$, accounting for random statistical errors? \\ \vspace{2cm}
\item \textbf{Kinetic Theory of Gases.} Recall that the internal energy of an ideal gas is $E_{\rm int} = \frac{3}{2}n R T$, and $R = 8.31$ J/mol/K.  Recall also that the \textit{ideal gas law} states that $pV = n R T$, where $p$ is the pressure in Pascals, $V$ is the volume in m$^3$, $T$ is the temperature in degrees Kelvin, and $n$ is the number of moles.
\begin{enumerate}
\item If a container with an ideal gas inside has a volume of 1000.0 L, a temperature of 200 K, and a pressure of 10.0 atm ($10^5$ Pascals), how many moles are inside? \\ \vspace{2cm}
\item Recall that the \textit{specific heat at constant volume} of an ideal gas, $c_{\rm V}$ is related to the heat added $Q$, the number of moles $n$, and the corresponding temperature rise $\Delta T$ by $Q = n c_{\rm V} \Delta T$.  How much heat will be required to raise the temperature of the container by 1 degree? \\ \vspace{2cm}
\end{enumerate}
\item \textbf{The First Law of Thermodynamics, and pV phase-space diagrams.}  Recall that an \textit{isothermic process} is a quasi-static process in phase-space, and that on a $pV$ diagram, $p \propto V^{-1}$ for an ideal gas.  Recall also the First Law of Thermodynamics: $\Delta E_{\rm int} = Q - W$.
\begin{enumerate}
\item Draw a $pV$ phase-space diagram, labeling the axes with volume units of liters and pressure units of atmospheres.  Add to it an isothermic process that begins from (8.0 L, 3.0 atm) and ends at (10.0 L, 2.0 atm). \\ \vspace{3cm}
\item If the process involves 2 moles of ideal gas, to what temperature does this isothermic process correspond?  (\textit{Hint: use the ideal gas law}). \\ \vspace{2cm}
\item How much heat is required to perform this work, according to the First Law? Is the work positive or negative?  (\textit{Hint: recall the formula for the work done by an isothermic process $W = nRT\ln(V_f/V_i)$}). \\ \vspace{2cm}
\end{enumerate}
\item \textbf{Heat Capacities of an Ideal Gas.} Recall that $c_{\rm V} = \frac{d}{2}R$ for an ideal gas with $d$ degrees of freedom on the molecular level, and that $c_{\rm P}$, the heat capacity of a gas at constant pressure is $c_{\rm P} = c_{\rm V} + R$.  How many degrees of freedom correspond to monatomic, diatomic, and polyatomic ideal gases?  Describe an experiment to measure $d$ for an ideal gas. \\ \vspace{3cm}
\item \textbf{The Second Law of Thermodynamics, and the Carnot Cycle.}  Recall that the efficiency of the Carnot cycle is $e = W/Q_{\rm h}$, where $W$ is the work performed and $Q_{\rm h}$ is the heat required.  $Q_{\rm c}$ is the heat required to return the engine to the original state (exhaust heat).  It may be shown that $e = 1 - \frac{T_c}{T_h}$, where $T_h$ and $T_c$ are the temperatures of $Q_h$ and $Q_c$.
\begin{enumerate}
\item What is the efficiency of an engine that operates with $T_h = 1000$ K and $T_c = 300$ K? \\ \vspace{2cm}
\item What is the work output if the heat input is 30 kJ? \\ \vspace{2cm}
\item What is the change in \textit{entropy} for the cycle? \\ \vspace{2cm}
\item Draw a diagram of this engine's 4 proceses in $pV$ phase-space.  Label all axes, temperatures, and heats, and indicate which processes are adiabatic and which are isothermal. \\ \vspace{4cm}
\item What are the entropies of the two isothermal processes? (Recall that entropy is $S = \frac{Q}{T}$ for an isothermal process).
\end{enumerate}
\end{enumerate}
\end{document}
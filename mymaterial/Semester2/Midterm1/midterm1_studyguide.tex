\title{Study Guide for Midterm 1 for Calculus-Based Physics: Electricity and Magnetism}
\author{Dr. Jordan Hanson - Whittier College Dept. of Physics and Astronomy}
\date{\today}
\documentclass[10pt]{article}
\usepackage[a4paper, total={18cm, 27cm}]{geometry}
\usepackage{outlines}
\usepackage[sfdefault]{FiraSans}
\usepackage{hyperref}
\usepackage{graphicx}
\begin{document}
\maketitle

\textbf{Instructions:} Work each problem before looking at the given answer.  See if you first understand the problem \textit{conceptually}, then work out the mathematics, then end with plugging in relevant data. \\ \vspace{0.25cm}

\textbf{Memory Bank:}
\begin{enumerate}
\item Coulomb Force: $\vec{F} = k \frac{q_1 q_2}{r^2}\hat{r}$
\item $k = 9 \times 10^{9}$ N C$^{-2}$ m$^{2}$
\item $q_e = 1.6 \times 10^{-19}$ C
\item Mass of a proton: $1.67 \times 10^{-27}$ kg
\item Electric field and charge: $\vec{F} = q \vec{E}$
\item Field of infinite wire of charge density $\lambda$: $\vec{E}(z) = \frac{2k\lambda}{z}\hat{z}$
\item Field of two oppositely charged infinite planes, with charge density $\sigma$: $\vec{E}(z) = \frac{\sigma}{\epsilon_0}\hat{z}$
\item $\epsilon_0 \approx 8.85 \times 10^{-12}$ F/m
\item Dipole moment: $\vec{p} = q \vec{d}$
\item Torque on dipole moment: $\vec{\tau} = \vec{p} \times \vec{E}$
\item Electric flux: $\Phi = \vec{E} \cdot \vec{A} = EA \cos\theta$
\item Gauss' law: $\Phi = Q_{enc}/\epsilon_0$
\item Potential energy and voltage: $U = q\Delta V$
\item Voltage of a point charge: $V(r) = k\frac{q}{r}$
\item Voltage and E-field: $\vec{E} = -\nabla V$, single-variable $\vec{E} = -\frac{dV}{dx}$
\item Constant E-field: $E = \frac{\Delta V}{\Delta x}$
\item E-field and voltage: $\Delta V = -\int \vec{E} \cdot d\vec{x}$
\item Capacitance: $Q = CV$
\item Parallel plate capacitor: $C = \frac{\epsilon_0 A}{d}$
\item Adding two capacitors in series: $C_{tot}^{-1} = C_1^{-1} + C_2^{-2}$
\item Adding two capacitors in parallel: $C_{tot} = C_1 + C_2$
\item Definition of current: $I(t) = \frac{dQ}{dt}$
\item Drift velocity: $v_d = \frac{I}{nAq}$
\item Ohm's law: $V = IR$
\item \textbf{Adding two resistors in series} $R_{tot} = R_1 + R_2$
\item \textbf{Adding two resistors in parallel} $R_{tot}^{-1} = R_1^{-1} + R_2^{-2}$
\end{enumerate}

\clearpage

\begin{enumerate}
\item \textbf{Chapter 5, Electrostatics}
\begin{enumerate}
\item Protons in an atomic nucleus are typically $10^{-15}$ m apart. What is the electric force of repulsion between nuclear protons? \\ \\\
Using $F = \frac{k q_e^2}{r^2}$, we find $F \approx 230$ N. \\
\item A charge $q_1 = 20 \mu$C and a charge $q_2 = 10 \mu$C are 1.0 m apart.  What is the force on a positive test charge halfway between them, and in which direction is the force? \\ \\
The electric field points away from $q_1$ and has a magnitude of $3.6\times 10^5$ N/C.  Suppose the test charge has a charge $q$.  The force is $F = qE$. \\
\item Suppose the ``deflector'' in Fig. \ref{fig:e1} is $d=12$ cm long.  If a proton (mass given in Memory Bank) has an initial speed of $v=1.5 \times 10^{7}$ m/s, and the field depicted is $4.0 \times 10^5$ N/C, by how much has it been deflected?  (What is $d$?). \\ \\
Using kinematics and the fact that $F=qE$, $d = \frac{1}{2} \frac{qE}{m}\left(\frac{\Delta x}{v}\right)^2 \approx 1.15$ mm.
\begin{figure}
\centering
\includegraphics[width=0.45\textwidth]{figures/e1.png}
\caption{\label{fig:e1} A constant E-field deflecting a positive charge $q$.}
\end{figure}
\end{enumerate}
\item \textbf{Chapter 6, Gauss' Law}
\begin{enumerate}
\item Show that the field a distance $z$ above an infinite line of charge with charge density $\lambda$ (C/m) is $\vec{E}(z) = \frac{2k\lambda}{z}\hat{z}$.  Use a Gaussian surface that has \textit{cylindrical symmetry.} \\ \\
See Figure 6.29 of the text. \\
\end{enumerate}
\item \textbf{Chapter 7, Voltage}
\begin{enumerate}
\item A lightning bolt strikes a tree, moving 20.0 C of charge through a potential difference of $10^8$ Volts. What energy was dissipated? \\ \\
$U = 2 \times 10^{9}$ J, or 2 GJ. \\
\item Consult again Fig. \ref{fig:e1}.  If the plates are 6 cm apart, and the field is still $4.0 \times 10^5$ N/C, what is the voltage difference between the plates? \\ \\
Using $E = \Delta V/\Delta x \rightarrow V = E\Delta x$, we find 24 kV. \\
\end{enumerate}
\item \textbf{Chapter 8, Capacitance}
\begin{enumerate}
\item Find the charge stored when 5.0 V is applied to an 8.00 pF capacitor. \\ \\
Using $q = CV$, we find 40 pC. \\
\item Find the charge stored when 5.0 V is applied to two 8.00 pF capacitors \textit{in parallel}. \\ \\
80 pC, from scaling. \\
\item Find the charge stored when 5.0 V is applied to two 8.00 pF capacitors \textit{in series}. \\ \\
20 pC, from scaling. \\
\item Find the total capacitance in the circuit diagram of Fig. \ref{fig:c1}. \\ \\
The two capacitors on the left side are added in series, and that is added in parallel with the right-hand capacitor.  The result is $2.8 \mu$F. \\
\begin{figure}
\centering
\includegraphics[width=0.4\textwidth]{figures/c1.png}
\caption{\label{fig:c1} Three capacitors connected together.}
\end{figure}
\end{enumerate}
\item \textbf{Chapter 9, Current and Ohm's law}
\begin{enumerate}
\item What current passes through a resistor with $R=1$ k$\Omega$, if the voltage applied is 12 V? \\ \\
Using Ohm's Law, we find 12 mA.
\\
\item What current passes through two resistors with $R=1$ k$\Omega$, if the voltage applied is 12 V, and the resistors are connected \textit{in series}? Draw a circuit diagram. \\ \\
6 mA, from scaling, and the fact that the total resistance has doubled.
\\
\item What current passes through two resistors with $R=1$ k$\Omega$, if the voltage applied is 12 V, and the resistors are connected \textit{in parallel}? Draw a circuit diagram. \\ \\
24 mA, from scaling, and the fact that the total resistance has been halved relative to the first question.
\end{enumerate}
\end{enumerate}
\end{document}
\title{Midterm 1 for Calculus-Based Physics: Electricity and Magnetism}
\author{Dr. Jordan Hanson - Whittier College Dept. of Physics and Astronomy}
\date{\today}
\documentclass[10pt]{article}
\usepackage[a4paper, total={18cm, 27cm}]{geometry}
\usepackage{outlines}
\usepackage[sfdefault]{FiraSans}
\usepackage{hyperref}
\usepackage{graphicx}
\begin{document}
\maketitle

\textbf{Memory Bank:}
\begin{enumerate}
\item Coulomb Force: $\vec{F} = k \frac{q_1 q_2}{r^2}\hat{r}$
\item $k = 9 \times 10^{9}$ N C$^{-2}$ m$^{2}$
\item $q_e = 1.6 \times 10^{-19}$ C
\item Mass of a proton: $1.67 \times 10^{-27}$ kg
\item Electric field and charge: $\vec{F} = q \vec{E}$
\item Field of infinite wire of charge density $\lambda$: $\vec{E}(z) = \frac{2k\lambda}{z}\hat{z}$
\item Field of two oppositely charged infinite planes, with charge density $\sigma$: $\vec{E}(z) = \frac{\sigma}{\epsilon_0}\hat{z}$
\item $\epsilon_0 \approx 8.85 \times 10^{-12}$ F/m
\item Dipole moment: $\vec{p} = q \vec{d}$
\item Torque on dipole moment: $\vec{\tau} = \vec{p} \times \vec{E}$
\item Electric flux: $\Phi = \vec{E} \cdot \vec{A} = EA \cos\theta$
\item Gauss' law: $\Phi = Q_{enc}/\epsilon_0$
\item Potential energy and voltage: $U = q\Delta V$
\item Voltage of a point charge: $V(r) = k\frac{q}{r}$
\item Voltage and E-field: $\vec{E} = -\nabla V$, single-variable $\vec{E} = -\frac{dV}{dx}$
\item Constant E-field: $E = \frac{\Delta V}{\Delta x}$
\item E-field and voltage: $\Delta V = -\int \vec{E} \cdot d\vec{x}$
\item Capacitance: $Q = CV$
\item Parallel plate capacitor: $C = \frac{\epsilon_0 A}{d}$
\item Adding two capacitors in series: $C_{tot}^{-1} = C_1^{-1} + C_2^{-1}$
\item Adding two capacitors in parallel: $C_{tot} = C_1 + C_2$
\item Definition of current: $I(t) = \frac{dQ}{dt}$
\item Drift velocity: $v_d = \frac{I}{nAq}$
\item Ohm's law: $V = IR$
\item Power: $P = IV$
\item \textbf{Adding two resistors in series} $R_{tot} = R_1 + R_2$
\item \textbf{Adding two resistors in parallel} $R_{tot}^{-1} = R_1^{-1} + R_2^{-1}$
\end{enumerate}

\clearpage

\begin{enumerate}
\item \textbf{Chapter 5, Electrostatics}
\begin{enumerate}
\item Two electrons approach each other in space.  What is the electric force of repulsion between them when they are separated by $10^{-13}$ m? \\ \vspace{1cm}
\item A charge $q_1$ is located at (-2,-2) m in a 2D coordinate system, a charge $q_2$ is located at (-2,2) m, a charge $q_3$ is located at (2,2) m, and a charge $q_4$ is located at (2,-2) m.  If $q_1 = q_2 = 20\mu$ C, and $q_3 = q_4 = 40\mu$ C, find the value of the electric field at (0,0). \\ \vspace{2cm}
\item 
\begin{figure}
\centering
\includegraphics[width=0.2\textwidth]{figures/cap.png}
\caption{\label{fig:cap} A device accelerating a \textbf{positively charged} particle to the right.}
\end{figure}
In Fig. \ref{fig:cap}, assume a proton is being accelerated to the right.  (a) If the electric field is $E = 2000$ N/C to the right, what is the force on the proton?  (b) Using Newton's Second Law, show that the acceleration is $a = (q/m) E$.  (c)  Recall that an object that is accelerating travels a distance $d$ in a time $t$ according to $d = \frac{1}{2}at^2$.  How far has the proton travelled in 1 $\mu$s? \\ \vspace{3cm}
\end{enumerate}
\item \textbf{Chapter 6, Gauss' Law}
\begin{enumerate}
\item Show that the electric field of an infinite line of charge with charge per unit length $\lambda$ is $E = \lambda/(2\pi \epsilon_0 r)$, if the test charge is a distance $r$ from the line. \\ \vspace{3cm}
\end{enumerate}
\item \textbf{Chapter 7, Voltage}
\begin{enumerate}
\item An arch of electricity sends 10.0 C of charge through a potential of $10^4$ Volts, and a second arch sends 1.0 C of charge through a potential of $10^5$ Volts. What total energy was dissipated? (Add the two energies). \\ \vspace{2cm}
\item Consult again Fig. \ref{fig:cap}.  (a) If the plates are 150 cm apart, and the field is still $2000$ N/C, what is the voltage difference between the plates? (b) Draw $V(x)$, the voltage as a function of distance $x$ between the plates. \\ \vspace{2cm}
\item If a charge $q_1 = 10$ nC is fixed in place, and a charge of $q_2 = 0.1$ nC is placed a distance of 1 m from it, (a) what is the potential energy of $q_2$? (b) If $q_2$ has a mass of $10^{-6}$ kg, what speed will it reach if it is released? \\ \vspace{2cm}
\end{enumerate}
\item \textbf{Chapter 8, Capacitance}
\begin{enumerate}
\item Find the (a) charge stored when 5.0 V is applied to an 50.0 pF capacitor. (b) What is the energy stored? \\ \vspace{1cm}
\item Find the charge stored when 5.0 V is applied to two 50.0 pF capacitors \textit{in parallel}. \\ \vspace{1cm}
\item Consult Fig. \ref{fig:cap2}.  If $C_1 = C_2 = C_3 = 0.1$ $\mu$ F, and $V=12.0$ Volts, what is the total charge stored? \\ \vspace{1.5cm}
\begin{figure}
\centering
\includegraphics[width=0.2\textwidth]{figures/cap2.png}
\caption{\label{fig:cap2} A circuit of capacitors.}
\end{figure}
\end{enumerate}
\item \textbf{Chapter 9, Current and Ohm's law}
\begin{enumerate}
\item Suppose the charge collected in a capacitor is measured to follow the function $Q(t) = - Q_0 \exp(-t/\tau)$.  (a) Using the definition of current, show that the current into the capacitor is $i(t) = (Q_0/\tau) \exp(-t/\tau)$.  (b) If $\tau = RC$, with $R = 1000\Omega$, and $C = 1.0\mu$F, and $Q_0 = 1\mu$ C, what is the current when $t = 5$ ms? \\ \vspace{2.5cm}
\item Suppose two resistors $R_1$ and $R_2$ are both $1000 \Omega$, and are connected in parallel to a 5V battery.  (a) What is the total current flowing from the battery? (b) What is the power being provided by the battery?
\end{enumerate}
\end{enumerate}
\end{document}
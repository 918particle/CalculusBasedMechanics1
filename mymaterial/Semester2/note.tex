\title{Guidelines for Scientific Article Discussions}
\author{
        Jordan Hanson, Department of Physics and Astronomy \\
        Whittier College	
}
\date{\today}

\documentclass[12pt]{article}
\usepackage{amsmath}

\begin{document}
\maketitle

\section{Purpose}

The purpose of this exercise is to \textbf{replace the third midterm with an interactive journal article analysis and presentation}.  Similar activities are common at research universities where students are expected to explain a piece of science to a group and answer challenging questions.  The goal is to evoke quantitative thought about a scientific result we don't fully understand.

\section{Format}

Each student will choose a scientific article from a peer-reviewed journal, and study it.  The student will stand in front of the class and present the article, explaining how the idea works, how the data was collected, and what results were derived from the data.  The audience is required to ask questions so that everyone better understands the article content.  I will also ask a few questions, but \textit{students will earn part of their grade by participating in the discussion.}  I will keep track of who is participating and how well the presenter understands the material.  Bonus points will be awarded for teaching the class something brand new regarding the concepts we covered in class.

\section{Grade summary}

\begin{enumerate}
\item 50\% - Understanding of the article content
\begin{itemize}
\item \textit{When explaining the research to the class, does the student present a coherent picture of the research?}
\item \textit{Does the student understand the figures and tables in the paper?}
\item \textit{Can the student trace the idea from abstract to conclusion, explaining why each part of the paper matters?}
\item \textit{Can the student highlight why or why not the data support the conclusion?}
\end{itemize}
\item 25\% - Ability to answer questions, and explain the research to the class
\begin{itemize}
\item \textit{Is the student able to explain the article in a way that people understand it?}
\item \textit{Is the student able to draw diagrams or point the person asking the question to a graph or table in the paper?}
\end{itemize}
\item 25\% - Ability to participate in the discussion by asking thoughtful questions
\begin{itemize}
\item \textit{Does the student participate in the discussions?}
\item \textit{Are the student's questions relevant to the paper?}
\end{itemize}
\end{enumerate}

\section{Locating a Good Article}

The following strategies are good ways to find original peer-reviewed journal articles:

\begin{enumerate}
\item Start with Google Scholar search - filter based on research area in which you are interested
\item INSPIRES database, or NASA abstract search
\item arXiv.org - Physics, math, and astronomy pre-print server
\item Click on a popular science article, and keep clicking on links until you get to the \textit{original study.}
\end{enumerate}

\end{document}

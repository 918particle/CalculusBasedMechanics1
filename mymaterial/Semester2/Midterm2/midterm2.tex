\title{Midterm 2 for Calculus-Based Physics: Electricity, Magnetism, and Thermodynamics}
\author{Dr. Jordan Hanson - Whittier College Dept. of Physics and Astronomy}
\date{\today}
\documentclass[10pt]{article}
\usepackage[a4paper, total={18cm, 27cm}]{geometry}
\usepackage{outlines}
\usepackage[sfdefault]{FiraSans}
\usepackage{hyperref}
\begin{document}
\maketitle

\begin{enumerate}
\item \textbf{Electrostatics}
\begin{enumerate}
\item \textbf{Conceptual, electrostatics}: Two lines of charge run parallel to the z-axis in an $xyz$ coordinate system.  (a) Where in the coordinate system is the electric field zero?  (b) What would happen to a positive charge released where the electric field is zero?  (c) What would happen to a negative charge? \\ \vspace{2.5cm}
\item \textbf{Conceptual, electrostatics:} Three positive charges $q$ lie on the x-axis, equally spaced. (a) Where in three-dimensional space is the electric field zero? \\ \vspace{2.5cm} 
\item \textbf{Dipole moments:} A charge $+q$ is located at $P_1 = (0,0,a)$, and a charge $-q$ is located at $P_2 = (0,0,-a)$.  (a) What is the vector dipole moment, if $q = 0.5$ nC, and $a = 0.5$ nm?  (b) What is the torque on this dipole if the external E-field is $\vec{E} = 10^3 \hat{x}$ N/C? (Recall that the formula for the dipole moment is $\vec{p} = q \vec{d}$, and that the torque is $\vec{\tau} = \vec{p} \times \vec{E}$).  \\ \vspace{2.5cm}
\item \textbf{Dipole moments, charged plates:} A charge $+q$ is located at $P_1 = (0,0,a)$, and a charge $-q$ is located at $P_2 = (0,0,-a)$.  At $x=-b$ and at $x=b$, there are large charged plates parallel to the z-axis and y-axis, with charge per unit area $\sigma$.  The charge on the plate at $x=-b$ is negative, and the charge on the plate at $x=b$ is positive.  (a) In terms of the given variables, what is the torque on the dipole?  (Recall that the E-field between such plates is $|\vec{E}| = \sigma/\epsilon_0$).  (b) What is the torque on the dipole if both plates are the same charge? \\ \vspace{3cm}
\end{enumerate}
\clearpage
\item \textbf{Gauss' Law}
\begin{enumerate}
\item Using Gauss' Law, prove that the Coulomb field around a single point charge $q$ is $\vec{E} = \frac{k q}{r^2}$.  What is the field strength of a charge of 1 $\mu C$, 1 $\mu$m away from the charge? \\ \vspace{3cm}
\item Using Gauss' Law, prove that the Coulomb field around a line of charge with charge per unit length $\lambda$ is $\vec{E} = \frac{2k\lambda}{r} \hat{r}$.  What is the field if $r = 1$ cm, and $\lambda = 1$ nC/cm? \\ \vspace{3cm}
\end{enumerate}
\item \textbf{Electric potential}
\begin{enumerate}
\item Suppose two charged plates are 0.1 mm apart.  One has 1 nC of charge, and the other has -1 nC of charge.  The plates are squares with side length of 10 cm.  (a) What is the surface charge density of each plate?  (b) What is the electric field between the plates?  (c) As accurately as possible, and labelling all axes, draw the voltage as a function of position between the plates. \\ \vspace{4cm}
\item Suppose a charge distribution creates a voltage described by $V(x,y,z) = (x_0 - a x^2) + (x_0 - b y^2) + (x_0 - c z^2)$.  Derive the $\vec{E}$-field.  (Recall that $\vec{E} = -\nabla V$).\\ \vspace{3cm}
\item Suppose an electric field is $\vec{E} = -\frac{K}{x} \hat{x}$.  What is the voltage difference between $P = (C_1,0,0)$ and $P = (C_2,0,0)$? (Recall that the potential difference is given by $\Delta V = - \int \vec{E} \cdot \vec{dx}$).
\end{enumerate}
\end{enumerate}
\end{document}
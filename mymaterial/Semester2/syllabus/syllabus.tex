\title{Syllabus for Calculus-Based Physics: Electricity, Magnetism, and Thermodynamics (PHYS180-02)}
\author{Dr. Jordan Hanson - Whittier College Dept. of Physics and Astronomy}
\date{\today}
\documentclass[10pt]{article}
\usepackage[a4paper, total={18cm, 27cm}]{geometry}
\usepackage{outlines}
\usepackage[sfdefault]{FiraSans}
\usepackage{hyperref}
\begin{document}
\maketitle

\begin{abstract}
The concepts of calculus-based electromagnetism will be presented within the context of interactive problem-solving.  The course will begin with the introduction electric charge, electrostatics, and electric potential.  The applications of DC circuits is then built, and followed by the addition of magnetism.  The course then proceeds to induction, and AC circuits, and concludes with geometric and wave optics.  The course work will include interactive computational exercises, analytic textbook problems, group-designed projects, and lab-based activities.
\end{abstract}
\noindent
\textit{\textbf{Pre-requisites}: PHYS 150 or PHYS 135A, and MATH 142 (may be concurrent).} \\
\textit{\textbf{Course credits, Liberal Arts Categorization}: 4 Credits, COM1} \\
\textit{\textbf{Regular course hours}: Monday, Wednesday and Friday from 15:00 - 16:30 in SLC 228} \\
\textit{\textbf{Instructor contact information}: jhanson2@whittier.edu, tel. 562.907.5130} \\
\textit{\textbf{Office hours}: Mondays from 8:00-12:00, SLC 212. Professor on Google Hangouts: \textbf{918particle}} \\
\textit{\textbf{Attendance/Absence}: Students needing to reschedule midterms and exams should notify the professor a reasonable time beforehand. Further attendance issues are left to the discretion of the instructor}.\\ 
\textit{\textbf{Late work policy}: Late work is generally not accepted, but is left to the discretion of the instructor.} \\
\textit{\textbf{Text}: University Physics Volume Two - \url{https://openstax.org/details/books/university-physics-volume-2}}.  Course begins with Unit 2 of the text.  Thermodynamics is covered in PHYS185: Calculus-based Physics III. \\
\textit{\textbf{Grading}: There will be three tests, each examining conceptual understanding in step-by-step problems. Each
midterm is worth 15\% of the final grade. The weekly online homework is worth 20\% of the grade. Interactive
in-class activities will be worth 10\% of the final grade. Lab groups will present results of a group project worth 10\% of the grade.  The final exam will be held on May 11th, 13:00-15:00, and will be worth 15\% of the grade.} \\
\textit{\textbf{Grade Settings}: $<60\%$ = F, $\geq 60\%, <70\%$ = D, $\geq 70\%, <80\%$ = C, $\geq 80\%, <90\%$ = B, $\geq 90\%, <100\%$ = A.  Pluses and minuses: 0-3\% minus, 3\%-6\% straight, 6\%-10\% plus (e.g. 79\% = C+, 91\% = A-)} \\
\textit{\textbf{Homework Sets}: Typically 5-10 problems per week, assigned and collected on Fridays.  See \url{http://goeta.link/USB06CA-9DE6A4-1ZT} for online homework setup.} \\
\textit{\textbf{Bonus Essay}: Students may submit an essay on the history of scientific developments covered in the course, due at
the end of the semester. The essay must be 10 pages, address scientific arguments and results, and must include
references. The grade of this paper will replace the lowest midterm grade.  Students wishing to submit must notify the professor no later than one week after the second midterm.} \\
\textit{\textbf{Friday Science Presentations:} Optionally, students may present a recent scientific article or publication to the class on Fridays for a bonus point on the most recent homework or midterm.  Limit one point per student per Friday.} \\
\textit{\textbf{ADA Statement on Disability Services}: The Americans with Disabilities Act (ADA) is a federal anti-discrimination statute that provides comprehensive civil rights protection for persons with disabilities. Among other things, this legislation requires that all students with disabilities be guaranteed a learning environment that provides for reasonable accommodation of their disabilities. If you believe you have a disability requiring an accommodation, please contact Disability Services: disabilityservices@whittier.edu, tel. 562.907.4825.} \\
\textit{\textbf{Academic Honesty Policy}: \url{http://www.whittier.edu/academics/academichonesty}} \\
\textit{\textbf{Changes due to COVID-19}: Our course is well-positioned for the online transition.  Several steps are necessary:+
\begin{enumerate}
\item Class will meet for \textbf{up to 90 minutes} using the Zoom web-conferencing application, at the normal class time.  Class time will be used to work a reading assessment, along with whiteboard solution via Zoom.  The reading assessments will now be sent out before class, and will still be based on the scheduled syllabus.
\item Homework assignments will continue as normal, no changes.  \textbf{Be sure to stay current, and contact the professor on Google Hangouts with homework questions.}
\item Online versions of laboratory activities will be moved to a service called \textbf{Pivot Interactives.} Further instructions will be sent via Moodle once we establish success with the Zoom web-conferences.
\item Group project results will be \textbf{presented via Zoom} during the last week of class.
\item The Friday Science Presentations will continue as normal.  Send the article via group chat in Zoom.
\item \textbf{Students may opt-out of taking the final exam.}  If a student chooses not to take the final exam, the grade will be calculated based on the other assignments with the same relative weighting.
\end{enumerate}}
\textit{\textbf{Course Objectives}:}
\begin{itemize}
\item To practice written and oral expression of scientifically technical ideas.
\item To solve word problems pertaining to physics and mathematics.
\item To construct mathematical models of electrical systems like DC circuits.
\item To apply logical thinking to conceptually-posed physics problems.
\item Logical thinking.
\item To practice scientific experimentation, data analysis, and reporting of results.
\end{itemize}
\textit{\textbf{Course Outline}:}
\begin{outline}[enumerate]
\1 \textbf{Unit 0}: Review of pre-requisite courses and introduction of electric charge.
\2 Unit analysis, kinematics and Newton's Laws
\2 Work and energy, momentum
\2 Electrostatics, I - \textbf{Chapters 5.1 - 5.4}
\3 Electric charge, the Coulomb force, and electric fields
\3 Monday reading quiz: chapter 5.3
\2 Electrostatics, II - \textbf{Chapters 5.5 - 5.7, 6.1 - 6.4}
\3 Electric fields and electric dipoles
\3 Wednesday reading quiz: chapter 5.7
\3 Gauss' Law
\1 \textbf{Unit 1}: Electric potential and capacitance
\2 Electric potential, voltage I - \textbf{Chapters 7.1 - 7.5}
\3 Electric potential energy, voltage
\3 Voltage to field, field to voltage
\3 Field lines and equipotential
\3 Monday reading quiz: chapter 6.3
\2 Capacitance - \textbf{Chapters 8.1 - 8.3}
\3 Capacitors and capacitance
\3 Series and parallel capacitors
\3 The concept of dielectric material
\3 Wednesday reading quiz: chapter 7.2
\1 \textbf{First midterm, end of Unit 1, February 28th, 2020.} The first midterm will be shorter than the second and third ones, focusing on the Coulomb force and electric fields, voltage, and capacitance.
\1 \textbf{Unit 2:} Current, resistance, and DC circuits
\2 Current and resistance - \textbf{Chapters 9.1 - 9.5}
\3 Current, charge, and conduction in metals
\3 Resistance and Ohm's law
\3 Monday reading quiz: chapter 9.3
\2 Voltage II and Electromotive Force (EMF) - \textbf{Chapters 10.1 - 10.3, 10.5}
\3 EMF
\3 Resistors in series and parallel, Kirchhoff's rules
\3 RC Circuits
\3 Wednesday reading quiz: chapter 10.3
\1 \textbf{Unit 3:} Magnetism I
\2 Magnetic fields, I - \textbf{Chapters 11.2 - 11.5}
\3 Force of a magnetic field on charge, conductors
\3 Sources of magnetic fields
\3 Magnetic torque
\3 Monday reading quiz: chapter 11.3
\2 Magnetic fields, II - \textbf{Chapters 11.6 - 11.7}
\3 The Hall effect
\3 The mass spectrometer and the cyclotron
\1 \textbf{Second midterm, end of Unit 3, March 27th, 2020.} The second midterm will cover DC circuits and Kirchhoff's rules, magnetic force on moving charge and current, and applications of magnetic torque like the mass spectrometer.
\1 \textbf{Unit 4:} Magnetism II
\2 Magnetic fields created by currents - \textbf{Chapters 12.1 - 12.4}
\3 The Biot-Savart Law and forces on wires
\3 The magnetic field of a current loop
\3 Monday reading quiz: chapter 12.2
\2 Amp\`{e}re's Law - \textbf{Chapters 12.5 - 12.7}
\3 Introduction to Amp\`{e}re's Law
\3 Solenoids, toroids, and magnetic fields in materials
\3 Wednesday reading quiz: chapter 12.5
\1 \textbf{Unit 5:} Field Induction and Inductance
\2 Faraday's Law and Lenz's Law - \textbf{Chapters 13.1 - 13.4, 13.6}
\3 Faraday's Law and Lenz's Law
\3 Motional EMF, induced field, and generators
\3 Monday reading quiz: chapter 13.1
\2 Inductance and Energy in Magnetic fields - \textbf{Chapters 14.1 - 14.3}
\3 Definition of mutual and self inductance
\3 Energy in a magnetic field
\3 Wednesday reading quiz: 14.2
\1 \textbf{Third midterm, end of Unit 5, April 24th, 2020.} The third midterm will focus on the Biot-Savart law and Amp\`{e}re's Law, Faraday's Law and induced fields, and induction and magnetic energy.
\1 \textbf{Unit 6:} Electromagnetic Waves
\2 Derivation of the speed of light and the wave equation - \textbf{Chapters 16.1-16.3}
\2 Monday reading quiz: 16.2
\1 \textbf{Unit 7:} Cumulative Review, group presentations, and final exam
\2 No reading quizzes
\2 Group presentations
\3 Worth 10\% of the grade
\3 Given as a group
\3 10-15 minute presentation with slides or board work
\3 Final exam reviews will be given the last week of class, or potentially faculty reading day
\3 The final exam will be held on May 11th, 13:00-15:00, and will be worth 15\% of the grade.
\end{outline}
\end{document}

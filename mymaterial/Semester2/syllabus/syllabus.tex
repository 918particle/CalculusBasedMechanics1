\title{Syllabus for Calculus-Based Physics-2: Electricity, Magnetism, and Thermodynamics (PHYS180-02)}
\author{Dr. Jordan Hanson - Whittier College Dept. of Physics and Astronomy}
\date{\today}
\documentclass[10pt]{article}
\usepackage[a4paper, total={18cm, 27cm}]{geometry}
\usepackage{outlines}
\usepackage[sfdefault]{FiraSans}
\usepackage{hyperref}
\begin{document}
\maketitle

\begin{abstract}
The concepts of thermodynamics and calculus-based electromagnetism will be presented within the context of interactive problem-solving.  The course will begin with the concepts of heat and temperature, and the laws of thermodynamics.  Next, the concepts of electric charge, electrostatics, electric potential and applications to DC circuits will be covered.  The course will proceed with the addition of magnetism, induction, and AC circuits, and conclude with geometric and wave optics.  As time permits, some selected topics from quantum mechanics and special relativity will be introduced.  The course work will include interactive computational exercises, analytic textbook problems, group-designed projects, and lab-based activities.
\end{abstract}
\noindent
\textit{\textbf{Pre-requisites}: PHYS-150 or PHYS-135A and MATH-141B (may be concurrent).} \\
\textit{\textbf{Course credits, Liberal Arts Categorization}: 5 Credits, COM1} \\
\textit{\textbf{Regular course hours}: Monday, Wednesday and Friday from 12:00 - 13:50 in SLC 228} \\
\textit{\textbf{Instructor contact information}: jhanson2@whittier.edu, tel. 562.907.5130} \\
\textit{\textbf{Office hours}: Monday 14:00-16:00 in SLC 212} \\
\textit{\textbf{Attendance/Absence}: Students needing to reschedule midterms and exams should notify the professor a reasonable time beforehand. Further attendance issues are left to the discretion of the instructor}.\\ 
\textit{\textbf{Late work policy}: Late work will not be accepted.} \\
\textit{\textbf{Text}: University Physics Volume One - https://openstax.org/details/books/university-physics-volume-2} \\
\textit{\textbf{Grading}: There will be three midterms, each worth 10\% of the final grade.  The weekly homework is worth 20\% of the grade.  Interactive in-class activities will be worth 15\% of the final grade.  Lab groups will present results of two group-designed projects worth 10\% of the grade each.  The final exam will be held on May 14th, 2018 from 8:00-10:00, and will be worth 15\% of the grade.} \\
\textit{\textbf{Grade Settings}: $<60\%$ = F, $>60\%,\leq 70\%$ = D, $>70\%,\leq80\%$ = C, $>80\%,\leq 90\%$ = B, $<90\%,\leq 100\%$ = A.  Pluses and minuses: 0-3\% minus, 3\%-6\% straight, 6\%-10\% plus (e.g. 79\% = C+, 91\% = A-)} \\
\textit{\textbf{Homework Sets}: Typically 10 problems per week, assigned and collected on Mondays.} \\
\textit{\textbf{Bonus Essay}: Students may submit an essay on the history of scientific developments covered in the course, due at the end of the semester.  The essay must address scientific arguments and results, must include library references, and must have at least 10 pages.  The grade of this paper will replace the lowest midterm score if submitted.} \\
\textit{\textbf{ADA Statement on Disability Services}: The Americans with Disabilities Act (ADA) is a federal anti-discrimination statute that provides comprehensive civil rights protection for persons with disabilities. Among other things, this legislation requires that all students with disabilities be guaranteed a learning environment that provides for reasonable accommodation of their disabilities. If you believe you have a disability requiring an accommodation, please contact Disability Services: disabilityservices@whittier.edu, tel. 562.907.4825.} \\
\textit{\textbf{Academic Honesty Policy}: \url{http://www.whittier.edu/academics/academichonesty}} \\
\textit{\textbf{Course Objectives}:}
\begin{itemize}
\item Written expression of quantitative and numerical ideas and arguments.
\item Oral expression of quantitative and numerical ideas and arguments.
\item Problem solving using numerical skills.
\item Mathematical modeling.
\item Logical thinking.
\item Analysis of data and results.
\end{itemize}
\clearpage
\small
\textit{\textbf{Course Outline}:}
\begin{outline}[enumerate]
\1 Unit 0: Review of pre-requisite course, 150
\2 Estimation, approximation, kinematics and Newton's Laws
\2 Work, energy and power
\2 Momentum, linear and angular
\1 Unit 1: Temperature, Heat and Thermodynamics 1 - \textbf{Chapters 1 and 2}
\2 Definitions of temperature, heat; measurements of each
\2 Thermal properties of matter
\2 Kinetic theory of gases
\1 Unit 2: The Laws of Thermodynamics - \textbf{Chapters 3 and 4}
\2 The First Law: Energy conservation 
\2 The Second Law: Entropy
\1 First midterm exam, end of Unit 2
\1 \textbf{Spring Break}: March 19th - March 23rd
\1 Unit 3: Electrostatics - \textbf{Chapters 5-7}
\2 The Coulomb Force, and Newton's Second Law for electric charges
\2 Force Fields, comparisons between Newtonian gravity and Coulomb Force
\2 Gauss' Law and symmetries
\2 Electric potential
\1 First In-Class Group Presentations, end of Unit 3
\1 Unit 4: Capacitance, Current and Resistors, and DC Circuits - \textbf{Chapters 8-10}
\2 Capacitance: stored charge and energy
\2 Dielectrics
\2 Ohm's law: current, resistance and resistors
\2 DC circuits: Voltage/EMF, circuit organization, and Kirchhoff's rules
\2 RC circuits
\1 Unit 5: Magnetism 1 and 2- \textbf{Chapters 11 and 12}
\2 Magnetic fields, forces and torques
\2 The Hall effect
\2 Applications of magnetic fields
\2 Biot-Savart Law, and various currents
\2 Amp\`{e}re's Law
\2 Magnetic fields in matter
\1 Second midterm exam, end of Unit 5
\1 Unit 6: Field Induction 1 and 2 - \textbf{Chapters 13 and 14}
\2 Faraday's Law and Lenz's Law
\2 Induced EMF
\2 Generators and other applications
\1 Unit 7: AC Circuits - \textbf{Chapter 15}
\2 Simple AC circuits and courses
\2 RL/RLC circuits
\2 Power and resonance, resonators and transformers
\1 \textit{\textbf{ Unit 8: Electromagnetic Waves and Maxwell's Equations}} - \textbf{Chapter 16}
\2 Maxwell's equations: predictions and observations unified
\2 Electromagnetic radiation, and the electromagnetic spectrum, energy in waves
\2 Fields and momentum
\2 Optics: rays, waves and other applications
\2 The full electromagnetic spectrum
\1 Third midterm exam, end of Unit 8
\1 \textit{Unit 9: Special Relativity and Quantum Mechanics - \textbf{(Time Permitting)}}
\2 \textit{Einstein's Postulates and modifications to kinematics and electromagnetism}
\2 \textit{Black-body (thermal) radiation, Planck's constant and radiation quanta}
\2 \textit{The photoelectric effect}
\2 \textit{Atomic and nuclear structure}
\1 Second In-Class Group Presentations, end of Unit 9
\1 Unit 10 - \textbf{Cumultive Review and Final Exam}
\end{outline}
\end{document}

\documentclass{beamer}
\usetheme{metropolis}
\usepackage{graphicx}
\usepackage{subfig}
\usepackage{hyperref}
\usepackage{tcolorbox}
\title{Calculus-Based Physics-1: Mechanics (PHYS150-01): Week 6}
\date{October 9th - October 13th, 2017}
\author{Jordan Hanson}
\institute{Whittier College Department of Physics and Astronomy}

\begin{document}
\maketitle

\section{Week 5 Review}

\begin{frame}{Week 5 Review}
\begin{enumerate}
\item \alert{Friction}
\begin{itemize}
\item Normal force and friction
\item Static, kinetic
\end{itemize}
\item \alert{Drag}
\begin{itemize}
\item Terminal velocity
\end{itemize}
\item \alert{Restoring Forces}
\begin{itemize}
\item Hooke's Law
\item Young's modulus
\item Shear modulus
\item Bulk modulus
\end{itemize}
\end{enumerate}
\end{frame}

\section{Week 5 Review Problem}

\begin{frame}{Week 5 Review Problem}
A car rests on four shock absorbers, and each is like a spring with a spring constant $k = 1000 N/cm$.  The car weighs 10000 N.  By what distance is each spring compressed?
\begin{itemize}
\item A: 2.5 cm
\item B: 10 cm
\item C: 1 meter
\item D: 0 cm
\end{itemize}
\end{frame}

\section{Week 6 Summary}

\begin{frame}{Week 6 Summary}
\begin{enumerate}
\item \alert{Work} has a scientifically precise definition
\begin{itemize}
\item Units
\item As a product of force and displacement vectors
\end{itemize}
\item Kinetic Energy and the \alert{Work-Energy Theorem}
\item Gravitational potential energy
\begin{itemize}
\item Potential energy
\item \textit{Simplifying otherwise complex calculations}
\item Potential energy near Earth's surface
\item ...in space
\end{itemize}
\item Definition of a \textbf{conservative force}
\begin{itemize}
\item Relationship between conservative forces and potential energy
\item Conservation of energy for conservative forces
\end{itemize}
\end{enumerate}
\end{frame}

\section{Definitions of Work}

\begin{frame}{Definitions of Work}
\begin{tcolorbox}[colback=white,colframe=red!40!blue,title=Physical Definition of Work]
\alert{Let $\vec{F}$ be a force exerted on a system, which is displaced by a displacement $\vec{x}$.  The \textbf{work} done on the system is} \\
\alert{$W = \vec{F} \cdot \vec{x}$} \\
\end{tcolorbox}
The units of work are N m = kg m/s$^2$, or \textit{Joules}. \\
\vspace{0.5cm}
\small
\textit{Extra credit opportunity}: \textbf{Do you like beer}?  Write a 10-page paper on the on the scientific challenge faced by James Prescott Joule, who began to formulate the modern view of energy in the 19th century, contrary to \textit{caloric theory}.  \textbf{Upon completion of this assignment I will change two homework scores to perfect scores.}
\end{frame}

\begin{frame}{Definitions of Work}
Let $\theta$ be the angle between the force and the displacement.  Then this equation
\begin{equation}
W = \vec{F} \cdot \vec{x}
\end{equation}
becomes
\begin{equation}
W = Fx\cos\theta
\end{equation}
\end{frame}

\begin{frame}{Definitions of Work}
\begin{figure}
\centering
\includegraphics[width=0.7\textwidth]{figures/lawn.png}
\caption{\label{fig:work} A case where $\theta \neq 0$.}
\end{figure}
\end{frame}

\begin{frame}{Definitions of Work}
\begin{figure}
\centering
\includegraphics[width=0.7\textwidth]{figures/lawn2.png}
\caption{\label{fig:work2} (Left): A case where $x = 0$, so $W=0$.  (Right): A case where $\theta = 90^{\circ}$, so $W=0$.}
\end{figure}
\small
Just because an action requires \textit{energy} doesn't mean we are performing \textit{work}.  It requires muscular energy to hold up a heavy briefcase but this is not what we mean by work.  Work is about moving objects.
\end{frame}

\begin{frame}{Definitions of Work}
What about Newton's 3rd Law?  If one system $A$ exerts a force $F_{\rm AB}$ on a system $B$, then Newton's 3rd law states that system $B$ exerts a force $-F_{\rm AB}$ on system A. \\
\vspace{0.5cm}
If \alert{the work done by $A$ on $B$} is $W = (F_{\rm AB})x\cos\theta$, then \alert{the work done by $B$ on $A$} is $W = -(F_{\rm AB})x\cos\theta$. \\
\vspace{0.5cm}
In Fig. \ref{fig:work}, the work done by the man on the mower is positive, but the work done by the mower on the man is negative.
\end{frame}

\begin{frame}{Definitions of Work}
\small
More units of energy:
\begin{table}
\centering
\begin{tabular}{c | c | c}
\alert{Unit Name} & \alert{Definition} & \alert{Value} \\ \hline
electron-volt (eV) & energy of 1 e$^{-}$ through 1 V & $1.60\times 10^{-19}$ J \\ \hline
1 Rydberg (Rd) & ionize 1 hydrogen atom & $21.8\times 10^{-19}$ J \\ \hline
\textbf{Joule} & \textbf{1 N$\cdot$m} & \textbf{1.0 J} \\ \hline
foot-pound & 1 ft$\cdot$lb & 1.36 J \\ \hline 
calorie & Raise 1 gram of water 1$^{\circ}$ C & 4.184 J \\ \hline
British Thermal Unit & Raise 1 lb of ice to boil ($^{\circ}$F) & 1054.3 J \\ \hline
food calorie (kcal) & 1000 calories & 4184 J \\ \hline
kilowatt hours & 1 kilowatt system for 1 hr & $3.6\times 10^6$ J \\ \hline
gasoline galon equiv. & burning a galon of gas & $\approx 120 \times 10^6$ J \\ \hline
$E = mc^2$, 1 mole of H$^{+}$ & Rest mass (fusion/fission) & $9 \times 10^{13}$ J \\ \hline
\end{tabular}
\end{table}
\end{frame}

\begin{frame}{Definitions of Work}
\begin{figure}
\centering
\includegraphics[width=0.6\textwidth]{figures/integral.png}
\caption{\label{fig:work3} Breaking the displacement $\vec{x}$ into pieces, and summing them.}
\end{figure}
\small
This interpretation naturally leads to the subject of \textit{integration} in calculus.  
\end{frame}

\begin{frame}{Definitions of Work}
\begin{figure}
\centering
\includegraphics[width=0.4\textwidth]{figures/line.png}
\caption{\label{fig:line} Summary of the concept of the work integral.}
\end{figure}
\begin{equation}
W = \int_{AB} \vec{F} \cdot d\vec{r}
\end{equation}
\end{frame}

\begin{frame}{Survey - The insights and data (under analysis)}
\small
\begin{itemize}
\item \textbf{Group problem solving is a boost}
\begin{enumerate}
\item Must mix participation, someone knows how to start
\item Prof. Hanson make rounds
\item Show each and every step for iClicker and examples
\end{enumerate}
\item \textbf{Exams}
\begin{enumerate}
\item Review day
\item Practice problems and study guides are a boost
\item \alert{Similarity to homework, lecture?}
\end{enumerate}
\item \textbf{Lab activities} and \alert{Final Group Project}
\begin{enumerate}
\item Some want to drop them
\item Scientific process is important (don't just \textit{assume})
\item \textit{The goal is to blend them}
\end{enumerate}
\item \textbf{Office hours are crucial}
\end{itemize}
\end{frame}

\begin{frame}{Survey - The ...interesting bits...}
\small
\begin{itemize}
\item "You're not a bad person, you just suck at explaining things."
\item "The tests are not fair."
\item \textbf{NO PARTIAL CREDIT??}
\item "Everything should change except lab activities, and it's good that those are not graded."
\end{itemize}
My response: There is something called \textbf{The 80/20 Rule}...  Bottom line: we are here to develop and grow into professionals.
\end{frame}

\begin{frame}{Definitions of Work}
Consider the case where the force doing the work on the system of mass $m$ is friction:
\begin{equation}
W = \int_{AB} \vec{F} \cdot d\vec{r} = -\int_{AB} \mu_{\rm k} N dx = -\mu_{\rm k} m g \int_{AB} dx 
\end{equation}
\small
\begin{itemize}
\item Friction acts in opposite direction, so the dot product gives a minus sign
\item Friction acts along path $AB$ (whatever direction of motion is)
\end{itemize}
\end{frame}

\begin{frame}{Definitions of Work}
The driver of a 900 kg car slams on the breaks, and the tires slide on the pavement with $\mu_{\rm k} = 0.2$.  The initial speed is 25 m/s.  Assuming $g = 10$ m/s$^2$, how far does the car travel before coming to a stop?
\begin{itemize}
\item A: 312.5 m
\item B: 625 m
\item C: 31.25 m
\item D: 62.5 m
\end{itemize}
\end{frame}

\begin{frame}{Definitions of Work}
What is the work done on the car?
\begin{itemize}
\item A: 280 J
\item B: -280 J
\item C: 280 kJ
\item D: -280 kJ
\end{itemize}
\end{frame}

\begin{frame}{Definitions of Work}
Take your \textit{algebraic} answer for the breaking distance (displacement from two slides ago), and substitute it into the expression for the work done on the car (force of friction times displacement).  What do you get?  (\textit{Check your units}).
\begin{itemize}
\item A: $m v_{\rm i}^2$
\item B: $m g \frac{1}{2} m v_{\rm i}^2$
\item C: $-\frac{1}{2} m v_{\rm i}^2$
\item D: $\mu_{\rm k} m v_{\rm i}^2$
\end{itemize}
\textit{Keep this result in mind}...
\end{frame}

\begin{frame}{Definitions of Work}
Suppose we raise the 900 kg car by a displacement of 10 meters.  What is the work done on the car?
\begin{itemize}
\item A: 9 J
\item B: 90 J
\item C: 900 kJ
\item D: 90 kJ
\end{itemize}
\end{frame}

\begin{frame}{Definitions of Work}
What if we drop the car from a height of 10 m?  What is the final velocity of the car?
\begin{itemize}
\item A: $10\sqrt{2}$ m/s
\item B: $20$ m/s
\item C: $10$ m/s
\item D: $10\sqrt{10}$ m/s
\end{itemize}
\end{frame}

\begin{frame}{Definitions of Work}
Stand up, and show in your groups that the work done on the falling car is equal to $\frac{1}{2}m v_{\rm f}^2$, both numerically and algebraically.
\end{frame}

\section{Conclusion}

\begin{frame}{Week 6 Summary}
\begin{enumerate}
\item \alert{Work} has a scientifically precise definition
\begin{itemize}
\item Units
\item As a product of force and displacement vectors
\end{itemize}
\item Kinetic Energy and the \alert{Work-Energy Theorem}
\item Gravitational potential energy
\begin{itemize}
\item Potential energy
\item \textit{Simplifying otherwise complex calculations}
\item Potential energy near Earth's surface
\item ...in space
\end{itemize}
\item Definition of a \textbf{conservative force}
\begin{itemize}
\item Relationship between conservative forces and potential energy
\item Conservation of energy for conservative forces
\end{itemize}
\end{enumerate}
\end{frame}

\section{Answers}

\begin{frame}{Answers}
\begin{columns}[T]
\begin{column}{0.5\textwidth}
\begin{itemize}
\item 2.5 cm
\item 312.5 m
\item -280 kJ
\item $-\frac{1}{2} m v_{\rm i}^2$
\item 90 kJ
\item $10\sqrt{2}$ m/s
\end{itemize}
\end{column}
\begin{column}{0.5\textwidth}
\begin{itemize}
\item ... 
\end{itemize}
\end{column}
\end{columns}
\end{frame}

\end{document}

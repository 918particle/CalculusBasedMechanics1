\title{Answer-key 2 for Calculus-Based Physics-1: Mechanics (PHYS150-01)}
\author{Dr. Jordan Hanson - Whittier College Dept. of Physics and Astronomy}
\date{October 16th, 2017}
\documentclass[10pt]{article}
\usepackage[a4paper, total={18cm, 27cm}]{geometry}
\usepackage{outlines}
\usepackage[sfdefault]{FiraSans}
\usepackage{graphicx}

\begin{document}
\maketitle

\section{Vectors and Newton's Laws}
\begin{enumerate}
\item Let $\vec{F}_{\rm 1} = -\frac{3}{2}\hat{x} + 2\hat{y}$ N, and $\vec{F}_{\rm 2} = -2\hat{x} + \frac{3}{2}\hat{y}$ N.  a) Give the magnitude of each force.  b) What is the net force?  c) What is the angle between these two forces? N.  a) Give the magnitude of each force.  b) What is the net force?  c) What is the angle between these two forces? \\
a) $|\vec{F}_{\rm 1}| = \sqrt{\left(\frac{3}{2}\right)^2+\left(2\right)^2} = \sqrt{\frac{25}{4}} = \frac{5}{2}$.  The magnitude of the forces are equal. \\
b) $\vec{F}_{\rm 1}+\vec{F}_{\rm 2} = \left(\frac{3}{2}-2\right)\hat{i} + \left(2+\frac{3}{2}\right)\hat{j} = -\frac{1}{2}\hat{i} + \frac{7}{2}\hat{j}$. \\
c) $\vec{F}_{\rm 1} \cdot \vec{F}_{\rm 2} = 0$, so the cosine of the angle is $0$, meaning $\theta = 90$ degrees.
\item Imagine you are sitting in an airplane that has just lifted off with an acceleration vector 45 degrees with respect to horizontal.  Draw a free-body diagram corresponding to you, showing all forces acting on you.
\begin{figure}[ht]
\centering
\includegraphics[width=0.2\textwidth]{figures/FBD1.pdf}
\caption{\label{fig:fbd1} Answer to number 2.}
\end{figure}
\item Imagine you are riding a skateboard down a hill (no friction) and the incline angle is 45 degrees.  Draw a free-body diagram corresponding to you, showing all forces acting on you.
\begin{figure}[ht]
\centering
\includegraphics[width=0.2\textwidth]{figures/FBD2.pdf}
\caption{\label{fig:fbd2} Answer to number 3, which turns out to be similar to number 2.}
\end{figure}
\end{enumerate}
\clearpage
\section{Newton's Laws, and Circular Motion}
<<<<<<< HEAD
\begin{figure}[ht]
\centering
\includegraphics[width=0.2\textwidth,trim=10cm 4.84cm 0cm 0.6cm,clip=true]{figures/bank.png}
\caption{\label{fig:bank} Let the weight be $\vec{w}$, and the total lift be $\vec{L}$, which may be be broken into two components: the turning force (equal to centripetal force $\vec{f}_{\rm C}$) and vertical lift (which balances weight).}
\end{figure}
\begin{enumerate}
\item When banking, the free-body diagram of a jet-fighter resembles Fig. \ref{fig:bank}.  To bank while maintaining altitude, the lift force $\vec{L}$ must \textit{both} balance the weight $\vec{w}$ and provide the centripetal force $\vec{f}_{\rm C}$.  Let the mass of the aircraft be $m$, the radius of the turn be $r$, and the angle between $\vec{L}$ and horizontal be $\theta$.
\begin{itemize}
\item Show that the angular velocity of the turn, $\omega$, is $\omega = \sqrt{\frac{L\cos\theta}{rm}}$ \\
Turning force provides centripetal force: $L\cos\theta = mr\omega^2$.  Solve for $\omega$: $\omega = \sqrt{\frac{L\cos\theta}{rm}}$
\item If $\omega$ is the angular velocity, then the \textit{period} is $T = 2\pi/\omega$.  This is the time required to fly in a complete circle.  Show that one-half period is $\frac{T}{2} = \pi\sqrt{\frac{r m}{L\cos\theta}}$.  This is the time required to turn. \\
If $\omega = \sqrt{\frac{L\cos\theta}{rm}}$, then $2\pi/\omega = 2\pi \sqrt{\frac{rm}{L\cos\theta}}$, and half of that is $\frac{T}{2} = \pi\sqrt{\frac{r m}{L\cos\theta}}$.  This takes the aircraft half-way around a circle, or reversing course.
\item Let $L = 8\times 10^5$ N, $m = 2\times 10^4$ kg, $r = \frac{1}{2}$ km, and $\theta = 60$ degrees.  How long does it take the jet fighter to turn? \\
Plugging in the numbers, we get aboutr $5\pi$ seconds, or about 15.7 seconds.
\item What is the speed of the jet fighter?\\
If $v = r\omega$, then $v = \sqrt{\frac{rL\cos\theta}{m}}$.  Putting in numbers, we have $\approx 100$ m/s.
\begin{itemize}
\item A: 10 m/s 
\item B: 20 m/s
\item C: \textbf{100 m/s} (this is also a good estimate)
\item D: 120 m/s (this is a good estimate, but off by 20\%).
\end{itemize}
\end{itemize} 
\end{enumerate}
\section{Frictional Forces}
\begin{enumerate}
\item There is a spill of a mystery toxic liquid on a shop floor, and no one wants to touch it.  Someone gets the bright idea that they can identify it by the coefficient of kinetic friction and a steel plate.  Draw a free body diagram corresponding to a steel plate sliding along the liquid/floor, with friction decelerating it.\\
\begin{figure}[ht]
\centering
\includegraphics[width=0.2\textwidth]{figures/FBD3.pdf}
\caption{\label{fig:fbd3} Answer to number 1.}
\end{figure}
\item What is the coefficient of kinetic friction, $\mu_{\rm k}$, if a steel plate with an initial speed of 5 m/s comes to a stop after 2.5 seconds, assuming $g = 10$ m/s$^2$?  (Use the definition of acceleration $\Delta v/\Delta t = a$). \\ \\
The net force is the frictional force, $\vec{F}_{\rm Net} = m\vec{a} = \mu m g \hat{x}$, so $a = \mu g$.  Putting this into the definition of acceleration, we have $g^{-1} \Delta v/\Delta t = \mu$.  That's $10^{-1} (5/2.5) = 0.2$, so
\begin{itemize}
\item 0.1
\item \textbf{0.2}
\item 0.5
\item 1.2 This answer is larger than 1, so forbidden.
\end{itemize}
\item Suppose they get a sample of the mystery liquid in a vile.  They assume the drag force is given by Stoke's Law, $F_{\rm D} = 6\pi r \eta v$, where $v$ is the velocity of a particle moving through they fluid, $r$ is the radius of the particle, and $\eta$ is the \textit{viscosity}.  They drop a bead with $r = 1$ mm and a mass of one gram into the fluid, and observe the bead sink with a constant (terminal) velocity of 1 m/s.  What is the viscosity of the fluid?  Units: kg/(m s). \\ \\
The net force is zero, because the bead is moving at constant velocity.  This implies that the force of gravity is balanced by the force of drag from Stoke's Law.  $mg = 6\pi\eta r v$.  All parameters are measured, so we may calculate $\eta$: $\eta = mg/(6\pi r v)$.  Using $g = 10$, we find $\eta = 5/(3\pi)$ kg/(m s).
\begin{itemize}
\item $5/(3\pi)$ kg/(m s) (Correct).  Also, we can estimate our way to this one by noticing that it is closer to 1 than the other answers.  This liquid is less sticky than honey, but more sticky than blood, and way stickier than water.
\item $5/(30\pi)$ kg/(m s)
\item $10$ kg/(m s)
\item $5$ kg/(m s)
\end{itemize}
\end{enumerate}
\end{document}
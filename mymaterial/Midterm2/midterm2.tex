\title{Midterm 2 for Calculus-Based Physics-1: Mechanics (PHYS150-01)}
\author{Dr. Jordan Hanson - Whittier College Dept. of Physics and Astronomy}
\date{October 16th, 2017}
\documentclass[10pt]{article}
\usepackage[a4paper, total={18cm, 27cm}]{geometry}
\usepackage{outlines}
\usepackage[sfdefault]{FiraSans}
\usepackage{graphicx}

\begin{document}
\maketitle

\section{Vectors and Newton's Laws}
\begin{enumerate}
\item Let $\vec{F}_{\rm 1} = -\frac{3}{2}\hat{x} + 2\hat{y}$ N, and $\vec{F}_{\rm 2} = -2\hat{x} + \frac{3}{2}\hat{y}$ N.  What is the angle between these two vectors?  \vspace{1.5 cm}
\item Suppose that 
\item Imagine you are taking off in an airplane.  Which of the following lists all of the forces on your body? \\
\begin{itemize}
\item A: The acceleration pushes you back against the seat, and gravity pulls you down against the seat.
\item B: The acceleration pushes you back against the seat, gravity pulls you down against the seat, and there is a normal force balancing gravity.
\item C: The acceleration pushes you forward down the runway, and gravity pulls you down against the seat.
\item D: The acceleration pushes you forward down the runway, gravity pulls you down against the seat, and there is a normal force balancing gravity.
\end{itemize}
\item Now the airplane lifts off, still accelerating forward but also accelerating vertically.  Which of the following lists all the forces on your body? \\
\begin{itemize}
\item A: The acceleration pushes you back against the seat, and gravity pulls you down against the seat.
\item B: The acceleration pushes you forward down the runway, gravity pulls you down against the seat, and there is a normal force balancing gravity.
\item C: The acceleration pushes you forward down the runway, gravity pulls you down against the seat, and there is a normal force that is larger than gravity.
\item D: The forces must be the same as in question 1.
\end{itemize}
\end{enumerate}
\section{Newton's First, Second, and Third Law}
\begin{enumerate}
\item The captain of a pirate ship is attempting to board a merchant ship by sailing alongside it, and bring the ships closer together.  The relative velocity of the pirate ship to the merchant ship is $v(t) = (2-0.3t)$ m/s, with $t$ measured in seconds.  The pirate ship reaches the merchant ship in 7.0 seconds.  When it touches the merchant ship, does it exert any force on it? Why or why not? \vspace{3cm}
\item \vspace{3cm}
\end{enumerate}
\section{Vectors}
\begin{enumerate}
\item 
\end{enumerate}
\end{document}
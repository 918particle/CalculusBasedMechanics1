\title{Final Exam for Calculus-Based Physics-1: Mechanics (PHYS150-01)}
\author{Dr. Jordan Hanson - Whittier College Dept. of Physics and Astronomy}
\date{December 13th, 2017}
\documentclass[10pt]{article}
\usepackage[a4paper, total={18cm, 27cm}]{geometry}
\usepackage{outlines}
\usepackage[sfdefault]{FiraSans}
\usepackage{graphicx}

\begin{document}
\maketitle

\section{Conceptual Questions}
\subsection{Kinematics and Angular Kinematics}
\begin{enumerate}
\item An object accelerates with constant acceleration.  The displacement versus time curve is quadratic.  The acceleration versus time plot should be $\rule{1cm}{0.15mm}$ and the velocity versus time plot should be $\rule{1cm}{0.15mm}$.
\begin{itemize}
\item quadratic, linear
\item linear, flat
\item flat, linear
\item linear, quadratic
\end{itemize}
\item An object experiences constant \textit{angular} acceleration.  The net external torque is $\rule{1cm}{0.15mm}$, and the angular velocity is a $\rule{1cm}{0.15mm}$ function of time.
\begin{itemize}
\item zero, linear
\item constant, linear
\item zero, constant
\item constant, constant
\end{itemize}
\item A battleship fires simultaneously two shells with the same speed at enemy ships (Fig. \ref{fig:battle}).  If the shells follow the parabolic trajectories shown, which ship gets hit first?
\begin{itemize}
\item A, because it has a smaller displacement from the cannon.
\item A, because the overall distance travelled is less.
\item Both at the same time, because the initial projectile velocity is the same.
\item B, because the projectile does not have to travel as high in the air.
\item B, because the initial velocity must be higher.
\end{itemize}
\begin{figure}[hb]
\centering
\includegraphics[width=0.6\textwidth,trim=0cm 30cm 0cm 45cm,clip=true]{battle.jpeg}
\caption{\label{fig:battle} Which ship is hit first?}
\end{figure}
\end{enumerate}
\subsection{Forces and Torque}
\begin{enumerate}
\item An elevator contains a person standing on a scale.  The elevator accelerates downward, then moves at constant velocity, then comes to a stop.  The scale reads a weight that is $\rule{1cm}{0.15mm}$, then $\rule{1cm}{0.15mm}$, and then $\rule{1cm}{0.15mm}$ the person's actual weight.
\begin{itemize}
\item More than, equal to, less than
\item Less than, equal to, more than
\item equal to, equal to, equal to
\item More than, equal to, equal to
\end{itemize}
\item A crate is pushed across a floor at constant velocity against friction.  The crate is flipped so that a side with less surface area is on the bottom.  If the required force to push it increases, which of the following is the proper conclusion?
\begin{itemize}
\item It's harder to push because there's more pressure now: pressure is force divided by area.
\item It's harder to push because the new side must have a different coefficient of friction.
\item It's harder to push because the normal force has increased.
\end{itemize}
\item A man needs to pull a rusty lever to release a mechanism, but he can't.  Which of the following will increase torque on the lever?
\begin{itemize}
\item Tying a rope to the end of the lever, and pulling on the rope perpendicular to the lever.
\item Bolting a metal rod to the lever, and pulling the rod perpendicular to the lever.
\item Tying a rope to the end of the lever, pulling the rope parallel to the lever.
\item Bolting a metal rod to the lever, and pulling the rod parallel to the lever.
\end{itemize}
\item An aircraft is in a banked turn, traveling in a circle.  Which of the following is most correct?
\begin{itemize}
\item The craft experiences centripetal acceleration, provided by a component of the lift force.
\item The craft experiences centripetal acceleration, provided by the thrust, which is tangent to the circle.
\item Moving at constant velocity, the craft experiences no acceleration.
\end{itemize}
\end{enumerate}
\subsection{Work and Energy}
\begin{enumerate}
\item In which of the follow situations would energy \textit{not} be conserved?
\begin{itemize}
\item An object is dropped from some height and experiences free-fall, neglecting air-resistance.
\item An external force compresses a mass against an oscillator for a given displacement and then the mass is released.
\item A pendelum is pulled away from equilibrium and then released.
\item A train skids to a halt, with the wheels sliding on the tracks.
\end{itemize}
\item A force does an amount of work $W$ on an object with initial velocity $v$ to stop it.  How much work would have to be done on the object if the initial velocity were $2v$?
\begin{itemize}
\item $2W$
\item $3W$
\item $4W$
\end{itemize}
\end{enumerate}
\subsection{Linear and Angular Momentum}
\begin{enumerate}
\item When a star undergoes a supernova, matter is blown away by a fusion reaction.  The more significant effect for angular momentum is that the star shrinks in size.  Suppose the radius decreases by a factor of $10^2$.  By what factor does the angular velocity increase, if angular momentum is conserved? (Assume the mass doesn't change significantly).
\begin{itemize}
\item 10$^3$
\item 10$^4$
\item 10$^5$
\end{itemize}
\item A mine cart holding two robbers is moving along a track at constant speed.  They're being chased, so one robber dives out the back.  The speed of the cart
\begin{itemize}
\item increases, because momentum is conserved and the jumper has momentum in the opposite direction.
\item decreases, because momentum is conserved and the mass of the cart has decreased.
\item remains constant, because there were only internal forces, not external forces.
\end{itemize}
\item If ball 1 in the arrangement shown in Fig. \ref{fig:newton} is pulled back and then let go, ball 5 bounces forward with equal velocity.  If balls 1 and 2 are pulled back and released, balls 4 and 5 bounce forward with equal velocity, and so on.  The number of balls bouncing on each side is equal because
\begin{itemize}
\item of conservation of momentum.
\item the collisions are elastic.
\item the collisions are inelastic.
\item neither of the above.
\end{itemize}
\begin{figure}
\centering
\includegraphics[width=0.2\textwidth,trim=20cm 5cm 15cm 20cm,clip=true]{newton.jpeg}
\caption{\label{fig:newton} This object is known as a Newton's cradle.}
\end{figure}
\end{enumerate}
\section{Technical Questions}
\subsection{Kinematics and Angular Kinematics}
\begin{enumerate}
\item A ball is kicked with an initial velocity of $\vec{v} = 3\hat{i}+4\hat{j}$ m/s. (a) For how long does the ball remain in the air?  (b) Where does the ball land? ($g=10$ m/s$^2$). ($\frac{1}{3}$ point for correct diagram, $\frac{2}{3}$ point for numerical answers). \\ \vspace{1.5cm}
\end{enumerate}
\subsection{Forces and Torque}
\begin{enumerate}
\item A 900 kg lunar probe hovers above the surface of the Moon.  On the Moon, $g \approx 5/3$ m/s$^2$.  An engine is pointed at a 30 degree angle from straight down, spraying propellant.  What force does the engine produce to keep the probe from decreasing in height?  ($\frac{1}{3}$ point for correct free-body diagram, $\frac{2}{3}$ point for answer).  \\ \vspace{1.5cm}
\end{enumerate}
\subsection{Work and Energy}
\begin{enumerate}
\item A snowboarder descends a hill with a height of 25 meters (neglect friction).  (a) What is her final speed?  (b) After descending, she travels along a flat stretch of snow.  She turns the board sideways, the coefficient of friction becomes relevant: $\mu = 0.5$.  How far does she travel before stopping? \\ \vspace{1.5cm}
\end{enumerate}
\subsection{Linear and Angular Momentum}
\begin{enumerate}
\item Two objects each of mass $m = 0.2$ kg rotate around the origin of a coordinate system, both at radius $r = 0.2$ m.  If the tangential velocity of each is $v = 2$ m/s ($p = mv$), (a) what is $L = L_1 + L_2 = r_1 p_1\sin\theta_1+r_2 p_2\sin\theta_2$, the total angular momentum?  (b) What is the value of the total \textit{moment of inertia}, $I = 2mr^2$, and the \textit{angular speed} $\omega = v/r$ of the particles?  (c) Show numerically that $I\omega = L$ from part (a).
\begin{figure}[hb]
\centering
\includegraphics[width=0.2\textwidth]{rotate.pdf}
\end{figure}
\end{enumerate}
\end{document}
\title{Study Guide for Midterm 1}
\author{Dr. Jordan Hanson - Whittier College Dept. of Physics and Astronomy}
\date{\today}
\documentclass[10pt]{article}
\usepackage[a4paper, total={18cm, 27cm}]{geometry}
\usepackage{outlines}
\usepackage[sfdefault]{FiraSans}
\usepackage{graphicx}
\usepackage{amsmath}

\begin{document}
\maketitle

\section{Types of Problems on the Midterm}
\begin{itemize}
\item There are three problems on estimation, approximation, and unit analyis.  One of these involves kinematics, and another is multiple choice.
\item There are two problems on kinematics with constant acceleration.
\item There is one four-part problem on vectors.
\end{itemize}

\section{Estimation, Approximation, and Unit Analysis}
Remember our three techniques for establishing estimates:
\begin{itemize}
\item Choose the right \textit{scale}.  For example, don't work in meters if the distances are between planets.  Work in $AU$ or light-minutes.
\item Obtain complex quantities from simpler ones.  For example, estimate lengths first, and then multiply them to estimate volumes and areas.
\item Constrain the unknown with upper and lower limits.  For example, the height of a child should not be 10 meters, or 0.1 meters.
\end{itemize}

\section{Displacement, Velocity, and Constant Acceleration Vectors}
The definition of average velocity is
\begin{equation}
\bar{v} = \frac{x_{f} - x_{i}}{t_{f} - t_{i}} = \frac{\Delta x}{\Delta t}
\label{eq:1}
\end{equation}
If the velocity is constant, the average velocity and the instantaneous velocity are the same.  The numerator of Eq. \ref{eq:1} is in general a vector called \textit{the displacement}: $\Delta \vec{x}$, describing the change in position of something.  If the velocity is constant, then
\begin{equation}
\Delta \vec{x} = \vec{v}\Delta t
\label{eq:2}
\end{equation}
If the velocity is not constant, but the acceleration is constant, we have a system of equations relating displacement, time, velocity, and acceleration:
\begin{align}
x(t) &= x_0 + v_0 t + \frac{1}{2} a t^2 \\
v(t) &= v_0 + a t \\
a(t) &= a \\
v^2 &= v_0^2 + 2a\Delta x
\end{align}

\section{Vectors}
\begin{itemize}
\item Practice adding and subtracting vectors.
\begin{enumerate}
\item Graphically
\item As lists of numbers
\end{enumerate}
\item Practice breaking vectors into components: $\hat{x} \cdot \vec{v} = v\cos\theta$, $\hat{y} \cdot \vec{v}=v\sin\theta$ (if $\theta$ goes from x-axis to vector).
\item Practice taking the magnitude of a vector: $\sqrt{\vec{v}\cdot\vec{v}}$, or Pythagorean theorem.
\end{itemize}
\end{document}
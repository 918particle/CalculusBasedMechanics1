\title{Syllabus for Calculus-Based Physics-1: Mechanics (PHYS150-01)}
\author{Dr. Jordan Hanson - Whittier College Dept. of Physics and Astronomy}
\date{\today}
\documentclass[10pt]{article}
\usepackage[a4paper, total={18cm, 27cm}]{geometry}
\usepackage{outlines}
\usepackage[sfdefault]{FiraSans}

\begin{document}
\maketitle

\begin{abstract}
The concepts of calculus-based mechanics will be presented within the context of interactive problem-solving.  First, the concepts of displacement, velocity, and acceleration in one and two dimensions will be introduced, building up to Newton's Laws of motion.  Next, the concepts of friction and rotational motion will be added.  More complex problems will be introduced through the conservation of energy and linear momentum, followed by the rotational equivalents.  The course work will include interactive computational exercises, analytic textbook problems, and lab-based activities.
\end{abstract}

\begin{outline}[enumerate]
\1 Week 1 - September 6th through September 8th - \textbf{Chapters 2.1-2.3, 3.1}
\2 Adding and subtracting vectors
\2 Displacement and translational motion
\2 Time, and the concept of relativity
\1 Week 2 - September 11th through September 15th - \textbf{Chapters 3.2-3.6}
\2 Instantaneous velocity and acceleration
\3 Review: single-variable differentiation
\2 Acceration due to gravity near the Earth's surface and other constant accelerations
\2 Common equations of motion for constant acceleration	
\1 Week 3 - September 18th through September 22nd - \textbf{Chapters 3.2-3.6}
\2 
\end{outline}

\end{document}
\title{Study Guide for Calculus-Based Physics-1: Mechanics (PHYS150-01)}
\author{Dr. Jordan Hanson - Whittier College Dept. of Physics and Astronomy}
\date{November 27th, 2017}
\documentclass[10pt]{article}
\usepackage[a4paper, total={18cm, 27cm}]{geometry}
\usepackage{outlines}
\usepackage[sfdefault]{FiraSans}
\usepackage{graphicx}

\begin{document}
\maketitle

\section{Definition of Momentum}
\begin{enumerate}
\item Remember that the units of momentum are \textbf{kg m/s}.
\item There are two requirements for momentum to be conserved.  First, there must be \textit{no net external force}.  Second, the \textit{masses of the particles cannot change}.
\item In vector form, the momentum is $\vec{p} = m \vec{v}$.
\end{enumerate}
\section{Conservation of Momentum}
\begin{enumerate}
\item If momentum is conserved, the total initial momentum is equal to the total final momentum.  In vector form, $\vec{P}_{\rm f} = \vec{P}_{\rm i}$.  If the interaction is a $2 \rightarrow 2$ type interaction, then $\vec{p}_{\rm 1} + \vec{p}_{\rm 2} = \vec{p}_{\rm 1}' + \vec{p}_{\rm 2}'$, where the prime notation indicates the final state.
\item Another way to state the conservation of momentum is $\frac{d\vec{p}}{dt} = 0$.  Since force is the derivative of momentum, this is equivalent to saying that there is no net external force.
\end{enumerate}
\section{Classifying Interactions}
\begin{enumerate}
\item An \textit{elastic interaction} is one in which kinetic energy is conserved as well as momentum.
\item An \textit{inelastic interaction} is one in which kinetic energy is not conserved, but momentum is still conserved.
\item \textit{Elastic interactions} are usually of type $n \rightarrow n$, and \textit{inelastic interactions} are usually of type $n \rightarrow 1$.
\item \textbf{Example problem}: A 1 kg particle has $v_{\rm 1} = -1$ m/s, and it interacts with a 1 kg particle with velocity $v_{\rm 2} = 1$ m/s.  If the collision is elastic, what is the final velocity of each particle? \\ \\
\textit{Momentum is conserved, so let's write}: \\ $\vec{p}_{\rm 1} + \vec{p}_{\rm 2} = \vec{p}_{\rm 1}' + \vec{p}_{\rm 2}'$ \\ $m\vec{v}_{\rm 1} + m\vec{v}_{\rm 2} = m\vec{v}_{\rm 1}' + m\vec{v}_{\rm 2}'$ \\ $-1 + 1 = 0 = \vec{v}_{\rm 1}' + \vec{v}_{\rm 2}'$ \\ $\vec{v}_{\rm 1}' = -\vec{v}_{\rm 2}'$ \\ We cannot procede without adding additional facts.  However, we know that this is an elastic interaction, meaning that kinetic energy is conserved: \\
$\frac{1}{2}m v_{\rm 1}^2 + \frac{1}{2}m v_{\rm 2}^2 = \frac{1	}{2}mv_{\rm 1}'^2 + \frac{1}{2}mv_{\rm 2}'^2$ \\ $v_{\rm 1}^2 + v_{\rm 2}^2 = v_{\rm 1}'^2 + v_{\rm 2}'^2$ \\ $2 = v_{\rm 1}'^2 + v_{\rm 2}'^2$ \\ Putting in the conclusion from momentum conservation, that the final velocities are equal and opposite, we find that the magnitude of each is 1 m/s.  Thus, $v_{\rm 1}' = 1$ m/s and $v_{\rm 2}' = -1$ m/s. \\
\textbf{This is intuitive, right?  They have equal mass and equal speed so they just bounce off of each other and go the other way}.
\end{enumerate}
\end{document}
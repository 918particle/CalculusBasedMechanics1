\documentclass{beamer}
\usetheme{metropolis}
\usepackage{graphicx}
\usepackage{subfig}
\usepackage{hyperref}
\usepackage{tcolorbox}
\title{Calculus-Based Physics-1: Mechanics (PHYS150-01): Week 8}
\date{October 23rd - October 27th, 2017}
\author{Jordan Hanson}
\institute{Whittier College Department of Physics and Astronomy}

\begin{document}
\maketitle

\section{Week 7 Review}

\begin{frame}{Week 7 Summary}
\begin{enumerate}
\item \alert{Work} and \alert{potential energy}
\begin{itemize}
\item \textbf{Lab activity:} Oscillator and gravity trading work and potential energy
\end{itemize}
\item Potential energy and \textbf{conservative forces}
\item \alert{\textbf{Conservation of Energy}}
\begin{itemize}
\item \textit{Calculus review: the fundamental theorem of calculus}
\item Graphical representations of integrals and energy
\end{itemize}
\end{enumerate}
\end{frame}

\section{Week 7 Review Problem}

\begin{frame}{Week 6 Review Problem}
Suppose a particle moves in a potential energy surface $U(x) = U_{\rm 0} \left( x^4 - x^2 \right)$.  What is the force at $x=1$?
\begin{itemize}
\item A: $-4U_{\rm 0} + 2U_{\rm 1}$
\item B: $4U_{\rm 0} - 2U_{\rm 1}$
\item C: $-U_{\rm 0} + U_{\rm 1}$
\item D: $0$
\end{itemize}
\end{frame}

\section{Week 8 Summary}

\begin{frame}{Week 8 Summary}
\begin{enumerate}
\item Definition of \alert{\textbf{momentum}}
\item \alert{\textbf{\textit{Conservation of momentum}}}
\begin{itemize}
\item The proof and the assumptions
\item Examples
\end{itemize}
\item \alert{\textbf{\textit{Classification of collisions}}}
\begin{itemize}
\item Elastic
\item Inelastic
\item $1 \rightarrow 1$, $1 \rightarrow n$, $n \rightarrow 1$, $n \rightarrow n$
\item \textbf{Lab activity}
\end{itemize}
\item \textbf{Momentum in multiple dimensions}
\item \textbf{Center of mass}
\begin{itemize}
\item Derivation of $\vec{F}_{\rm Net} = \frac{d\vec{P}_{\rm CM}}{dt}$
\item Center of mass motion
\end{itemize}
\end{enumerate}
\end{frame}

\section{Definition of Momentum}

\begin{frame}{Definition of momentum}
\textit{Ready to jump down the rabbit hole?  Good.}  Momentum is defined as follows: \\ \vspace{1cm}
\begin{tcolorbox}[colback=white,colframe=red!40!blue,title=Definition of Momentum]
\alert{A particle of mass $m$ and velocity $\vec{v}$ has the vector \textit{momentum}:} \\ \\
\alert{$\vec{p} = m\vec{v}$}
\end{tcolorbox}
\end{frame}

\begin{frame}{Definition of momentum}
There is a corollary: \\ \vspace{1cm}
\begin{tcolorbox}[colback=white,colframe=red!40!blue,title=Newton's Second Law with momentum]
\alert{If a particle has acceleration $\vec{a} = \frac{d\vec{v}}{dt}$, then} \\ \\
\alert{$\vec{F}_{\rm Net} = \frac{d\vec{p}}{dt}$}
\end{tcolorbox}
\end{frame}

\begin{frame}{Definition of momentum}
An object that has a small mass and an object that has a large mass have the same momentum. Which mass has the largest kinetic energy?
\begin{itemize}
\item A: The one with the small mass
\item B: The one with the large mass
\item C: If the momentum is the same the kinetic energy is the same
\item D: Cannot determine the answer
\end{itemize}
\end{frame}

\begin{frame}{Definition of momentum}
An object that has a small mass and an object that has a large mass have the same kinetic energy. Which mass has the largest momentum?
\begin{itemize}
\item A: The one with the small mass
\item B: The one with the large mass
\item C: If the momentum is the same the kinetic energy is the same
\item D: Cannot determine the answer
\end{itemize}
\end{frame}

\begin{frame}{Definition of momentum}
The unit of linear momentum is kg m/s.  Suppose that a raindrop reaches a terminal velocity of 1 m/s, and the density of water is 1 gram per cm$^3$.  If a 1 cm$^3$ water droplet reaches terminal velocity, what is the momentum of the droplet?
\begin{itemize}
\item A: $10^{-3}$ kg m/s
\item B: $10^{-2}$ kg m/s
\item C: $10^{-1}$ kg m/s
\item D: $1$ kg m/s
\end{itemize}
\end{frame}

\begin{frame}{Definition of momentum}
If $\vec{F}_{\rm Net} = \frac{d\vec{p}}{dt}$, and an object is undergoing constant acceleration, which of the following is true of the momentum?
\begin{itemize}
\item A: It is constant in time.
\item B: It is a linear function of time.
\item C: It is a quadratic function of time.
\item D: It is zero.
\end{itemize}
\end{frame}

\section{Conservation of Momentum}

\begin{frame}{Conservation of Momentum}
\textit{...continuing to fall down the rabbit hole...} \\ \vspace{1cm}
\begin{tcolorbox}[colback=white,colframe=red!40!blue,title=Conservation of Momentum]
\alert{The momentum of a system of $N$ particles undergoing no external forces is conserved.} \\ \\
\alert{$\frac{d\vec{P}}{dt}=0$}
\end{tcolorbox}
\end{frame}

\begin{frame}{Conservation of Momentum}
Suppose two objects with momenta $\vec{p}_{\rm 1} = m_{\rm 1}\vec{v}_{\rm 1}$ and $\vec{p}_{\rm 2} = m_{\rm 2}\vec{v}_{\rm 2}$ collide. The new momenta after the collision are $\vec{p}'_{\rm 1} = m_{\rm 1}\vec{v}'_{\rm 1}$ and $\vec{p}'_{\rm 2} = m_{\rm 2}\vec{v}'_{\rm 2}$.  If $m_{\rm 2} = 2 m_{\rm 1}$ and $\vec{v}_{\rm 1} = 2\vec{v}_{\rm 2}$, and $\vec{v}'_{\rm 2}$ is observed to equal $\vec{v}_{\rm 1}$, what is $\vec{v}'_{\rm 1}$? \\ \vspace{1cm}
\textbf{Solve together in groups on boards.}
\end{frame}

\begin{frame}{Conservation of Momentum}
\small
The proof of conservation of momentum is the combination of two concepts: \alert{\textbf{Newton's 3rd Law}} and \alert{\textbf{Newton's 2nd Law}}.  The net forces on two particles by Newton's 3rd Law are 
\begin{equation}
\vec{F}_{\rm 21} = -\vec{F}_{\rm 12}
\end{equation}
Substituting Newton's 2nd Law for the forces,
\begin{equation}
m_{\rm 1} \vec{a}_{\rm 1} = -m_{\rm 2}\vec{a}_{\rm 2}
\end{equation}
Acceleration is defined as the change in velocity, implying
\begin{align}
m_{\rm 1} \frac{d\vec{v}_{\rm 1}}{dt} &= -m_{\rm 2} \frac{d\vec{v}_{\rm 2}}{dt} \\
\frac{d\vec{p}_{\rm 1}}{dt} &= -\frac{d\vec{p}_{\rm 2}}{dt} \\
\frac{d\vec{p}_{\rm 1}}{dt} + \frac{d\vec{p}_{\rm 2}}{dt} &= 0 \\
\frac{d}{dt}\left(\vec{p}_{\rm 1} + \vec{p}_{\rm 2}\right) &= 0
\end{align}
\end{frame}

\begin{frame}{Conservation of Momentum}
\begin{equation}
\boxed{
\frac{d}{dt}\left(\vec{p}_{\rm 1} + \vec{p}_{\rm 2}\right) = 0
} \label{eq:momentum}
\end{equation}
Equation \ref{eq:momentum} states that the total momentum does not change with time.  The assumptions hold even if there are more than two particles, for every particle in the system exerts some force on every other particle, even if that force is zero.
\begin{equation}
\frac{d}{dt}\sum_{\rm i} \vec{p}_{\rm i} = \frac{d\vec{P}}{dt} = 0
\end{equation}
\end{frame}

\begin{frame}{Conservation of Momentum}
What two assumptions were necessary in the above proof?
\begin{itemize}
\item A: Each mass is constant in time, and the total velocity is zero.
\item B: The total velocity is zero, and the net force is $\frac{dP}{dt}$.
\item C: The net external force is zero, and the total velocity is zero.
\item D: The net external force is zero, and the mass of each particle is zero.
\end{itemize}
\end{frame}

\begin{frame}{Conservation of Momentum}
\begin{figure}
\centering
\includegraphics[width=0.7\textwidth]{figures/cars.png}
\caption{\label{fig:cars} The assumptions for momentum conservation.  (a) Which car exerts more force? (b) At which point are the cars accelerating?}
\end{figure}
\end{frame}

\begin{frame}{Conservation of Momentum}
\begin{figure}
\centering
\includegraphics[width=0.8\textwidth]{figures/carts.png}
\caption{\label{fig:carts} The laboratory activity for today.}
\end{figure}
\small
\begin{itemize}
\item One cart still, other moving (magnet side)
\item One cart still, other moving (velcro side)
\item Both carts moving (magnet side)
\item Both carts moving (velcro side)
\end{itemize}
\end{frame}

\begin{frame}{Conservation of Momentum}
\small
\textbf{Test cases from lab:}
\begin{itemize}
\item One cart still, other moving (magnet side)
\item One cart still, other moving (velcro side)
\item Both carts moving (magnet side)
\item Both carts moving (velcro side)
\end{itemize}
\textbf{Answer the following questions:}
\begin{itemize}
\item In which of the above is the momentum conserved?
\item In which of the above is the kinetic energy conserved?
\item In which of the above is both the kinetic energy and momentum conserved?
\item In which of the above is the final kinetic energy zero?
\end{itemize}
\end{frame}

\begin{frame}{Conservation of Momentum}
\begin{figure}
\centering
\includegraphics[width=0.6\textwidth]{figures/Flow.pdf}
\caption{\label{fig:flow} Classification of momentum interactions.}
\end{figure}
\end{frame}

\begin{frame}{Conservation of Momentum}
\textit{\alert{Special case of an} \textbf{explosion}...}
\url{https://www.youtube.com/watch?v=5zxVQBnmyDA}
\end{frame}

\begin{frame}{Conservation of Momentum}
\begin{figure}
\centering
\includegraphics[width=0.75\textwidth]{figures/alpha.png}
\caption{\label{fig:alpha} Knowledge of the atom via momentum conservation!}
\end{figure}
\end{frame}

\section{Elastic Collisions}

\begin{frame}{Elastic Collisions}
\begin{figure}
\centering
\includegraphics[width=0.55\textwidth]{figures/elastic.png}
\caption{\label{fig:elastic} \small Internal kinetic energy and momentum are conserved if the collision is \textit{elastic}.}
\end{figure}
\end{frame}

\begin{frame}{Elastic Collisions}
Suppose two objects undero an elastic collision.  Given the conditions below, find the quadratic equation for $v_{\rm 1}'$.
\begin{align}
m_{\rm 1} &= 0.5 kg \\
m_{\rm 2} &= 1.0 kg \\
v_{\rm 1} &= 2 m/s \\
v_{\rm 2} &= 0 m/s
\end{align}
\textit{\textbf{Solve in groups on boards.}}
\end{frame}

\begin{frame}{Elastic Collisions}
Suppose two objects undero an elastic collision.  Given the conditions below, find the quadratic equation for $v_{\rm 1}'$. \\ \vspace{0.5cm}
\textit{Answer:} $\frac{3}{2}v_{\rm 1}'^2 - 2v_{\rm 1}' - 2 = 0$ \\
\vspace{0.5cm}
Which root of this equation is correct, and why?
\end{frame}

\section{Inlastic Collisions}

\begin{frame}{Inlastic Collisions}
\begin{figure}
\centering
\includegraphics[width=0.9\textwidth]{figures/inelastic.png}
\caption{\label{fig:inelastic} Internal kinetic energy is not conserved, and momentum is conserved if the collision is \textit{inelastic}.}
\end{figure}
\end{frame}

\begin{frame}{Inlastic Collisions}
Suppose two objects undero an inelastic collision.  Given the conditions below, solve for the final velocity.
\begin{align}
m_{\rm 1} &= 0.5 kg \\
m_{\rm 2} &= 1.0 kg \\
v_{\rm 1} &= 2 m/s \\
v_{\rm 2} &= 0 m/s
\end{align}
\textit{\textbf{Solve in groups on boards.}}
\end{frame}

\begin{frame}{Inelastic Collisions}
What is the right answer? \\ \vspace{0.5cm}
$v' = \frac{2}{3}$ m/s.  Should it be positive or negative?
\end{frame}

\section{Momentum Conservation in Two Dimensions}

\begin{frame}{Momentum Conservation in Two Dimensions}
\small
A particle interacts with another at rest.  The incoming particle has mass $1$ kg, and moves parallel to the x-axis at 1 m/s.  In the final state, it moves at $\theta_{\rm 1} = 60$ degrees with respect to the x-axis.  The second particle, with mass $0.5$ kg, moves at $\theta_{\rm 2} = 30$ degrees with respect to the x-axis.  Solve for the final velocities. \\ \vspace{0.5cm}
\begin{figure}
\centering
\includegraphics[width=0.5\textwidth]{figures/2d_1.pdf}
\caption{\label{fig:2d_1} \textbf{Solve as a group on the board.}  There are two unknowns, \alert{\textit{but also two equations}.}}
\end{figure}
\end{frame}

\begin{frame}{Momentum Conservation in Two Dimensions}
\small
A particle interacts with another at rest.  The incoming particle has mass $1$ kg, and moves parallel to the x-axis at 1 m/s.  In the final state, it moves at $\theta_{\rm 1} = 60$ degrees with respect to the x-axis.  The second particle, with mass $0.5$ kg, moves at $\theta_{\rm 2} = 30$ degrees with respect to the x-axis.  Solve for the final velocities. \\ \vspace{0.5cm}
\textit{Answers:} $v_{\rm 1}' = \frac{1}{2}$ m/s, and $v_{\rm 2}' = \sqrt{3}$ m/s. \\ \vspace{0.5cm}
Does this answer make sense?  Why should the second particle move faster than the first?
\end{frame}

\begin{frame}{Momentum Conservation in Two Dimensions}
\small
A particle interacts with another at rest.  The incoming particle has mass $1$ kg, and moves parallel to the x-axis at 1 m/s.  In the final state, it moves at $\theta_{\rm 1} = 60$ degrees with respect to the x-axis.  The second particle, with mass $0.5$ kg, moves at $\theta_{\rm 2} = 30$ degrees with respect to the x-axis.  Solve for the final velocities. \\ \vspace{0.5cm}
\textit{Answers:} $v_{\rm 1}' = \frac{1}{2}$ m/s, and $v_{\rm 2}' = \sqrt{3}$ m/s. \\ \vspace{0.5cm}
Was this interaction elastic or not?  Why?  \textit{Answer: the kinetic energy actually increases.  This is only possible if some \textbf{internal} energy is released.}
\end{frame}

\begin{frame}{Momentum Conservation in Two Dimensions}
\small
A particle interacts with another.  The first has mass $1$ kg, and moves at 60 degrees with respect to the x-axis at $\frac{1}{2}$ m/s.  The second moves at $\sqrt{3}$ m/s, at 30 degrees with respect to the x-axis and has mass 0.5 kg.  Solve for the final velocity of the combined object if they stick together. \\ \vspace{0.5cm}
\begin{figure}
\centering
\includegraphics[width=0.5\textwidth]{figures/2d_2.pdf}
\caption{\label{fig:2d_2} \textbf{Solve as a group on the board.}}
\end{figure}
\end{frame}

\begin{frame}{Momentum Conservation in Two Dimensions}
\small
A particle interacts with another.  The first has mass $1$ kg, and moves at 60 degrees with respect to the x-axis at $\frac{1}{2}$ m/s.  The second moves at $\sqrt{3}$ m/s, at 30 degrees with respect to the x-axis and has mass 0.5 kg.  Solve for the final velocity of the combined object if they stick together. \\ \vspace{0.5cm}
\textit{Answer:} $v' = \frac{2}{3}$ m/s. \\ \vspace{0.5cm}
Does this answer make sense?  Should it be larger or smaller than the first two velocities?
\end{frame}

\begin{frame}{Momentum Conservation in Two Dimensions}
\small
A particle interacts with another.  The first has mass $1$ kg, and moves at 60 degrees with respect to the x-axis at $\frac{1}{2}$ m/s.  The second moves at $\sqrt{3}$ m/s, at 30 degrees with respect to the x-axis and has mass 0.5 kg.  Solve for the final velocity of the combined object if they stick together. \\ \vspace{0.5cm}
\textit{Answer:} $v' = \frac{2}{3}$ m/s. \\ \vspace{0.5cm}
Was this an inelastic or elastic collision?  Why?
\end{frame}

\begin{frame}{Momentum Conservation in Two Dimensions}
\small
A particle interacts with another.  The first has mass $1$ kg, and moves at 60 degrees with respect to the x-axis at $\frac{1}{2}$ m/s.  The second moves at $\sqrt{3}$ m/s, at 30 degrees with respect to the x-axis and has mass 0.5 kg.  Solve for the final velocity of the combined object if they stick together. \\ \vspace{0.5cm}
\textit{Answer:} $v' = \frac{2}{3}$ m/s. \\ \vspace{0.5cm}
Why does this final velocity not equal the initial velocity of Fig. \ref{fig:2d_1}?
\end{frame}

\section{Conclusion}

\begin{frame}{Week 8 Summary}
\begin{enumerate}
\item Definition of \alert{\textbf{momentum}}
\item \alert{\textbf{\textit{Conservation of momentum}}}
\begin{itemize}
\item The proof and the assumptions
\item Examples
\end{itemize}
\item \alert{\textbf{\textit{Classification of collisions}}}
\begin{itemize}
\item Elastic
\item Inelastic
\item $1 \rightarrow 1$, $1 \rightarrow n$, $n \rightarrow 1$, $n \rightarrow n$
\item \textbf{Lab activity}
\end{itemize}
\item \textbf{Momentum in multiple dimensions}
\item \textbf{Center of mass}
\begin{itemize}
\item Derivation of $\vec{F}_{\rm Net} = \frac{d\vec{P}_{\rm CM}}{dt}$
\item Center of mass motion
\end{itemize}
\end{enumerate}
\end{frame}

\section{Answers}

\begin{frame}{Answers}
\begin{columns}[T]
\begin{column}{0.5\textwidth}
\begin{itemize}
\item $-4U_{\rm 0} + 2U_{\rm 1}$
\item The one with the small mass
\item The one with the large mass
\item $10^{-2}$ kg m/s
\item It is a linear function of time.
\item The net external force is zero, and the mass of each particle is zero.
\end{itemize}
\end{column}
\begin{column}{0.5\textwidth}
\begin{itemize}
\item ...
\end{itemize}
\end{column}
\end{columns}
\end{frame}

\end{document}
